\subsection[Tricky Basics]{Tricky Basics}

\begin{frame}[fragile,t]
\frametitle{Keyword Typename}
\framesubtitle{From Vandevoorde \& Jossutis}
Consider:

{\scriptsize
\begin{verbatim}
template<class T> 
class MyTemplateClass
{
  ...

  typename T::subclass * ptr;
};
\end{verbatim}
}
\begin{itemize}[<+->]
\item The second \texttt{typename} helps the compiler by
  disambiguating the construct
\item Without it, 
{\scriptsize
\begin{verbatim}
           T::subclass * ptr;
\end{verbatim}
}
looks like a static member that you're trying to multiply by something.
\item More generally, use \texttt{typename} when a construct involving
  the template parameter is supposed to produce a type.
\end{itemize}
\pause
\begin{center}
We'll come back to this in a moment; the rule is simple, the reasons
less so
\end{center}
\end{frame}


%%%%%%%%%%%%%%%%%%%%%%%%%%%%%%%%%%%%%%%%%%%%%%%%%%%%%%%%%%%%%%%%%%%%%%
\begin{frame}[fragile,t]
\frametitle{The .template construct}
Similarly:
{\scriptsize
\begin{verbatim}
template<int N> void printBitset (const std::bitset<N>& bs)
{
  // std::bitset has a member template:
  // template <class charT, class traits, class Alloc>
  // basic_string<charT,traits,Alloc> std::bitset<T,N>::to_string() const;

  bs.to_string<char, char_traits<char>, allocator<char>>(); // ERROR

  bs.template to_string< ... >; // OK
}
\end{verbatim}
}
\begin{itemize}[<+->]
\item Without the wierd \texttt{.template} thing
{\scriptsize
\begin{verbatim}
  bs.to_string<char, char_traits<char>, allocator<char>>();
  // parses as:
  (bs.to_string) < (other stuff...)
  // that's a less-than operator!
\end{verbatim}
}
\item use \texttt{.template} (and, sometimes, \texttt{->template}) only
  in front of an explicit template reference of something that depends
  on the template argument.
\item In real life, I ignore these until I get wierd error messages, and
then pull out the book.

\end{itemize}
\end{frame}

%%%%%%%%%%%%%%%%%%%%%%%%%%%%%%%%%%%%%%%%%%%%%%%%%%%%%%%%%%%%%%%%%%%%%%
\begin{frame}[fragile,t]
\frametitle{Using this->}

\begin{itemize}[<+->]
\item for class templates with base classes, inherited names must be
  qualified:
{\scriptsize
\begin{verbatim}
template<typename T> struct Base {
  int m;
  void exit(); 
};

template<typename T> struct Derived : Base<T> {
  void foo() {
    m=3;        // error, 'm' not known
    exit();     // calls some external 'exit'

    this->m = 3;     // OK
    this->exit();    // Yup
  }
};
\end{verbatim}
}

\item Annoying, but necessary, because the compiler doesn't know if
  \texttt{Base<T>} is specialized, can't assume  it knows what those
  names mean.

\end{itemize}

\end{frame}


\subsection {Names in Templates}
%%%%%%%%%%%%%%%%%%%%%%%%%%%%%%%%%%%%%%%%%%%%%%%%%%%%%%%%%%%%%%%%%%%%%%
\begin{frame}[fragile,t]
\frametitle{Name Taxonomy}
\framesubtitle{What's going on here?}
Two kinds of names in templates:

\begin{itemize}[<+->]
\item {\bf Dependent Names} are those that depend on a template
  parameter: \texttt{std::vector<T>::iterator} depends on T for definition.
\item {\bf Non-Dependent Names} are those that ... don't.
\item Non-dependent symbols are resolved at the point of the template
  \emph{definition}
\item Dependent symbols are resolved at the point of the template \emph{instantiation}
 \begin{itemize}
  \item Consider: what if \texttt{vector<bool>} is specialized, and
    \texttt{iterator} is now a function or something?
  \end{itemize}
\item This is the dreaded ``Two Phase Lookup'' rule.
\end{itemize}

\end{frame}


%%%%%%%%%%%%%%%%%%%%%%%%%%%%%%%%%%%%%%%%%%%%%%%%%%%%%%%%%%%%%%%%%%%%%%
\begin{frame}[fragile,t]
\frametitle{Name Taxonomy}
\framesubtitle{Dependent Name Issues}
{\scriptsize
\begin{verbatim}
template<typename T> struct Trap { enum { x };};
// (1) x is not a type here

template<typename T> struct Victim
{
  int y;
  void poof()  {  Trap<T>::x*y;  }  
                  // (2) declaration or multiplication?
};

// evil specialization
template<> struct Trap<void> {typedef int x};
// (3) x is a type here

void boom(Victim<void>& bomb)  {bomb.poof(); }
\end{verbatim}
}


\begin{itemize}[<+->]
\item When parsing line 2 the compiler must decide what it's seeing.
\item It \emph{cannot} just look in Trap to find out, because it
  doesn't know about the specialization
\item General rule: the compiler has to wait as long as possible
  before binding names!
\item Therefore we have to help out.
\end{itemize}
\end{frame}

%%%%%%%%%%%%%%%%%%%%%%%%%%%%%%%%%%%%%%%%%%%%%%%%%%%%%%%%%%%%%%%%%%%%%%
\begin{frame}[fragile,t]
\frametitle{Name Taxonomy}
\framesubtitle{Dependent Name Issues II}
Similarly
{\scriptsize
\begin{verbatim}
template<typename T> struct Shell {
  template<int N> struct In {
    template<int M> struct Deep {
      virtual void f();
   };
 };
};

template<typename T, int N> struct Wierd {
  void case1(Shell<T>::template In<N>::template Deep<N> p {
    p.template Deep<N>::f();        // inhibit virtual fn call
    (&p)->template Deep<N>::f();
  }
};
\end{verbatim}
}

\begin{itemize}[<+->]

\item The extra \texttt{.template}s keep 
{\scriptsize
\begin{verbatim}
  p.Deep< N >::f()
\end{verbatim}
}
from begin parsed as
{\scriptsize
\begin{verbatim}
((p.Deep)<N)>::f()
\end{verbatim}
}

\end{itemize}

\end{frame}

%%%%%%%%%%%%%%%%%%%%%%%%%%%%%%%%%%%%%%%%%%%%%%%%%%%%%%%%%%%%%%%%%%%%%%
\begin{frame}[fragile,t]
\frametitle{Derivation and Class Templates}
\framesubtitle{Nondependent Base Class}
A nondependent base is one that doesn't depend on template arguments.
{\scriptsize
\begin{verbatim}
template<typename X> struct Base {
  int basefield;
  typedef int T;
};

class D1 : public Base<Base<void>> {  // not a template case
  void foo() { basefield = 3;}   // normal
  // Base::T is int, obviously
};

template<typename T> class D2 : public Base<double> {
  void foo() { basefield = 3; }   // sure
  T strange;  // T is Base<double>::T, not the template param
};

\end{verbatim}
}

This, because nondependent bases are considered before the list of
template parameters...

\end{frame}


%%%%%%%%%%%%%%%%%%%%%%%%%%%%%%%%%%%%%%%%%%%%%%%%%%%%%%%%%%%%%%%%%%%%%%
\begin{frame}[fragile,t]
\frametitle{Derivation and Class Templates}
\framesubtitle{Dependent Base Class}
A Dependent base is one that depends on the template parameter
{\scriptsize
\begin{verbatim}
template<typename X> struct Base {
  int basefield;
  typedef int T;
};

template<typename T> struct DD : Base<T> {
  void f() { basefield = 0; }   // (1) problem
};

template<> class Base<bool> {
  enum { basefield = 42; }     // (2) hmm
};

void g (DD<bool>& d)
{
  d.f();                      // (3) oops?
}
\end{verbatim}
}

\begin{itemize}[<+->]
\item At point 2, the specialization changes the meaning of basefield.
\item At point 1, if we bind basefield to the int, we have the wrong meaning.
\item At point 3, an error message should be issued, but can't be if
  we bound the meaning too early.
\end{itemize}
\end{frame}
%%%%%%%%%%%%%%%%%%%%%%%%%%%%%%%%%%%%%%%%%%%%%%%%%%%%%%%%%%%%%%%%%%%%%%
\begin{frame}[fragile,t]
\frametitle{Derivation and Class Templates}
\framesubtitle{Dependent Base Class II}
\begin{itemize}[<+->]
\item Rule: \CC\ standards says that nondependent names are \emph{not}
  looked up in dependent base classes.  (1) should be an error.
\item If, instead, we make \texttt{basefield} dependent: 
{\scriptsize
\begin{verbatim}
  void f() { this->basefield = 0; }     // (1) no problem
  void f() { Base<T>::basefield = 0; }  // or this

\end{verbatim}
}
\emph{we delay the binding until later}.
\item  The second variation works, but note that it inhibits virtual
  function calls...
\end{itemize}


\end{frame}

%%%%%%%%%%%%%%%%%%%%%%%%%%%%%%%%%%%%%%%%%%%%%%%%%%%%%%%%%%%%%%%%%%%%%%
\begin{frame}[fragile,t]
\frametitle{Template Template Parameters}
\framesubtitle{The problem}
Template parameters can be templates in-and-of themselves:
\begin{itemize}[<+->]
\item Consider a stack template that lets the user choose the
  underlying container:
{\scriptsize
\begin{verbatim}
Stack<int, std::vector<int>> vstack;
\end{verbatim}
}
\vskip 12pt
\item Specifying 'int' twice is clunky and a source of errors (DRY
  principle).  This is better:
{\scriptsize
\begin{verbatim}
Stack<int, std::vector> vstack;
\end{verbatim}
}

\end{itemize}
\end{frame}

%%%%%%%%%%%%%%%%%%%%%%%%%%%%%%%%%%%%%%%%%%%%%%%%%%%%%%%%%%%%%%%%%%%%%%
\begin{frame}[fragile,t]
\frametitle{Template Template Parameters}
\framesubtitle{The solution}
\begin{itemize}[<+->]
\item How is it done?
{\scriptsize
\begin{verbatim}
template <typename T, 
          template<typename ELEM> class CONT = std::deque> class Stack
{
  private: 

    CONT<T> data;
  
    typedef typename CONT<T>::iterator iterator;

   ...
}
\end{verbatim}
}
\item The \texttt{ELEM} is unnecessary, it's a placeholder, and isn't
  actually needed (by the compiler... Humans tend to like it)
\item Oddly, it must be \texttt{class CONT}, ``typename'' won't work there.


\end{itemize}
\end{frame}
%%%%%%%%%%%%%%%%%%%%%%%%%%%%%%%%%%%%%%%%%%%%%%%%%%%%%%%%%%%%%%%%%%%%%%
\begin{frame}[fragile,t]
\frametitle{Template Template Parameters}
\framesubtitle{Details}
\begin{itemize}[<+->]
\item Another example: out-of-line member function definition...


{\scriptsize
\begin{verbatim}
template<typename T, template<template> class CONT> class Stack
void Stack<T,CONT>::Push (const T& elem)
{
  data.push_back(elem);
}
\end{verbatim}
}
\item Template template parameters for function templates are not
  allowed.
\item Note this won't actually compile, since the parameters for the
  template template parameter must match exactly... and vectors and
  deques have two template parameters...so it should really be:
{\scriptsize
\begin{verbatim}
template <typename T, 
          template<typename ELEM, 
                   typename = std::allocator> 
             class CONT = std::deque>               class Stack
{...}


template<typename T, 
         template<template, template> class CONT>   class Stack
void Stack<T,CONT>::Push (const T& elem)
{
  data.push_back(elem);
}
\end{verbatim}
}

\end{itemize}


\end{frame}



