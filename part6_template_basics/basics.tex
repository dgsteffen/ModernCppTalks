\subsection{Function Templates}

%%%%%%%%%%%%%%%%%%%%%%%%%%%%%%%%%%%%%%%%%%%%%%%%%%%%%%%%%%%%%%%%%%%%%%
\begin{frame}[fragile,t]
\frametitle{Function Templates}
\framesubtitle{The obvious problem}

\begin{itemize}[<+->]

\item Overloaded functions are wonderful:
\begin{columns}[t]
\column{.5\textwidth}
This sucks:
{\scriptsize
\begin{verbatim}
double      fabs(double x);
float       fabsf(float x);
long double fabsl(long double x);
\end{verbatim}
}
\column{.5\textwidth}
This is awesome:
{\scriptsize
\begin{verbatim}
double      abs(double x);
float       abs(float x);
long double abs(long double x);
\end{verbatim}
}
\end{columns}
\vskip 12pt
\item But the unrepentant C-programmers say:  ``That's Terrible!  We
  don't know which function is being called!''
\begin{itemize}
  \item Incorrect technically: figure out what the type is
  \item Incorrect philosophically: \Emph{it doesn't and shouldn't
    matter} if they all do semantically the same thing
\end{itemize}
\item But, then, we need to make sure that all overloads are
  semantically equivalent
\vskip 6pt
\item So, make it easy to make them all the same!
\end{itemize}
\end{frame}

%%%%%%%%%%%%%%%%%%%%%%%%%%%%%%%%%%%%%%%%%%%%%%%%%%%%%%%%%%%%%%%%%%%%%%
\begin{frame}[fragile,t]
\frametitle{Function Templates}
\begin{itemize}[<+->]
\item This effectively creates an unbounded set of type-appropriate functions:
{\scriptsize
\begin{verbatim}
template<typename T> inline
T abs(const T& a) { return a<0 ? -a : b; }
\end{verbatim}
}
\vskip 6pt
\item Template instantiation is automatic:
{\scriptsize
\begin{verbatim}
int i;    abs(i); // instantiates and calls abs<T> with T==int
short s;  abs(s); // T == short
double d; abs(d); // T == double
Widget q; abs(q); // T == Widget... which had better make sense
\end{verbatim}
}
\vskip 6pt
\item If \texttt{Widget} doesn't have comparison and negation
semantics, this is a compile time error.
\vskip 6pt
\item \Emph{Note: this is Duck Typing, not IS-A typing}
\begin{itemize}
  \item Usual \CC inheritance creates IS-A typing (Liskov Substitution
    Principle)
  \item Templates use ``Duck Typing''$^*$ : if it makes sense
    \Emph{syntactically} it's OK.
  \item ``If it looks like a number, walks like a number, and quacks
    like a number, then we can take the absolute value of it''.
  \item More commonly seen in E.G. Python (except here we catch
    mistakes at compile-time).
\end{itemize}
\end{itemize}
\pause
\begin{center}
{\scriptsize\Emph{$^*$Yes, this is the correct technical term;
    Wikipedia has a whole page on Duck Typing.}
}
\end{center}
\end{frame}


%%%%%%%%%%%%%%%%%%%%%%%%%%%%%%%%%%%%%%%%%%%%%%%%%%%%%%%%%%%%%%%%%%%%%%
\begin{frame}[fragile,t]
\frametitle{Typename or Class?}
\begin{itemize}[<+->]
\item These are identical:
\begin{columns}[t]
\column{.5\textwidth}
{\scriptsize
\begin{verbatim}
template<class T> ...
\end{verbatim}
}
\column{.5\textwidth}
{\scriptsize
\begin{verbatim}
template<typename T> ...
\end{verbatim}
}
\end{columns}
\vskip 12pt
\item Notional but common idiom:
\begin{itemize}
\item Use \texttt{class} when it must be a user-defined type
\item Use \texttt{typename} if it can also be a primitive type
\end{itemize}
\end{itemize}
\end{frame}


%%%%%%%%%%%%%%%%%%%%%%%%%%%%%%%%%%%%%%%%%%%%%%%%%%%%%%%%%%%%%%%%%%%%%%
\begin{frame}[fragile,t]
\frametitle{Overloading and Function Templates}
\begin{itemize}[<+->]
\item We can overload as usual:
{\scriptsize
\begin{verbatim}
template<typename T> T& max( T& a, T& b) { return (a>b) ? a : b; }  // #1

Widget& max( Widget& a, Widget& b) {                                // #2
   return WidgetComp(a,b) ? a : b; 
}
\end{verbatim}
}
\item Overloads are preferred over template instantiations
\item We can overload with additional template functions:
{\scriptsize
\begin{verbatim}
template<typename T*> T* max(T* a, T*b) { return (*a < *b) ? b : a ; } // #3
\end{verbatim}
}
\item All these overloads happily coexist, and overload resolution does what you'd expect.
{\scriptsize
\begin{verbatim}
int i, j;           max(i,j)   // instantiates and calls #1 with T == int
double x,y;         max(x,y)   // Ditto #1, T == double
Widget wick, wock;  max(wick,
                        wock)  //  Calls #2
int* p, q;          max(p, q)  // instantiates #3 with T == int
\end{verbatim}
}
\end{itemize}

\end{frame}



%%%%%%%%%%%%%%%%%%%%%%%%%%%%%%%%%%%%%%%%%%%%%%%%%%%%%%%%%%%%%%%%%%%%%%
\begin{frame}[fragile,t]
\frametitle{Argument Deduction}

\begin{itemize}[<+->]
\item For template functions \emph{only}, the compiler will deduce the desired type:
{\scriptsize
\begin{verbatim}
int i = max ( 3, 4 );  // deduces T is int (3 and 4 are integer literals

int j = -1;

max (i,j);            // T == int, deduced from argument types

max(wick, wock);      // T == Widget, ditto

max (j, 3.14);        // ints promote to doubles, 3.14 is a double, so
                      // T deduced to be double

max (j, wock);        // ERROR unless Widgets have a ctor that takes ints,
                      // in which case T == Widget

\end{verbatim}
}

\end{itemize}
\end{frame}

\begin{frame}[fragile,t]
\frametitle{Embarassing Template Problems}

\begin{itemize}[<+->]
\item Given our definition earlier, the last two won't actually compile:
{\scriptsize
\begin{verbatim}
// Takes by nonconst&, returns nonconst &
template<typename T> T& max( T& a, T& b) { return (a>b) ? a : b; }  // #1

int i;     double d;

max (i,d);  // oops
\end{verbatim}
}
\item T is deduced to be double...
\item \emph{but} the first parameter is an rvalue!
\item ... which won't bind to a normal non-const \&.



\item You might be tempted to do this:
{\scriptsize
\begin{verbatim}
template <typename A, typename B>
????  max( A& a, B& b ) { return (a<b) ? b : a ; } 

// wait... what's the return type?
\end{verbatim}
}
\item This is actually a difficult and embarassing problem: using a macro works, using a template doesn't!

\item This drove the development of several items in \CC11.

\end{itemize}
\end{frame}

%%%%%%%%%%%%%%%%%%%%%%%%%%%%%%%%%%%%%%%%%%%%%%%%%%%%%%%%%%%%%%%%%%%%%%
\begin{frame}[fragile,t]
\frametitle{Class Templates}
\framesubtitle{More things you already know}
\begin{itemize}[<+->]
\item Parameterize a class by a type:
{\scriptsize
\begin{verbatim}
template<typename T> class Vector { ... };
\end{verbatim}
}
\item Again, \texttt{typename} or \texttt{class} mean the same thing.
\item Implementing a member function out-of-line, it's still a template:
{\scriptsize
\begin{verbatim}
template<typename T> T& Vector<T>::operator[](int i) {...};

// and if it has a static integer member _count:

template<class T> int Vector<T>::_count = 0; // define & initialize static member

\end{verbatim}
}
\end{itemize}

\end{frame}

%%%%%%%%%%%%%%%%%%%%%%%%%%%%%%%%%%%%%%%%%%%%%%%%%%%%%%%%%%%%%%%%%%%%%%
\begin{frame}[fragile,t]
\frametitle{Member Templates}
\begin{itemize}[<+->]
\item Member functions can be templates too, and work just like non-member functions:
{\scriptsize
\begin{verbatim}
struct Fidget
{
   template<typename T> T Fuss(T&);   // unbounded set of overloaded member funs

   Widget Fuss(Widget&);              // Add an overloaded version
};

// out-of-line definition of the template:

template<typename T>  T Fidget::Fuss(T&) { ... }

// out-of-line definition of the overload, just like normal

Widget Fidget::Fuss(Widget& ) { ...}
\end{verbatim}
}

\end{itemize}

\end{frame}


%%%%%%%%%%%%%%%%%%%%%%%%%%%%%%%%%%%%%%%%%%%%%%%%%%%%%%%%%%%%%%%%%%%%%%
\begin{frame}[fragile,t]
\frametitle{Specialization of Class Templates}
\begin{itemize}[<+->]
\item Like providing overloads, we can customize class templates for certain types:
{\scriptsize
\begin{verbatim}
template<typename T> class vector { .... };

template<> class vector<bool> { ... } ;   // seemed like a good idea at the time...
\end{verbatim}
}
\item Note the \texttt{template<>} up front; it's still a template, just with no free parameters
\item Note that \emph{all} member functions must be specialized; you have to do the whole thing.
\item Note that \texttt{vector<bool>} has \Emph{nothing} in common with \texttt{vector<anything else>}.  There's no Liskov Substitution principle here.
\end{itemize}
\end{frame}


%%%%%%%%%%%%%%%%%%%%%%%%%%%%%%%%%%%%%%%%%%%%%%%%%%%%%%%%%%%%%%%%%%%%%%
\begin{frame}[fragile,t]
\frametitle{Specialization of Class Templates}
\framesubtitle{Continued... fiddly bits}
\begin{itemize}[<+->]
\item Partial specialization:
{\scriptsize
\begin{verbatim}
// it really looks like
template <typename T, class Alloc> class vector { ... };

// specialize for bool, but still allow any allocator
template <class Alloc> class vector<bool, Alloc> { ... };
\end{verbatim}
}
\item And you can have default template parameters
{\scriptsize
\begin{verbatim}
// ACTUALLY, it really really looks like
template <typename T, class Alloc = allocator<T>> class vector { ... };

// specialize for bool, but still allow any allocator
template <class Alloc = allocator<bool>> class vector<bool, Alloc> { ... };
\end{verbatim}
}

\end{itemize}
\end{frame}

%%%%%%%%%%%%%%%%%%%%%%%%%%%%%%%%%%%%%%%%%%%%%%%%%%%%%%%%%%%%%%%%%%%%%%
\begin{frame}[fragile,t]
\frametitle{Nontype Template Parameters}
\begin{itemize}[<+->]

\item Not all template arguments have to be classes:
{\scriptsize
\begin{verbatim}
template<typename T, size_t N>
class array {
  ...
  T data[N]; // N is a compile-time constant, so this is fine --
             // stores data on the stack
};
\end{verbatim}
}
\vskip 12pt
\item Specialization is still possible if you really want to...
{\scriptsize
\begin{verbatim}
template<size_t N> class array<bool, N> { ... }; // they didn't do this

template<typename T> class array<T, 1> { ... }; // maybe an optimized impl?

template<> class array<bool, 1> { ... }; // if you *really* want to
\end{verbatim}
}

\vskip 12pt

\item Nontype Template Parameters must generally be 
\begin{itemize}
  \item constant integral values, or
  \item pointers to objects with external linkage (A.K.A. constant, compile-time known addresses)
\end{itemize}
\end{itemize}

\end{frame}


%%%%%%%%%%%%%%%%%%%%%%%%%%%%%%%%%%%%%%%%%%%%%%%%%%%%%%%%%%%%%%%%%%%%%%
\begin{frame}[fragile,t]
\frametitle{String Literals as Template Parameters}

\begin{itemize}[<+->]
\item String Literals are interesting: they work, but....
{\scriptsize
\begin{verbatim}
template <const char* STR>  class Foo { .... };

Foo<"Hello"> hi;  // ERROR: string literla ``Hello'' not allowed.

extern const char* Greeting; // define in some .cpp somewhere

Foo<Greeting> hi;   // OK, now we're all right
\end{verbatim}
}
\vskip 12pt 
\item This, because string literals in different compilation units probably don't actually have the same address.
\end{itemize}

\end{frame}

%%%%%%%%%%%%%%%%%%%%%%%%%%%%%%%%%%%%%%%%%%%%%%%%%%%%%%%%%%%%%%%%%%%%%%
\begin{frame}[fragile,t]
\frametitle{... and other oddities}

Floating-point literals not allowed either (\CC98)
\begin{itemize}[<+->]
 \item Mainly a fiddly business about how identical floating-point ops carried out at compile time vs. run time can produce different results.
 \item Fixed in \CC11 (floating points ops are just fine)
 \item However, be careful; each computation probably produces an independent template instantiation.  EG:
{\scriptsize
\begin{verbatim}
template<double D> CTC { ... }; // some compile-time computation

template<> CTC< 1.0 >} { ... };  // specialization

CTC< (1.0/3)*3 > specialized;   // probably not the specialized version!
\end{verbatim}
}

\end{itemize}

\end{frame}



