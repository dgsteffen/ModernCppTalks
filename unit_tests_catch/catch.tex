\section{Catch}


\begin{frame}[fragile,t]
\frametitle{Catch}
\framesubtitle{Anatamy of a Catch test}
{\scriptsize\
\begin{verbatim}
#include "catch.hpp"

TEST_CASE("test case name") {
    REQUIRE(false);
}
\end{verbatim}}
produces
{\scriptsize\
\begin{verbatim}
Running Tests...
~~~~~~~~~~~~~~~~~~~~~~~~~~~~~~~~~~~~~~~~~~~~~~~~~~~~~~~~~~~~~~~~~~~~~~~~~~~~~~~
tests is a Catch v2.4.1 host application.
Run with -? for options

-------------------------------------------------------------------------------
Base
-------------------------------------------------------------------------------
/home/dsteffen/sofa/sofa/common-core/catch/example_test.cpp:3
...............................................................................

/home/dsteffen/sofa/sofa/common-core/catch/example_test.cpp:4: FAILED:
  REQUIRE( false )

===============================================================================
test cases: 1 | 1 failed
assertions: 1 | 1 failed
\end{verbatim}}
\end{frame}




\begin{frame}[fragile,t]
\frametitle{Catch}
\framesubtitle{Anatamy of a Catch test}
{\scriptsize\
\begin{verbatim}
#include "catch.hpp"

TEST_CASE("test case name", "[tag1][tag2]") {
    CHECK( any expression ); // if false, report and keep going
    REQUIRE( any expression ); // if false, abort test case
}
\end{verbatim}}

\begin{itemize}
  \item Special tags: 
    \begin{itemize}
      \item {[!hide]} or [.] skips the test by default. ''[.mytag]'' for
        example.
      \item {[!mayfail]} -- don't fail the test (but still report). ``Expected to fail''
      \item {[!shouldfail]} -- as above, but \emph{fails} the test if it passes.
      \item etc...
    \end{itemize}
\end{itemize}

Note the support for {\bf Ease of use}: it's easy to run the
executable, and easy to run selected subsets of the test suite.

Also, note that the test is (mostly) silent, and produces output only
relevent to failures.

\end{frame}


%%%%%%%%%%%%%%%%%%%%%%%%%%%%%%%%%%%%%%%%%%%%%%%%%%%%%%%%%%%%%%%%%%%%%%%%%%%%%%%%
%%%%%%%%%%%%%%%%%%%%%%%%%%%%%%%%%%%%%%%%%%%%%%%%%%%%%%%%%%%%%%%%%%%%%%%%%%%%%%%%

\begin{frame}[fragile,t]
\frametitle{Testing Floating Point}
\framesubtitle{}
This may be obvious, but it bears repeating:
{\scriptsize\
\begin{verbatim}
TEST_CASE( "Floating Point Division") {
  REQUIRE ( (1.0/3.0) * 3 == 3 );
}
\end{verbatim}}
is a bad idea. Catch2 provides a mechanism for this:
{\scriptsize\
\begin{verbatim}
TEST_CASE( "Floating Point Division") {
  REQUIRE ( (1.0/3.0) * 3 == Approx(3) );
}
\end{verbatim}}
where the \texttt{Approx} class can take arguments to control the comparison.
 Good frameworks have a set of floating-point
``approximately equals-to'' assertions of the $\Delta < \epsilon$
variety:
\begin{itemize}
  \item Consider making adapters for user-defined mathematical types
    (e.g. Vectors) so they are similarly supported by the framework.
  \item Proceed with care!  Consider if ``approximately the same to
    within $\epsilon$'' means
\begin{itemize}
  \item Relative error?
  \item Absolute error?
  \item Units-of-Last-Place (ULP) error?  (used for looking
    specifically at floating-point roundoff).
\end{itemize}
\end{itemize}
\end{frame}

%%%%%%%%%%%%%%%%%%%%%%%%%%%%%%%%%%%%%%%%%%%%%%%%%%%%%%%%%%%%%%%%%%%%%%%%%%%%%%%%
%%%%%%%%%%%%%%%%%%%%%%%%%%%%%%%%%%%%%%%%%%%%%%%%%%%%%%%%%%%%%%%%%%%%%%%%%%%%%%%%
\begin{frame}[fragile,t]
\frametitle{Testing Edge Cases Modes}
Test edge cases, including values at the edges of the domain.
{\scriptsize\begin{verbatim}
int factorial(int n)
{
  if (n < 0)
    throw (std::range_error("Factorial of negative number"));

  if (n > 12) return std::numeric_limits<int>::max();

  if (n == 0) return 1;

  return n * factorial(n-1);
}


TEST_CASE("Factorial success") {
  REQUIRE( factorial(0)  == 1 );
  REQUIRE( factorial(1)  == 1 );
  REQUIRE( factorial(5)  == 120 );
  REQUIRE( factorial(12) == 479001600 );   // this is as big as an int can go

}
\end{verbatim}}

Note: the domain of our function is [0, 12].  We have tested both
edges, the ``interesting'' cases ($0! = 1$), and one in the middle.

\end{frame}
%%%%%%%%%%%%%%%%%%%%%%%%%%%%%%%%%%%%%%%%%%%%%%%%%%%%%%%%%%%%%%%%%%%%%%%%%%%%%%%%
\begin{frame}[fragile,t]
\frametitle{Testing Failure Modes}
\framesubtitle{}
It is even more important to test failure modes -- make sure it fails,
in the right way, for the right reasons.
{\scriptsize\begin{verbatim}
int factorial(int n)
{
  if (n < 0)                                 TEST_CASE("Factorial error condition") {    
    throw (std::range_error())                 REQUIRE_THROWS(    factorial(-1) );          
  if (n > 12)                                  REQUIRE_THROWS_AS( factorial(-1) ,        
    return std::numeric_limits<int>::max();                       std::range_error);                    
  if (n == 0) return 1;                        REQUIRE( factorial(13) ==                 
  return n * factorial(n-1);                         std::numeric_limits<int>::max() );  
}                                              REQUIRE( factorial(13) == factorial(14) );
\end{verbatim}}

This test case fails if:
\begin{itemize}
  \item Calling factorial (-1) throws something other than std::range\_error 
  \item Succeeds (doesn't throw anything) 
  \item Calling factorial (13) returns 1932053504 (which is what you get on my laptop after integer overflow -- 
  the answer is 6227020800) rather than max int.
\end{itemize}

That is to say, the unit test \emph{fails} if the function \emph{doesn't} fail the right way.  This can be confusing.

\end{frame}

%%%%%%%%%%%%%%%%%%%%%%%%%%%%%%%%%%%%%%%%%%%%%%%%%%%%%%%%%%%%%%%%%%%%%%%%%%%%%%%%
%%%%%%%%%%%%%%%%%%%%%%%%%%%%%%%%%%%%%%%%%%%%%%%%%%%%%%%%%%%%%%%%%%%%%%%%%%%%%%%%
\subsection{Readability}
\begin{frame}[fragile,t]
\frametitle{Catch is good at storytelling}
Consider recasting the above in ``novel'' form:
{\scriptsize\begin{verbatim}
SCENARIO("Factorial tests") 
{
  SECTION("Success Conditions") 
  {
    REQUIRE( factorial(0)  == 1 );
    REQUIRE( factorial(1)  == 1 );
    REQUIRE( factorial(5)  == 120 );
    REQUIRE( factorial(12) == 479001600 );   // this is as big as an int can go
   }

  WHEN("passing a negative argument, we get an exception") 
  {
    REQUIRE_THROWS_AS( factorial(-1) , std::range_error);
  }

  WHEN("exceeding the max argument, we get int max") 
  {
    REQUIRE( factorial(13) == std::numeric_limits<int>::max() );  
    REQUIRE( factorial(13) == factorial(14) );
  }
}
\end{verbatim}}

Note the improved readability and ``story line''.

\end{frame}

\subsection{Classes with pure virtual functions}
%%%%%%%%%%%%%%%%%%%%%%%%%%%%%%%%%%%%%%%%%%%%%%%%%%%%%%%%%%%%%%%%%%%%%%%%%%%%%%%%
%%%%%%%%%%%%%%%%%%%%%%%%%%%%%%%%%%%%%%%%%%%%%%%%%%%%%%%%%%%%%%%%%%%%%%%%%%%%%%%%
\begin{frame}[fragile,t]
\frametitle{Testing Abstract Base Classes (ABCs)}
\framesubtitle{the problem}
Consider:
{\scriptsize\
\begin{verbatim}
class AbstractBase {
  public:

  virtual int PureVirtual () = 0

  int NonVirtualSquare(int i) {
    return i*i;
  }
};

class Widget : public AbstractBase; // non-abstract
class Wicket : public AbstractBase; // "
\end{verbatim}}
How do I unit test the NonVirtualSquare method?
{\scriptsize\
\begin{verbatim}
\\in AbstractBase_test.cpp

SCENARIO( "AbstractBase test") {
  AbstractBase i;  // oops! compile error
  REQUIRE ( i.square(5) ==  25 );
}
\end{verbatim}}
\end{frame}
%%%%%%%%%%%%%%%%%%%%%%%%%%%%%%%%%%%%%%%%%%%%%%%%%%%%%%%%%%%%%%%%%%%%%%%%%%%%%%%%
%%%%%%%%%%%%%%%%%%%%%%%%%%%%%%%%%%%%%%%%%%%%%%%%%%%%%%%%%%%%%%%%%%%%%%%%%%%%%%%%
\begin{frame}[fragile,t]
\frametitle{Testing ABC's}
\framesubtitle{the solution?}
You could test in one of the derived classes:
{\scriptsize\
\begin{verbatim}
// in Widget_test.cpp

SCENARIO( "AbstractBase Test", "[widget]") {
  Widget i;  // ok
  REQUIRE ( i.square(5) == 25 );
}
\end{verbatim}}
... but a failure shows up in Widget's test suite, not in
AbstractBase's test suite where it belongs.  Widget's test suite
fails, but Widget isn't incorrect.

\end{frame}

%%%%%%%%%%%%%%%%%%%%%%%%%%%%%%%%%%%%%%%%%%%%%%%%%%%%%%%%%%%%%%%%%%%%%%%%%%%%%%%%
%%%%%%%%%%%%%%%%%%%%%%%%%%%%%%%%%%%%%%%%%%%%%%%%%%%%%%%%%%%%%%%%%%%%%%%%%%%%%%%%
\begin{frame}[fragile,t]
\frametitle{Testing ABC's}
\framesubtitle{the solution!}
Instead, derive a test class for the purpose:
{\scriptsize\
\begin{verbatim}
// in AbstractBase_test.cpp  (the AbstractBase's test suite)

class AbstractBaseTester : public AbstractBase {
  virtual int PureVirtual () { ... } // some implementation 
};

SCENARIO( "AbstractBase Test", "[abstract base]") {
  AbstractBaseTester i;  // ok
  REQUIRE( i.square(5) ==  25 );
}
\end{verbatim}}
The tester class is only visible in the unit test, and allows all the
functionality of the abstract base class to be exercised.

\end{frame}


%%%%%%%%%%%%%%%%%%%%%%%%%%%%%%%%%%%%%%%%%%%%%%%%%%%%%%%%%%%%%%%%%%%%%%%%%%%%%%%%
%%%%%%%%%%%%%%%%%%%%%%%%%%%%%%%%%%%%%%%%%%%%%%%%%%%%%%%%%%%%%%%%%%%%%%%%%%%%%%%%
\subsection{Behavior Driven Testing}
\begin{frame}[fragile,t]
\frametitle{Testing Class Behavior}
\framesubtitle{The plot thickens}
This section based on
``Test Smells and Fragrances Kevlin Henney'' 
{\scriptsize\
\begin{verbatim}
class Cup {
  ...
  Cup();            // new cups are empty
  bool IsEmpty(); 
  bool Fill();    // Fills completely, returns false on failure
  bool Drink();   // Empties completely, "
};

\end{verbatim}}

Side note: the interface is interesting, in that:
\begin{itemize}
  \item Filling a full cup is not an error, it's just a failure
    (wastes beer).
  \item Drinking from an empty cup is not an error, it's just
    dissapointing.
\end{itemize}


\end{frame}

%%%%%%%%%%%%%%%%%%%%%%%%%%%%%%%%%%%%%%%%%%%%%%%%%%%%%%%%%%%%%%%%%%%%%%%%%%%%%%%%
%%%%%%%%%%%%%%%%%%%%%%%%%%%%%%%%%%%%%%%%%%%%%%%%%%%%%%%%%%%%%%%%%%%%%%%%%%%%%%%%
\begin{frame}[fragile,t]
\frametitle{Black-box testing conundrum}
``A test suite tests a collection of member functions'' : test each member function separately
{\scriptsize\
\begin{verbatim}
// test the constructor             
TEST_CASE( "Cup constructor test", "[cup]") {
  Cup cup;                  // new cups are supposed to be empty       
  REQUIRE( cup.IsEmpty() ); // test this.
}                                   

// test IsEmpty                       
TEST_CASE( "Cup isEmpty test", "[cup]") {
  Cup cup;                   // empty by construction 
  REQUIRE ( cup.IsEmpty() ); // test ... uh, didn't we just do this?
 }

\end{verbatim}}
Without reaching into the Cup class and examining internal state:
\begin{itemize}
\item we can't verify the constructor's postconditions without a correct
IsEmpty method
\item  and we can't verify the IsEmpty method without a cup
in a known state.
\end{itemize}

\Emph{Is this class untestable?}

\end{frame}

%%%%%%%%%%%%%%%%%%%%%%%%%%%%%%%%%%%%%%%%%%%%%%%%%%%%%%%%%%%%%%%%%%%%%%%%%%%%%%%%
%%%%%%%%%%%%%%%%%%%%%%%%%%%%%%%%%%%%%%%%%%%%%%%%%%%%%%%%%%%%%%%%%%%%%%%%%%%%%%%%
\begin{frame}[fragile,t]
\frametitle{Black-box testing conundrum}
\begin{itemize}[<+->]
\item We could ignore the problem (``Goofus'').
\item Maybe test something else?
{\scriptsize\
\begin{verbatim}
// test IsEmpty                       
TEST_CASE( "Cup isEmpty Tests", "[cup]") {
  Cup cup;

  REQUIRE( cup.IsEmpty() ); // just to be sure

  cup.Fill(); // should be full

  REQUIRE( not cup.IsEmpty() );
 }                                     
\end{verbatim}}
\item First... ``just to be sure''?  Do you really expect that to
  fail?  What are you testing? (but people do this a lot)
\item Second... this assumes that Fill works, this is even worse
\end{itemize}
\pause{}
\begin{center}
\Emph{Fundamentally, if we only test via public interface, we eventually
have a circular-logic problem: you have to use the interface to
test the interface, so at some point you have to trust something, right?}
\end{center}
\end{frame}


%%%%%%%%%%%%%%%%%%%%%%%%%%%%%%%%%%%%%%%%%%%%%%%%%%%%%%%%%%%%%%%%%%%%%%%%%%%%%%%%
%%%%%%%%%%%%%%%%%%%%%%%%%%%%%%%%%%%%%%%%%%%%%%%%%%%%%%%%%%%%%%%%%%%%%%%%%%%%%%%%
\begin{frame}[fragile,t]
\frametitle{White vs Black Box Testing}
This is the ``White Box'' vs ``Black Box'' debate.
\begin{itemize}
\item If black-box testing isn't viable, open up the box and look at
  the innards:
{\scriptsize\ \begin{verbatim}
TEST_CASE( "Cup isEmpty Tests", "[cup]") {
  Cup cup;

  REQUIRE( cup.empty_ ); // examine private data
                           // (which won't compile)

  cup.Fill(); // should be full

  REQUIRE( not cup.empty_ );
 }    
\end{verbatim}}
(which requires access to protected and private data members -- more
  on this later)
\item This is a very common approach
\item This is also a very frowned-upon approach (we'll see why)
\item How, technically, is it done?
\end{itemize}


\end{frame}

%%%%%%%%%%%%%%%%%%%%%%%%%%%%%%%%%%%%%%%%%%%%%%%%%%%%%%%%%%%%%%%%%%%%%%%%%%%%%%%
%%%%%%%%%%%%%%%%%%%%%%%%%%%%%%%%%%%%%%%%%%%%%%%%%%%%%%%%%%%%%%%%%%%%%%%%%%%%%%%%
\subsection{White Box Testing}
\begin{frame}[fragile,t]
\frametitle{White Box Testing}
\framesubtitle {How to access private data?}
This is a horrible idea, but it works...
{\scriptsize\
\begin{verbatim}
// Cup.h                                       // Cup_test.cpp                  
                                                                                
class Cup {                                    #define private public                                  
  Cup() : empty_(true) {}                      #include <Cup.h>                                                          

  bool IsEmpty() { return empty_; }            TEST_CASE("Cup ctor") 
                                               {                                         
  bool Fill() {                                   Cup c;                                 
   if (not empty_) return false;                  REQUIRE( c.empty_ ); 
   empty_ = false;                             }
   return true;
  }

  bool Drink() { ... }

private:
  bool empty_;
};
\end{verbatim}}
\begin{itemize}
\item This is, formally, undefined behavior.
\item One undisputed advantage: we can test Cup without changing the code.
\end{itemize}
\begin{center}
\Emph{Don't do this!}
\Emph{\#define private public is pure evil}
\end{center}
\end{frame}

%%%%%%%%%%%%%%%%%%%%%%%%%%%%%%%%%%%%%%%%%%%%%%%%%%%%%%%%%%%%%%%%%%%%%%%%%%%%%%%%
%%%%%%%%%%%%%%%%%%%%%%%%%%%%%%%%%%%%%%%%%%%%%%%%%%%%%%%%%%%%%%%%%%%%%%%%%%%%%%%%
\begin{frame}[fragile,t]
\frametitle{White Box Testing}
\framesubtitle {The standard solution}
If we are going to break encapsulation, do it correctly
{\scriptsize\
\begin{verbatim}
// Cup.h                                 // Cup_test.cpp
                                         #include <Cup.h>
class Cup {                              
                                         class CupTester {
  Cup() : empty_(true) {}                   bool& empty;    // proxy 
                                            
  bool IsEmpty() { return empty_; }         CupTester(Cup& c) 
                                              : empty(c.empty_) {}
  bool Fill() {                           };  
    if (empty) {                         
    ...                                  TEST( CupTests, TestConstructor)     
  }                                      {                                    
                                            Cup c;                            
  friend class CupTester;                   CupTester tester(c);
                                            REQUIRE( tester.empty ); 
};                                        }
\end{verbatim}}
\begin{itemize}
\item Cup declares a tester friend class
\item The friend class exposes Cup's innards
\end{itemize}
This is the recommended way to do white-box testing.
\end{frame}

%%%%%%%%%%%%%%%%%%%%%%%%%%%%%%%%%%%%%%%%%%%%%%%%%%%%%%%%%%%%%%%%%%%%%%%%%%%%%%%%
%%%%%%%%%%%%%%%%%%%%%%%%%%%%%%%%%%%%%%%%%%%%%%%%%%%%%%%%%%%%%%%%%%%%%%%%%%%%%%%%
\begin{frame}[fragile,t]
\frametitle{White Box Testing}
\framesubtitle {The obvious problem}
Test code is now tightly coupled to {\bf implementation details}.  If
we rig our cup with a pressure sensor, and query it to see if the cup
is empty...
{\scriptsize\
\begin{verbatim}
// Cup.h                                 // Cup_test.cpp
                                         #include <Cup.h>
class Cup {                              
                                         class CupTester {
  Cup() : empty_(true) {}                   bool& empty;    // proxy 
                                            
  bool IsEmpty() {                          CupTester(Cup& c) 
    return sensor_.query() == 0;              : empty_(c.empty) {} // <- build break
  }                                       };  
private:                                         
  friend class CupTester;                
  PressureSensor sensor_;                
};                                       
\end{verbatim}}
\begin{itemize}
\item We have to update the unit test code to match the new implementation
\item Additional maintenance burden
\begin{itemize}
\item ... and consider what happens to the team writing the
  PresureSensor class -- they make a change and our build breaks
\end{itemize}
\item The unit test now needs to be re-validated (re-reviewed), and
  can't stand as an independent source of validation
\end{itemize}

\end{frame}






%%%%%%%%%%%%%%%%%%%%%%%%%%%%%%%%%%%%%%%%%%%%%%%%%%%%%%%%%%%%%%%%%%%%%%%%%%%%%%%%
%%%%%%%%%%%%%%%%%%%%%%%%%%%%%%%%%%%%%%%%%%%%%%%%%%%%%%%%%%%%%%%%%%%%%%%%%%%%%%%%
\subsection{Behavior Driven Testing}
\begin{frame}[fragile,t]
\frametitle{Test for consistent behavior}
The solution is to \emph{not} take a
member-function-by-member-function approach.  \Emph{Test the class, not the
member functions.}
{\scriptsize\
\begin{verbatim}
TEST_CASE("A new cup is empty") {
  Cup cup;
  REQUIRE( cup.IsEmpty() ); 
 }                                     

TEST_CASE( "An empty cup can be filled" ) {
  Cup cup;               // empty, proved above
  REQUIRE( cup.Fill() ); // Fill returns true on success
}

TEST_CASE( "A filled cup is full") {
  Cup cup;
  cup.Fill();
  REQUIRE( not cup.IsEmpty() );
}  
\end{verbatim}}

\begin{itemize}
\item Note the vastly improved test names.
\item And we can now tell a much better story...
\end{itemize}
\end{frame}

%%%%%%%%%%%%%%%%%%%%%%%%%%%%%%%%%%%%%%%%%%%%%%%%%%%%%%%%%%%%%%%%%%%%%%%%%%%%%%%%
%%%%%%%%%%%%%%%%%%%%%%%%%%%%%%%%%%%%%%%%%%%%%%%%%%%%%%%%%%%%%%%%%%%%%%%%%%%%%%%%
\begin{frame}[fragile,t]
\frametitle{Test for consistent behavior}
\framesubtitle{Tell the story}
Re-write the above using Catch's BDD sections...
{\tiny \begin{verbatim}
TEST_CASE("Cup Tests") {

  GIVEN("A new cup") {
   Cup cup;

   THEN("A new cup is empty") { REQUIRE( cup.IsEmpty() ); }

   THEN("An empty cup can be filled" ) {
     REQUIRE( cup.Fill() );

     AND_THEN("A filled cup is full") { REQUIRE( not cup.IsEmpty() ); }

     WHEN("We try to drink from an empty cup") {
       // remember -- the cup is empty here!
       REQUIRE (not cup.Drink() );

       THEN("The cup stays empty") {
         REQUIRE( not cup.IsFull() )

         AND_THEN("The cup can be filled") {
           // Verify that no error condition persists
           REQUIRE( cup.Fill());
         }
      }
   }
}  
\end{verbatim}
}
\begin{itemize}
\item Testing the class \emph{as a whole}, not individual methods.
\item The test is easy to read, tells a story, contains context for
  all tests.
\item But have we really broken the circular logic loop?
\end{itemize}
\end{frame}

%%%%%%%%%%%%%%%%%%%%%%%%%%%%%%%%%%%%%%%%%%%%%%%%%%%%%%%%%%%%%%%%%%%%%%%%%%%%%%%%
%%%%%%%%%%%%%%%%%%%%%%%%%%%%%%%%%%%%%%%%%%%%%%%%%%%%%%%%%%%%%%%%%%%%%%%%%%%%%%%%
\begin{frame}[fragile,t]
\frametitle{Test for consistent behavior}

\begin{itemize}
\item Note that we are now testing \emph{consistency of the
  interface}, \Emph{not} \emph{correctness of the implementation}
\item Repeat the previous sentence to yourself several times.
\item This is Popper's falsifiability criterion: you can't prove it
  true, you can just fail to prove it false.
\item \Emph{SCIENCE!}
\end{itemize}

This class exhibits \emph{cohesion}, which is a good thing, and it
means precicely that we can't get ``under its skin'' without surgery.

\begin{center}
\Emph{If a class exhibits correct and consistent behavior in all
  cases, we must declare it correct, even if the implementation is
  nothing but mutually cancelling bugs.}
\end{center}

\end{frame}

%%%%%%%%%%%%%%%%%%%%%%%%%%%%%%%%%%%%%%%%%%%%%%%%%%%%%%%%%%%%%%%%%%%%%%%%%%%%%%%%
%%%%%%%%%%%%%%%%%%%%%%%%%%%%%%%%%%%%%%%%%%%%%%%%%%%%%%%%%%%%%%%%%%%%%%%%%%%%%%%%
\begin{frame}[fragile,t]
\frametitle{Test for consistent behavior}
\framesubtitle{prove it's not correct}
{\scriptsize\
\begin{verbatim}
class Cup {

  Cup() : empty_(false) {}                 // wrong!

  bool IsEmpty() { return not empty_; }      // wrong!

  bool Fill() { 
    if (not empty_) {  // backwards
      empty_ = true;   // backwards
      return true;
    ...
  }
  ...
private:
  bool empty_;
};
\end{verbatim}}
\vskip 6pt
\begin{itemize}
\item This \emph{should not pass code review}, but it is nevertheless
\emph{provably correct}.

\item The implementation is correct, it's just horribly confusing.

\item Black-box testing can't tell the difference, and that's the point.
\end{itemize}
\end{frame}



%%%%%%%%%%%%%%%%%%%%%%%%%%%%%%%%%%%%%%%%%%%%%%%%%%%%%%%%%%%%%%%%%%%%%%%%%%%%%%%%
%%%%%%%%%%%%%%%%%%%%%%%%%%%%%%%%%%%%%%%%%%%%%%%%%%%%%%%%%%%%%%%%%%%%%%%%%%%%%%%%
\begin{frame}[fragile,t]
\frametitle{White-Box Pros and Cons}

Cons:
\begin{itemize}
\item Tests are tightly coupled to implementation.
\begin{itemize}
  \item Changing the implementation breaks the unit tests.
  \item Keeping the unit tests working is an increased maintenance burden.
\end{itemize}
\item Tests no longer document the correct way to use a class
\item Tests can get much more complicated
\item Test authors are no longer bound by the class interface,
  reducing the pressure to improve a bad interface design.
\end{itemize}

Pros:
\begin{itemize}
\item Eash access to internal state makes tests easy to write
\item Requires very minor changes to code
\item Doesn't require redesign or refactoring
\item For some complex cases, it's the only viable option.
\end{itemize}


\end{frame}


%% %%%%%%%%%%%%%%%%%%%%%%%%%%%%%%%%%%%%%%%%%%%%%%%%%%%%%%%%%%%%%%%%%%%%%%%%%%%%%%%%
%% %%%%%%%%%%%%%%%%%%%%%%%%%%%%%%%%%%%%%%%%%%%%%%%%%%%%%%%%%%%%%%%%%%%%%%%%%%%%%%%%
%% \begin{frame}[fragile,t]
%% \frametitle{Software Engineering Considerations}
%% \end{frame}

