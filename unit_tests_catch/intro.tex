\section{Introduction}

\begin{frame}[fragile]
\frametitle{Unit Testing with Catch2}

This talk has three parts:

\begin{itemize}
\item Overview of Unit Testing
\item Unit Test Goals (high level, generic)
\item Using Catch2 to actually do those things
\end{itemize}

\end{frame}


\begin{frame}[fragile]
\frametitle{References}
\framesubtitle{E.G., where Dave stole all this material from}

\begin{itemize}
\item CppCon 2015: T. Winters \& H. Wright ``All Your Tests are Terrible...''
  \url{https://www.youtube.com/watch?v=u5senBJUkPc}
  \Emph{The} talk to start from.  The examples are in \CC but the
  content applies to all languages.
\vskip 6pt
\item Meeting C++ 2014: Phil Nash, ``Test Driven C++ with Catch''
  \url{https://www.youtube.com/watch?v=C2LcIp56i-8} An intro to Catch
  (v1) from the designer himself.

\end{itemize}

\end{frame}

%%%%%%%%%%%%%%%%%%%%%%%%%%%%%%%%%%%%%%%%%%%%%%%%%%%%%%%%%%%%%%%%%%%%%%
%%%%%%%%%%%%%%%%%%%%%%%%%%%%%%%%%%%%%%%%%%%%%%%%%%%%%%%%%%%%%%%%%%%%%%

\begin{frame}[fragile]
\frametitle{Definitions}
\framesubtitle{}
``Unit Testing'' is testing at the ``smallest unit of
functionality''.  For software engineers, these 
``smallest units'' are
\begin{itemize}
\item functions (or sets of related functions)
\item member functions (or classes)
\end{itemize}

In both cases, we are testing at the individual ``chunk'' of
executable code, the function or member function level.
\vskip 6pt
(The term can be used to mean different things in other areas of
engineering, but this definition is standard in software.)
\end{frame}

%%%%%%%%%%%%%%%%%%%%%%%%%%%%%%%%%%%%%%%%%%%%%%%%%%%%%%%%%%%%%%%%%%%%%%
%%%%%%%%%%%%%%%%%%%%%%%%%%%%%%%%%%%%%%%%%%%%%%%%%%%%%%%%%%%%%%%%%%%%%%

\begin{frame}[fragile]
\frametitle{Why Unit Test}
\framesubtitle{The bigger picture}
Definition: 

Programming is ``make it work now''.

Software Engineering is ``it still works in 5 years''.

Unit testing helps with the first, but it also...

\end{frame}

\begin{itemize}
\item Google Test (GTest), probably the most commonly used.
\item Boost.Test (part of the Boost library), also widely used.
\item Catch2 (``Catch''), newer ``more advanced'' framework, still
  young but rapidly gaining market share
\end{itemize}
\vskip 6pt
GTest is the standard framework in use at SciTec.  SOFA and OCEANA are using
Catch2.
\vskip 6pt
\begin{center}
\Emph{Unit testing frameworks make it easy to write unit tests; they reduce
the friction involved, provide consistent testing facilities, and
enforce consistent testing practices.}
\end{center}

\end{frame}


%%%%%%%%%%%%%%%%%%%%%%%%%%%%%%%%%%%%%%%%%%%%%%%%%%%%%%%%%%%%%%%%%%%%%%
%%%%%%%%%%%%%%%%%%%%%%%%%%%%%%%%%%%%%%%%%%%%%%%%%%%%%%%%%%%%%%%%%%%%%%

\begin{frame}[fragile]
\frametitle{Frameworks}
\framesubtitle{}
A unit test framework is a library (or language facility) that
provides unit testing features. Some languages have a framework built
in (C\#, Python(?)), some have one standard framwork (Java uses
JUnit), \CC has many:
\begin{itemize}
\item Google Test (GTest), probably the most commonly used.
\item Boost.Test (part of the Boost library), also widely used.
\item Catch2 (``Catch''), newer ``more advanced'' framework, still
  young but rapidly gaining market share
\end{itemize}
\vskip 6pt
GTest is the standard framework in use at SciTec.  SOFA and OCEANA are using
Catch2.
\vskip 6pt
\begin{center}
\Emph{Unit testing frameworks make it easy to write unit tests; they reduce
the friction involved, provide consistent testing facilities, and
enforce consistent testing practices.}
\end{center}

\end{frame}


%%%%%%%%%%%%%%%%%%%%%%%%%%%%%%%%%%%%%%%%%%%%%%%%%%%%%%%%%%%%%%%%%%%%%%
%%%%%%%%%%%%%%%%%%%%%%%%%%%%%%%%%%%%%%%%%%%%%%%%%%%%%%%%%%%%%%%%%%%%%%

