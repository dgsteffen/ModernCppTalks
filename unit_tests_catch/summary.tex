\section{summary}

\begin{frame}[fragile,t]
\frametitle{Summary of Catch2}
\begin{itemize}
  \item Investigate the documentation, learn about the special tags,
    assertions, etc.
  \item Prefer to write BDD-style tests
    \begin{itemize}
      \item By convention, ``Scenario'' is BDD, ``TEST\_CASE'' is
        old-school
        \item But do what reads best.
    \end{itemize}
  \item Balance BDD-style nesting with monolithic tests.
    \begin{itemize}
      \item Write novels...
        \item but maybe not War and Peace.
        \item Trilogies work
\end{itemize}

\item Additional good advice will be discovered while working on
  SOFA.  Stay tuned.
\end{itemize}

\end{frame}




\begin{frame}[fragile,t]
\frametitle{Unit Testing Summary}

\begin{itemize}
\item TEST!
\begin{itemize}
  \item even if it's more work (it isn't)
  \item even if it slows you down (it doesn't)
\end{itemize}
\pause{}
\item Write good tests when you can, bad tests when you can't, any
  test at all is better than none.
\pause{}
\item Listen to the experts (go watch the video again, take it to heart).
\pause{}
\item Write the test before you fix the bug (or write the code)
\pause{}
\item Let the requirement of testability guide your design
\item Prefer black-box testing (with necessary redesigns)
\item White-box test when you have to.
\item 
\end{itemize}
\vskip 12pt

\begin{center}
\Emph{Without testing we have no proof our code works, or that it will
  continue to work in the face of inevitable change.}
\end{center}


\begin{center}
\Emph{Test or die.}

\pause{}

Or at least, prepare to put in a lot of late nights.
\end{center}

\end{frame}


%\subsection{make\_shared, make\_unique}




