%%%%%%%%%%%%%%%%%%%%%%%%%%%%%%%%%%%%%%%%%%%%%%%%%%%%%%%%%%%%%%%%%%%%%%%%%%%%%%%%
%%%%%%%%%%%%%%%%%%%%%%%%%%%%%%%%%%%%%%%%%%%%%%%%%%%%%%%%%%%%%%%%%%%%%%%%%%%%%%%%
\section{Goals}

\begin{frame}[fragile,t]
\frametitle{The Goals}
\begin{itemize}
\item 0. Write Tests
\item 1. Write tests that test what you wanted to test
\item 2. Write readable tests: correct by inspection
\item 3. Write complete tests: test failure modes and edge cases
\item 4. Write demonstrative tests: show how to use the API
\item 5. Write resilient tests: hermetic, only breaks when there is an
  unacceptable behavior change
\end{itemize}
\end{frame}

%%%%%%%%%%%%%%%%%%%%%%%%%%%%%%%%%%%%%%%%%%%%%%%%%%%%%%%%%%%%%%%%%%%%%%%%%%%%%%%%
%%%%%%%%%%%%%%%%%%%%%%%%%%%%%%%%%%%%%%%%%%%%%%%%%%%%%%%%%%%%%%%%%%%%%%%%%%%%%%%%


\begin{frame}[fragile,t]
\frametitle{Properties of good tests}

\begin{itemize}
\item \Emph{Unit tests must be easy to run (one button push).}
\item \Emph{Unit tests must return an unambiguous pass/fail result.}
\item Correctness : it tests what it's supposed to test (and
  nothing else)
\item Readability :
\begin{itemize}
  \item Tests don't have tests, so they \emph{must be correct by inspection}
  \item Should read like a novel: setup, action, conclusion, and a happy ending
\end{itemize}
\item Completeness : tests cover the whole API, and only the API
  \begin{itemize}
  \item including edge cases, \emph{particularly} including edge cases
  \item excluding stuff that isn't under test
  \end{itemize}
\item Demonstrability : best documentation of how to use the API.
\item Resilience : ``the notion that your tests should only fail if the
  thing they were actually intended to demonstrate became false, and
  should fail \emph{for no other reason}''
\item \Emph{Test code is just as important as system code}
\end{itemize}

"Note that we didn't call these rules, we did call them goals of good
tests and I don't know that you can necessarily always get all of
them.''

%% \begin{itemize}
%% \item good tests are 100\% better than bad tests
%% \item even bad tests are 10000\% better than no tests
%% \end{itemize}

\end{frame}

%%%%%%%%%%%%%%%%%%%%%%%%%%%%%%%%%%%%%%%%%%%%%%%%%%%%%%%%%%%%%%%%%%%%%%%%%%%%%%%%
%%%%%%%%%%%%%%%%%%%%%%%%%%%%%%%%%%%%%%%%%%%%%%%%%%%%%%%%%%%%%%%%%%%%%%%%%%%%%%%%
\begin{frame}[fragile,t]
\frametitle{Ease of use}
\begin{itemize}
\item Unit tests must be easy to run. This is a function of the unit
  test framework, but there are a few things developers can keep in
  mind:
\begin{itemize}
  \item Group unit tests with some care, so developers can run just
    the tests relevant to their work. (We'll see how to group tests
    later)
  \item Don't write tests that require external setup; all unit tests
    should be self-contained executables. (All modern frameworks
    provide this, unless you're really doing something odd.)
\end{itemize}
\item Unit tests must run quickly, or allow relevant subsets to be run quickly.
\item Tests should be silent, unless they are failing; and failing
  tests should output only text relevant to the failure.  This implies
  that neither tests nor application code should be writing to the
  console. All "debugging" output should be removed prior to check-in.
  (Example later)
\end{itemize}
\end{frame}

%%%%%%%%%%%%%%%%%%%%%%%%%%%%%%%%%%%%%%%%%%%%%%%%%%%%%%%%%%%%%%%%%%%%%%%%%%%%%%%%
%%%%%%%%%%%%%%%%%%%%%%%%%%%%%%%%%%%%%%%%%%%%%%%%%%%%%%%%%%%%%%%%%%%%%%%%%%%%%%%%
\begin{frame}[fragile,t]
\frametitle{Readability}
\begin{itemize}
\item Tests must be provably correct by inspection (since we don't test our tests). Code should be clear and local.
\item    Avoid large (or even small) amounts of boilerplate or setup code.
\end{itemize}

Catch2 is particularly good at supporting this (as we will see).

\end{frame}

%% %%%%%%%%%%%%%%%%%%%%%%%%%%%%%%%%%%%%%%%%%%%%%%%%%%%%%%%%%%%%%%%%%%%%%%%%%%%%%%%%
%% %%%%%%%%%%%%%%%%%%%%%%%%%%%%%%%%%%%%%%%%%%%%%%%%%%%%%%%%%%%%%%%%%%%%%%%%%%%%%%%%
%% \begin{frame}[fragile,t]
%% \frametitle{Demonstrability}
%% \begin{itemize}
%% \item 
%%   looking elsewhere.
%% \end{itemize}
%% \vskip
%% \end{frame}


%% %%%%%%%%%%%%%%%%%%%%%%%%%%%%%%%%%%%%%%%%%%%%%%%%%%%%%%%%%%%%%%%%%%%%%%%%%%%%%%%%
%% %%%%%%%%%%%%%%%%%%%%%%%%%%%%%%%%%%%%%%%%%%%%%%%%%%%%%%%%%%%%%%%%%%%%%%%%%%%%%%%%
%% \begin{frame}[fragile,t]
%% \frametitle{Black box tests}
%% \framesubtitle{$\approx 21:00 - 33:00$}
%% Definitions:
%% \begin{itemize}
%%   \item {\bf Black-box testing:} tests use only the public
%%     interface of the test article.  (The test article is a ``black box'' you can't see
%%     inside of.)
%%   \item {\bf White-box testing:} tests access the internal state of
%%     the test article. (You can ``see inside the box'').  Also
%%     ``clear-box'', ``glass-box'', etc.
%% \end{itemize}
%% \vskip 6pt
%% Black box testing is much preferred over white-box, to the degree that
%% many people say ``Black-box is the only way to test'' and ``White-box
%% testing is a mistake''.

%% \vskip 6pt
%% That's overstating the case.  Black-box is better, but not always
%% doable. We'll see how to white-box test below,
%% and discuss the pros and cons.

%% \end{frame}

%% %%%%%%%%%%%%%%%%%%%%%%%%%%%%%%%%%%%%%%%%%%%%%%%%%%%%%%%%%%%%%%%%%%%%%%%%%%%%%%%%
%% %%%%%%%%%%%%%%%%%%%%%%%%%%%%%%%%%%%%%%%%%%%%%%%%%%%%%%%%%%%%%%%%%%%%%%%%%%%%%%%%
%% \begin{frame}[fragile,t]
%% \frametitle{Test Result Data}
%% \framesubtitle{$\approx 29:00$}
%% Hyrum and Titus make fun of the ``run your code to generate the output
%% that you verify your code against''.  But...
%% \begin{itemize}[<+->]
%% \item We did exactly this on OCEANA as a system test -- run the whole
%%   thing, get output, save it to a ``canon'' file, compare with that.
%% \item Note that this does not test that the system is correct, just
%%   that it hasn't changed.
%% \item This is \emph{still} really useful when refactoring or
%%   optimizing.
%% \item And.... \emph{it found a bug}.  The output didn't compare
%%   roughly 1/3 times.
%% \item We had nondeterministic behavior, and who knows how long it
%%   would have taken us to notice.
%% \item (Look up ``strict weak ordering'' in sorting predicates)
%% \end{itemize}
%% \pause{}
%% \Emph{Bad tests are still 10000\% better than no tests.}
%% \end{frame}



%% %%%%%%%%%%%%%%%%%%%%%%%%%%%%%%%%%%%%%%%%%%%%%%%%%%%%%%%%%%%%%%%%%%%%%%%%%%%%%%%%
%% %%%%%%%%%%%%%%%%%%%%%%%%%%%%%%%%%%%%%%%%%%%%%%%%%%%%%%%%%%%%%%%%%%%%%%%%%%%%%%%%
%% \begin{frame}[fragile,t]
%% \frametitle{Sustainability, and Properties of good tests}
%% \framesubtitle{$\approx 44:58 - 47:00$}


%% \begin{itemize}
%% \item Again: write tests, bad tests are better than no tests
%% \end{itemize}


%% \end{frame}

%%%%%%%%%%%%%%%%%%%%%%%%%%%%%%%%%%%%%%%%%%%%%%%%%%%%%%%%%%%%%%%%%%%%%%%%%%%%%%%%
%%%%%%%%%%%%%%%%%%%%%%%%%%%%%%%%%%%%%%%%%%%%%%%%%%%%%%%%%%%%%%%%%%%%%%%%%%%%%%%%
\begin{frame}[fragile,t]
\frametitle{Culture}
\framesubtitle{$\approx 57:25$}
\begin{itemize}
\item Tests are only part of the solution (Defence in depth)
\begin{itemize}
\item code review
\item proper testing
\item static analysis
\item dynamic analysis
\item hardening code
\item canarying
\item production monitoring
\end{itemize}

\item Culture
\end{itemize}


\end{frame}

