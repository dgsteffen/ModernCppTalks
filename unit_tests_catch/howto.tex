\section{How To Test...}
%%%%%%%%%%%%%%%%%%%%%%%%%%%%%%%%%%%%%%%%%%%%%%%%%%%%%%%%%%%%%%%%%%%%%%%%%%%%%%%%
%%%%%%%%%%%%%%%%%%%%%%%%%%%%%%%%%%%%%%%%%%%%%%%%%%%%%%%%%%%%%%%%%%%%%%%%%%%%%%%%
\subsection{Floating point operations}
\begin{frame}[fragile,t]
\frametitle{Testing Floating Point}
\framesubtitle{}
This may be obvious, but it bears repeating:
{\scriptsize\
\begin{verbatim}
TEST( FloatingTest, division_test) {
  ASSERT_TRUE ( (1.0/3.0) * 3 == 3 );
}
\end{verbatim}}
is a bad idea.  Good frameworks have a set of floating-point
``approximately equals-to'' assertions of the $\Delta < \epsilon$
variety:
{\scriptsize\
\begin{verbatim}
  ASSERT_NEAR( (1.0/3)*3, 3, 1e-14 );
\end{verbatim}}
\begin{itemize}
  \item Consider making adapters for user-defined mathematical types
    (e.g. Vectors) so they are similarly supported by the framework.
  \item Proceed with care!  Consider if ``approximately the same to
    within $\epsilon$'' means
\begin{itemize}
  \item Relative error?
  \item Absolute error?
  \item Units-of-Last-Place (ULP) error?  (used for looking
    specifically at floating-point roundoff).
\end{itemize}
\end{itemize}
Warning: most test frameworks provide OK-but-not-great support for
this.  You may have to roll your own if you're serious about error bounds.
\end{frame}

%%%%%%%%%%%%%%%%%%%%%%%%%%%%%%%%%%%%%%%%%%%%%%%%%%%%%%%%%%%%%%%%%%%%%%%%%%%%%%%%
%%%%%%%%%%%%%%%%%%%%%%%%%%%%%%%%%%%%%%%%%%%%%%%%%%%%%%%%%%%%%%%%%%%%%%%%%%%%%%%%
\subsection{Test for Failure}
\begin{frame}[fragile,t]
\frametitle{Testing Failure Modes}
\framesubtitle{}
It is vitally important to test not just that the code works, but that
it fails the right way.
{\scriptsize\
\begin{verbatim}
int factorial (int n);

TEST( FactorialFunction, Factorial_Failure) {

  ASSERT_THROW( factorial(-1), std::range_error );

  ASSERT_EQ ( factorial (13), -1 );
}
\end{verbatim}}
This test case fails if:
\begin{itemize}
  \item Calling factorial (-1) throws something other than std::range\_error 
  \item Succeeds (doesn't throw anything) 
  \item Calling factorial (13) returns 1932053504 (which is what you get on my laptop after integer overflow -- 
    the answer is 6227020800) rather than -1
\end{itemize}

That is to say, the unit test \emph{fails} if the function \emph{doesn't} fail the right way.  This can be confusing.

\end{frame}


\subsection{Classes with pure virtual functions}
%%%%%%%%%%%%%%%%%%%%%%%%%%%%%%%%%%%%%%%%%%%%%%%%%%%%%%%%%%%%%%%%%%%%%%%%%%%%%%%%
%%%%%%%%%%%%%%%%%%%%%%%%%%%%%%%%%%%%%%%%%%%%%%%%%%%%%%%%%%%%%%%%%%%%%%%%%%%%%%%%
\begin{frame}[fragile,t]
\frametitle{Testing Abstract Base Classes (ABCs)}
\framesubtitle{the problem}
Consider:
{\scriptsize\
\begin{verbatim}
class AbstractBase {
  public:

  virtual int PureVirtual () = 0

  int NonVirtualSquare(int i) {
    return i*i;
  }
};

class Widget : public AbstractBase; // non-abstract
class Wicket : public AbstractBase; // "
\end{verbatim}}
How do I unit test the NonVirtualSquare method?
{\scriptsize\
\begin{verbatim}
\\in AbstractBase_test.cpp

TEST( AbstractBase_test, NVSquareTest {
  AbstractBase i;  // oops! compile error
  ASSERT_EQ( i.square(5), 25 );
}
\end{verbatim}}
\end{frame}
%%%%%%%%%%%%%%%%%%%%%%%%%%%%%%%%%%%%%%%%%%%%%%%%%%%%%%%%%%%%%%%%%%%%%%%%%%%%%%%%
%%%%%%%%%%%%%%%%%%%%%%%%%%%%%%%%%%%%%%%%%%%%%%%%%%%%%%%%%%%%%%%%%%%%%%%%%%%%%%%%
\begin{frame}[fragile,t]
\frametitle{Testing ABC's}
\framesubtitle{the solution?}
You could test in one of the derived classes:
{\scriptsize\
\begin{verbatim}
// in Widget_test.cpp

TEST( WidgetTest, NVSquareTest {
  Widget i;  // ok
  ASSERT_EQ( i.square(5), 25 );
}
\end{verbatim}}
... but a failure shows up in Widget's test suite, not in
AbstractBase's test suite where it belongs.  Widget's test suite
fails, but Widget isn't incorrect.

\end{frame}

%%%%%%%%%%%%%%%%%%%%%%%%%%%%%%%%%%%%%%%%%%%%%%%%%%%%%%%%%%%%%%%%%%%%%%%%%%%%%%%%
%%%%%%%%%%%%%%%%%%%%%%%%%%%%%%%%%%%%%%%%%%%%%%%%%%%%%%%%%%%%%%%%%%%%%%%%%%%%%%%%
\begin{frame}[fragile,t]
\frametitle{Testing ABC's}
\framesubtitle{the solution!}
Instead, derive a test class for the purpose:
{\scriptsize\
\begin{verbatim}
// in AbstractBase_test.cpp  (the AbstractBase's test suite)

class AbstractBaseTester : public AbstractBase {
  virtual int PureVirtual () { ... } // some implementation 
};

TEST( AbstractBaseTest, NVSquareTest {
  AbstractBaseTester i;  // ok
  ASSERT_EQ( i.square(5), 25 );
}
\end{verbatim}}
The tester class is only visible in the unit test, and allows all the
functionality of the abstract base class to be exercised.

\end{frame}


%%%%%%%%%%%%%%%%%%%%%%%%%%%%%%%%%%%%%%%%%%%%%%%%%%%%%%%%%%%%%%%%%%%%%%%%%%%%%%%%
%%%%%%%%%%%%%%%%%%%%%%%%%%%%%%%%%%%%%%%%%%%%%%%%%%%%%%%%%%%%%%%%%%%%%%%%%%%%%%%%
\subsection{Test Classes Not Methods}
\begin{frame}[fragile,t]
\frametitle{Testing Class Behavior}
\framesubtitle{The plot thickens}
This section based on
``Test Smells and Fragrances Kevlin Henney'' 
{\scriptsize\
\begin{verbatim}
class Cup {
  ...
  Cup();            // new cups are empty
  bool IsEmpty(); 
  bool Fill();    // Fills completely, returns false on failure
  bool Drink();   // Empties completely, "
};

\end{verbatim}}

Side note: the interface is interesting, in that:
\begin{itemize}
  \item Filling a full cup is not an error, it's just a failure
    (wastes beer).
  \item Drinking from an empty cup is not an error, it's just
    dissapointing.
\end{itemize}


\end{frame}

%%%%%%%%%%%%%%%%%%%%%%%%%%%%%%%%%%%%%%%%%%%%%%%%%%%%%%%%%%%%%%%%%%%%%%%%%%%%%%%%
%%%%%%%%%%%%%%%%%%%%%%%%%%%%%%%%%%%%%%%%%%%%%%%%%%%%%%%%%%%%%%%%%%%%%%%%%%%%%%%%
\begin{frame}[fragile,t]
\frametitle{Black-box testing conundrum}
``A test suite tests a collection of member functions'' : test each member function separately
{\scriptsize\
\begin{verbatim}
// test the constructor             
TEST( CupTests, constructor) {      
  Cup cup;                      // new cups are supposed to be empty       
  ASSERT_TRUE( cup.IsEmpty() ); // test this.
}                                   

// test IsEmpty                       
TEST( CupTests, testIsEmpty) {        
  Cup cup;                      // empty by construction 
  ASSERT_TRUE( cup.IsEmpty() ); // test ... uh, didn't we just do this?
 }

\end{verbatim}}
Without reaching into the Cup class and examining internal state:
\begin{itemize}
\item we can't verify the constructor's postconditions without a correct
IsEmpty method
\item  and we can't verify the IsEmpty method without a cup
in a known state.
\end{itemize}

\Emph{Is this class untestable?}

\end{frame}

%%%%%%%%%%%%%%%%%%%%%%%%%%%%%%%%%%%%%%%%%%%%%%%%%%%%%%%%%%%%%%%%%%%%%%%%%%%%%%%%
%%%%%%%%%%%%%%%%%%%%%%%%%%%%%%%%%%%%%%%%%%%%%%%%%%%%%%%%%%%%%%%%%%%%%%%%%%%%%%%%
\begin{frame}[fragile,t]
\frametitle{Black-box testing conundrum}
\begin{itemize}[<+->]
\item We could ignore the problem (``Goofus'').
\item Maybe test something else?
{\scriptsize\
\begin{verbatim}
// test IsEmpty                       
TEST( CupTests, testIsEmpty) {        
  Cup cup;

  ASSERT_TRUE( cup.IsEmpty() ); // just to be sure

  cup.Fill(); // should be full

  ASSERT_TRUE( not cup.IsEmpty() );
 }                                     
\end{verbatim}}
\item First... ``just to be sure''?  Do you really expect that to
  fail?  What are you testing? (but people do this a lot)
\item Second... this assumes that Fill works, this is even worse
\end{itemize}
\pause{}
\begin{center}
\Emph{Fundamentally, if we only test via public interface, we eventually
have a circular-logic problem: you have to use the interface to
test the interface, so at some point you have to trust something, right?}
\end{center}
\end{frame}


%%%%%%%%%%%%%%%%%%%%%%%%%%%%%%%%%%%%%%%%%%%%%%%%%%%%%%%%%%%%%%%%%%%%%%%%%%%%%%%%
%%%%%%%%%%%%%%%%%%%%%%%%%%%%%%%%%%%%%%%%%%%%%%%%%%%%%%%%%%%%%%%%%%%%%%%%%%%%%%%%
\begin{frame}[fragile,t]
\frametitle{Test for consistent behavior}
The solution is to \emph{not} take a
member-function-by-member-function approach.  \Emph{Test the class, not the
member functions.}
{\scriptsize\
\begin{verbatim}
TEST( CupTests, A_New_Cup_Is_Empty) {        
  Cup cup;
  ASSERT_TRUE( cup.IsEmpty() ); 
 }                                     

Test( CupTests, An_Empty_Cup_Can_Be_Filled) {
  Cup cup;                   // empty, proved above
  ASSERT_TRUE( cup.Fill() ); // Fill returns true on success
}

TEST( CupTests, A_Filled_Cup_Is_Full) {
  Cup cup;
  cup.Fill();
  ASSERT_TRUE( not cup.IsEmpty() );
}  
\end{verbatim}}

\begin{itemize}
\item Note the vastly improved test names.
\item Testing the class \emph{as a whole}, not individual methods.
\item But have we really broken the circular logic loop?
\end{itemize}
\end{frame}

%%%%%%%%%%%%%%%%%%%%%%%%%%%%%%%%%%%%%%%%%%%%%%%%%%%%%%%%%%%%%%%%%%%%%%%%%%%%%%%%
%%%%%%%%%%%%%%%%%%%%%%%%%%%%%%%%%%%%%%%%%%%%%%%%%%%%%%%%%%%%%%%%%%%%%%%%%%%%%%%%
\begin{frame}[fragile,t]
\frametitle{Test for consistent behavior}

\begin{itemize}
\item Note that we are now testing \emph{consistency of the
  interface}, \Emph{not} \emph{correctness of the implementation}
\item Repeat the previous sentence to yourself several times.
\item This is Popper's falsifiability criterion: you can't prove it
  true, you can just fail to prove it false.
\item \Emph{SCIENCE!}
\end{itemize}

This class exhibits \emph{cohesion}, which is a good thing, and it
means precicely that we can't get ``under its skin'' without surgery.

\begin{center}
\Emph{If a class exhibits correct and consistent behavior in all
  cases, we must declare it correct, even if the implementation is
  nothing but mutually cancelling bugs.}
\end{center}

\end{frame}

%%%%%%%%%%%%%%%%%%%%%%%%%%%%%%%%%%%%%%%%%%%%%%%%%%%%%%%%%%%%%%%%%%%%%%%%%%%%%%%%
%%%%%%%%%%%%%%%%%%%%%%%%%%%%%%%%%%%%%%%%%%%%%%%%%%%%%%%%%%%%%%%%%%%%%%%%%%%%%%%%
\begin{frame}[fragile,t]
\frametitle{Test for consistent behavior}
\framesubtitle{prove it's not correct}
{\scriptsize\
\begin{verbatim}
class Cup {

  Cup() : empty_(false) {}                 // wrong!

  bool IsEmpty() { return not empty_; }      // wrong!

  bool Fill() { 
    if (not empty_) {  // backwards
      empty_ = true;   // backwards
      return true;
    ...
  }
  ...
private:
  bool empty_;
};
\end{verbatim}}
\vskip 6pt
\begin{itemize}
\item This \emph{should not pass code review}, but it is nevertheless
\emph{provably correct}.

\item The implementation is correct, it's just horribly confusing.

\item Black-box testing can't tell the difference, and that's the point.
\end{itemize}
\end{frame}


%%%%%%%%%%%%%%%%%%%%%%%%%%%%%%%%%%%%%%%%%%%%%%%%%%%%%%%%%%%%%%%%%%%%%%%%%%%%%%%%
%%%%%%%%%%%%%%%%%%%%%%%%%%%%%%%%%%%%%%%%%%%%%%%%%%%%%%%%%%%%%%%%%%%%%%%%%%%%%%%%
\subsection{White Box Testing}
\begin{frame}[fragile,t]
\frametitle{White Box Testing}
\framesubtitle {The idea}
Alternately, we could abandon black-box testing.
{\scriptsize\
\begin{verbatim}
// Cup.h                                 // Cup_test.cpp
                                         
class Cup {                              
                                         #include <Cup.h>                    
  Cup() : empty_(true) {}                                                    
                                                                             
  bool IsEmpty() { return empty_; }      TEST( CupTests, TestConstructor)         
                                         {                                        
  bool Fill() {                             Cup c;                                
    if (empty) {                            ASSERT_TRUE( c.empty_ ); // error!
    ...                                  }                           
  }
  ...                                       
private:
  bool empty_;
};
\end{verbatim}}
This won't compile (empty\_ is a private member variable), but if it
did, we would have absolute, unambigous proof that new cups are empty.

How can we make this work?

\end{frame}

%%%%%%%%%%%%%%%%%%%%%%%%%%%%%%%%%%%%%%%%%%%%%%%%%%%%%%%%%%%%%%%%%%%%%%%%%%%%%%%%
%%%%%%%%%%%%%%%%%%%%%%%%%%%%%%%%%%%%%%%%%%%%%%%%%%%%%%%%%%%%%%%%%%%%%%%%%%%%%%%%
\begin{frame}[fragile,t]
\frametitle{White Box Testing}
\framesubtitle {The idea}
This is a horrible idea, but it actually works on most compilers:
{\scriptsize\
\begin{verbatim}
// Cup.h                                 // Cup_test.cpp
                                         #define private public
class Cup {                              
                                         #include <Cup.h>
  Cup() : empty_(true) {}                
                                         
  bool IsEmpty() { return empty_; }      TEST( CupTests, TestConstructor)         
                                         {                                        
  bool Fill() {                             Cup c;                                
    if (empty) {                            ASSERT_TRUE( c.empty_ ); // fine!
    ...                                  }
  }
  ...
private:
  bool empty_;
};
\end{verbatim}}
\begin{itemize}
\item This is, formally, undefined behavior.
\item One undisputed advantage: we can test Cup without changing the code.
\end{itemize}
\begin{center}
\Emph{Don't do this!}
\end{center}
\end{frame}

%%%%%%%%%%%%%%%%%%%%%%%%%%%%%%%%%%%%%%%%%%%%%%%%%%%%%%%%%%%%%%%%%%%%%%%%%%%%%%%%
%%%%%%%%%%%%%%%%%%%%%%%%%%%%%%%%%%%%%%%%%%%%%%%%%%%%%%%%%%%%%%%%%%%%%%%%%%%%%%%%
\begin{frame}[fragile,t]
\frametitle{White Box Testing}
\framesubtitle {The standard solution}
If we are going to break encapsulation, do it correctly
{\scriptsize\
\begin{verbatim}
// Cup.h                                 // Cup_test.cpp
                                         #include <Cup.h>
class Cup {                              
                                         class CupTester {
  Cup() : empty_(true) {}                   bool& empty_;    // proxy 
                                            
  bool IsEmpty() { return empty_; }         CupTester(Cup& c) 
                                              : empty_(c.empty_) {}
  bool Fill() {                           };  
    if (empty) {                         
    ...                                  TEST( CupTests, TestConstructor)     
  }                                      {                                    
                                            Cup c;                            
  friend class CupTester;                   CupTester tester(c);
                                            ASSERT_TRUE( tester.empty_ ); 
};                                        }
\end{verbatim}}
\begin{itemize}
\item Cup declares a tester friend class
\item The friend class exposes Cup's innards
\end{itemize}
This is the recommended way to do white-box testing.
\end{frame}


%%%%%%%%%%%%%%%%%%%%%%%%%%%%%%%%%%%%%%%%%%%%%%%%%%%%%%%%%%%%%%%%%%%%%%%%%%%%%%%%
%%%%%%%%%%%%%%%%%%%%%%%%%%%%%%%%%%%%%%%%%%%%%%%%%%%%%%%%%%%%%%%%%%%%%%%%%%%%%%%%
\begin{frame}[fragile,t]
\frametitle{White-Box Pros and Cons}

Cons:
\begin{itemize}
\item Tests are tightly coupled to implementation.
\begin{itemize}
  \item Changing the implementation breaks the unit tests.
  \item Keeping the unit tests working is an increased maintenance burden.
\end{itemize}
\item Tests no longer document the correct way to use a class
\item Tests can get much more complicated
\item Test authors are no longer bound by the class interface,
  reducing the pressure to improve a bad interface design.
\end{itemize}

Pros:
\begin{itemize}
\item Eash access to internal state makes tests easy to write
\item Requires very minor changes to code
\item Doesn't require redesign or refactoring
\item For some complex cases, it's the only viable option.
\end{itemize}


\end{frame}


%% %%%%%%%%%%%%%%%%%%%%%%%%%%%%%%%%%%%%%%%%%%%%%%%%%%%%%%%%%%%%%%%%%%%%%%%%%%%%%%%%
%% %%%%%%%%%%%%%%%%%%%%%%%%%%%%%%%%%%%%%%%%%%%%%%%%%%%%%%%%%%%%%%%%%%%%%%%%%%%%%%%%
%% \begin{frame}[fragile,t]
%% \frametitle{Software Engineering Considerations}
%% \end{frame}

