
%%%%%%%%%%%%%%%%%%%%%%%%%%%%%%%%%%%%%%%%%%%%%%%%%%%%%%%%%%%%%%%%%%%%%%
\begin{frame}[fragile]
\frametitle{Lambdas}
\begin{columns}[t]
\column{.4\textwidth}
\Emph{Previously: }
\begin{itemize}
\item  <1->{\scriptsize\begin{verbatim}vector<int>::iterator i = v.begin();  \end{verbatim}}
\item[]<1->{\scriptsize\begin{verbatim}for (; i != v.end(); ++i) {\end{verbatim}}
\item[]<1->{\scriptsize\begin{verbatim}  if (*i > x && *i < y)    \end{verbatim}}
\item[]<1->{\scriptsize\begin{verbatim}     break;  \end{verbatim}}
\item[]<1->{\scriptsize\begin{verbatim} }   \end{verbatim}}
\vskip 6pt
\item  <2->{\scriptsize\begin{verbatim}struct my_pred {  \end{verbatim}}
\item[]<2->{\scriptsize\begin{verbatim}  int x, y; // assume a ctor\end{verbatim}}
\item[]<2->{\scriptsize\begin{verbatim}  bool operator()(int i) {   \end{verbatim}}
\item[]<2->{\scriptsize\begin{verbatim}    return (i > x && i < y); \end{verbatim}}
\item[]<2->{\scriptsize\begin{verbatim}}; \end{verbatim}}
\vskip 6pt
\item[]<2->{\scriptsize\begin{verbatim}auto i = find_if( begin(v), end(v), \end{verbatim}}
\item[]<2->{\scriptsize\begin{verbatim}             my_pred(x,y) );     \end{verbatim}}
\end{itemize}

\column{.6\textwidth}
\only<3-4>{\hskip 1in \Emph{But now....}}
\vskip 48pt
\begin{itemize}
\item  <4->{\scriptsize\begin{verbatim}auto i = find_if( begin(v), end(v), \end{verbatim}}
\item[]<4->{\scriptsize\begin{verbatim}[=](int i){ return *i > x && *i < y; } );\end{verbatim}}
\end{itemize}

\end{columns}
\end{frame}

%%%%%%%%%%%%%%%%%%%%%%%%%%%%%%%%%%%%%%%%%%%%%%%%%%%%%%%%%%%%%%%%%%%%%%

\begin{frame}[fragile,t]
\frametitle{Lambda Syntax and return types}

{\scriptsize
\begin{verbatim}
[]             (int i)      ->int         { ... ;}

[capture list] (parameters) ->return_type {body} 

 =  by value     optional     optional
 &  by ref      
\end{verbatim}
}

Deduced return type:
\begin{itemize}
\item One return statement? Return type deduced from expression.
\item No return statement?  Return type \texttt{void}
\item Multiple return statements? C++11, no; C++14, fine if all return
  statements return the same type.
\end{itemize}
\end{frame}

%%%%%%%%%%%%%%%%%%%%%%%%%%%%%%%%%%%%%%%%%%%%%%%%%%%%%%%%%%%%%%%%%%%%%%
\begin{frame}[fragile,t]
\frametitle{Lambda Syntax and capture lists}

{\scriptsize
\begin{verbatim}
[]             (int i)      ->int         { ... ;}

[capture list] (parameters) ->return_type {body} 

 =  by value     optional     optional
 &  by ref      
\end{verbatim}
}

Capture list:
\begin{itemize}
\item \texttt{'='} means 'by value',  \texttt{'\&'} means 'by reference'
\item Default:  \texttt{[=]} means 'capture all local static variables by value' 
\item Default:  \texttt{[\&]} means 'capture all local static variables by reference' 
\item Explicit \texttt{[i, \&j, k]}  or \texttt{[\&, j]}  
\end{itemize}

\end{frame}

%%%%%%%%%%%%%%%%%%%%%%%%%%%%%%%%%%%%%%%%%%%%%%%%%%%%%%%%%%%%%%%%%%%%%%
\begin{frame}[fragile,t]
\frametitle{More complete example}

{\scriptsize
\begin{verbatim}
int findFirstInRange( int x, double d, const vector<int>& v)
{

int y = static_cast<int>(sqrt(d));

auto i = find_if( begin(v), 
                  end  (v), 
   [x, &y](int i){ return *i > x && *i < y; } );

return *i;
}
\end{verbatim}
}
\pause

\begin{itemize}
\item The lambda captures x by value and y by reference
\item If we had written [=], the lambda would have captured x,y,d,
  \Emph{and v} by value (but only if we use them).
\item If we had written [\&], the lambda would have captured by
  reference (but only if we use them).
\end{itemize}

\end{frame}



%%%%%%%%%%%%%%%%%%%%%%%%%%%%%%%%%%%%%%%%%%%%%%%%%%%%%%%%%%%%%%%%%%%%%%
\begin{frame}[fragile,t]
\frametitle{Lambda Gotchas}
\begin{itemize}
\item Only one that I know of:
{\scriptsize
\begin{verbatim}
auto foo(double d) -> decltype(...) {
  int i;
  auto fn = [&](){return pow(d,i);}
  return fn;   // Boom!  Holds refs to local vars!
}
\end{verbatim}}

Be careful with returning lambdas.  Lambdas that capture local
variables by reference \emph{must not} leave that scope!

\vskip 18pt


\item Why would you want to do this?  Example:
{\scriptsize
\begin{verbatim}
  auto my_sort = sorting_lambda_factory(...)

  sort ( begin(v), end(v), my_sort );  // sort a vector
  set<foo, my_sort> m_set;             // set sorted the same way
\end{verbatim}
}

\end{itemize}
\end{frame}


%%%%%%%%%%%%%%%%%%%%%%%%%%%%%%%%%%%%%%%%%%%%%%%%%%%%%%%%%%%%%%%%%%%%%%
\begin{frame}[fragile,t]
\frametitle{Performance Improvements}
\center{\Emph{Live Demo} (TBD: re-recreate this)}

\pause

\begin{tabular}{cc}
Function pointers: &  43 sec  \\
Functors:         &  19 sec   \\
Lambdas:          &  18.5 sec \\
\end{tabular}

\begin{itemize}
\item Functors and Lambdas can and will be inlined; function pointers
  can't.
\item Lambdas \emph{may} be faster because of compiler or library optimizations.
\end{itemize}
\end{frame}

%%%%%%%%%%%%%%%%%%%%%%%%%%%%%%%%%%%%%%%%%%%%%%%%%%%%%%%%%%%%%%%%%%%%%%
\begin{frame}[fragile,t]
\frametitle{Reprise: Function Declaration Syntax}
\begin{itemize}
\item Original synax:
{\scriptsize\begin{verbatim}
Widget WidgetFactory (int, double, Gadget&) {...}
\end{verbatim}
}
\pause{}
\vskip 6pt
\item Suffix return type:
{\scriptsize\begin{verbatim}
auto WidgetFactory (int, double, Gadget&) -> Widget {...}
\end{verbatim}
}
\pause{}
\vskip 6pt
\item Lambda:
{\scriptsize\begin{verbatim}
auto WidgetFactory = [](int, double, Gadget&) -> Widget {...}
\end{verbatim}
}
\pause{}
\vskip 6pt
\item C++14 extends deduced return types to ordinary functions:
{\scriptsize\begin{verbatim}
auto WidgetFactory (int, double, Gadget&) {...}

auto WidgetFactory = [](int, double, Gadget&) {...}
\end{verbatim}
}
\end{itemize}
\end{frame}
 

%%%%%%%%%%%%%%%%%%%%%%%%%%%%%%%%%%%%%%%%%%%%%%%%%%%%%%%%%%%%%%%%%%%%%%
\begin{frame}[fragile,t]
\frametitle{Lambdas, Lambdas, Everywhere?}
We may be moving towards a style where \emph{most functions} are
written as lambdas:
\begin{columns}[t]
\column{.5\textwidth}
\Emph{\CC 98 Predicates:}
{\scriptsize\begin{verbatim}
struct isWocket 
{
  bool operator() (const Widget* r) 
  {
    return r->getType() == TypeEnum::Wocket;
  }
};

struct isIncompleteWocket
{
  bool operator() (const Widget* const p)
  {
  return ( isWocket()(p) && 
           ( !(p->isComplete() ) ) );
  }
};
\end{verbatim}
}
\pause{}
\column{.5\textwidth}
\Emph{With Lambdas}
{\scriptsize\begin{verbatim}
auto isWocket = [](const Widget* r)
{  
  return r->getType() == TypeEnum::Wocket;
}




auto isIncompleteWocket = 
                [](const Widget* const p)
{
  return ( isWocket(p) && 
           ( !(p->isComplete() ) ) ); 
}
\end{verbatim}
}
\end{columns}

\end{frame}


%%%%%%%%%%%%%%%%%%%%%%%%%%%%%%%%%%%%%%%%%%%%%%%%%%%%%%%%%%%%%%%%%%%%%%
\begin{frame}[fragile,t]
\frametitle{And Templates}
With C++14, generic lambdas (templates!)
\begin{columns}[t]
\column{.5\textwidth}
{\scriptsize\begin{verbatim}
template<typename T>
bool isWocket (const T* r)
{
    return r->getType() == TypeEnum::Wocket;

};
\end{verbatim}
}
\pause{}
\column{.5\textwidth}
{\scriptsize\begin{verbatim}
auto isWocket = [](const auto* r)
{  
  return r->getType() == TypeEnum::Wocket;
}
\end{verbatim}
}
\end{columns}

\vskip 12pt
\pause{}
\begin{block}{Irony...}
This is more-or-less the template syntax that Stroustrup wanted back
in the late 80's.
\end{block}

\end{frame}
