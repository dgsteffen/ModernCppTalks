
%%%%%%%%%%%%%%%%%%%%%%%%%%%%%%%%%%%%%%%%%%%%%%%%%%%%%%%%%%%%%%%%%%%%%%
\begin{frame}[fragile,t]
\frametitle{Universal References, review and encore}
\begin{itemize}[<+->]
\item Universal References: \texttt{T\&\&} is a \emph{Universal
  Reference}, not an \emph{rvalue reference} in specific conditions:
  \begin{itemize}
    \item \texttt{T} is a \emph{deduced type}
    \item The syntax is exactly \texttt{T\&\&}.  No \texttt{const,
      pointer, volatile}... 
    \item These rules are a very useful lie.  
    \item It really is an rvalue reference, but in a context where
      reference collapsing takes place.
  \end{itemize}
\item These rules crafted to support two desired behaviors:
  \begin{itemize}
  \item Move semantics
  \item Perfect Forwarding
  \end{itemize}
\item Move semantics we've covered... but what's this ``perfect
  fowarding'' thing about?
\end{itemize}
\end{frame}


%%%%%%%%%%%%%%%%%%%%%%%%%%%%%%%%%%%%%%%%%%%%%%%%%%%%%%%%%%%%%%%%%%%%%%
\begin{frame}[fragile,t]
\frametitle{Perfect Fowarding: The Problem}
\framesubtitle{part 1}
\begin{itemize}[<+->]
\item ``Rvalue References Explained'' by Thomas Becker, \url{http://thbecker.net/articles/rvalue_references/section_07.html}
\item Consider a factory function:
{\scriptsize\begin{verbatim}
template<typename T, typename A>
shated_ptr<T> Factory(A arg)
{
  return shared_ptr<T>{new T(arg);;
}
\end{verbatim}
}
\item Great, we just forward the argument to T's ctor.
\item But, we're always making a copy of Arg, which might be bad for
  many reasons.
\item OK, so take by reference:
{\scriptsize\begin{verbatim}
template<typename T, typename Arg> shated_ptr<T> Factory(Arg& arg) {...}
\end{verbatim}
}
\item ... and now it won't work for const\& or rvalue arguments.  OK,
  overloading is our friend:
{\scriptsize\begin{verbatim}
template<typename T, typename A> shared_ptr<T> Factory(      A&  arg) {...}
template<typename T, typename A> shared_ptr<T> Factory(const A&  arg) {...}
template<typename T, typename A> shared_ptr<T> Factory(      A&& arg) {...}
\end{verbatim}
}
\end{itemize}
\end{frame}

%%%%%%%%%%%%%%%%%%%%%%%%%%%%%%%%%%%%%%%%%%%%%%%%%%%%%%%%%%%%%%%%%%%%%%
\begin{frame}[fragile,t]
\frametitle{Perfect Fowarding: The Problem}
\framesubtitle{part 2}
\begin{itemize}[<+->]
\item But if we have two parameters, we have a combinatorial
  explosion: 
{\scriptsize\begin{verbatim}
template<typename T, typename A, typename B> shared_ptr<T> ...

... F(      A&  a, B& b ) {...}
... F(const A&  a, B& b ) {...}
... F(      A&& a, B& b ) {...}

... F(      A&  a, const B& b ) {...}
... F(const A&  a, const B& b ) {...}
... F(      A&& a, const B& b ) {...}

... F(      A&  a, B&& b ) {...}
... F(const A&  a, B&& b ) {...}
... F(      A&& a, B&& b ) {...}
\end{verbatim}
}
\end{itemize}
\pause
\Emph{Perfect fowarding to the rescue!}
\end{frame}

%%%%%%%%%%%%%%%%%%%%%%%%%%%%%%%%%%%%%%%%%%%%%%%%%%%%%%%%%%%%%%%%%%%%%%
\begin{frame}[fragile,t]
\frametitle{Perfect Fowarding}
\begin{itemize}[<+->]
\item One parameter
{\scriptsize\begin{verbatim}
template<typename T, typename A> shared_ptr<T> Factory(A&&  arg) {...}
\end{verbatim}
}
\item Two parameters
{\scriptsize\begin{verbatim}
template<typename T, typename A, typename B>
shared_ptr<T> Factory(A&& a, B&& b)          {...}

\end{verbatim}
}
\item arbitrary parameters
{\scriptsize\begin{verbatim}

template<typename T, typename ...As>
shared_ptr<T> Factory(As&&.... args) { ...}
\end{verbatim}
}

\vskip 12pt
\item So go back to our earlier examples of variadic templates, and
  put ``\&\&'' everywhere.
\end{itemize}
\end{frame}

%% %%%%%%%%%%%%%%%%%%%%%%%%%%%%%%%%%%%%%%%%%%%%%%%%%%%%%%%%%%%%%%%%%%%%%%
%% \begin{frame}[fragile,t]
%% \frametitle{Real Example (from last week): NullableValue<T>}
%% \begin{itemize}[<+->]
%% \item Problem Statement:
%%   \begin{itemize}
%%     \item JSON data structures are allowed to have ``null'' values
%%     \item For SmartBarcoding and other work, the C++ Terminal needs to
%%       support null fields
%%   \end{itemize}
%% \item Possible Solutions:
%%   \begin{itemize}
%%   \item That's easy.... store a \texttt{T*} that can be null.
%%     Potentially dangerous, but effective.

%%     \begin{itemize}
%%       \item However, we run into ownership and lifetime issues -- who
%%         calls \texttt{delete} and when?
%%       \item ... and has all the other disadvantages of dynamic memory
%%         (heap fragmentation, etc)
%%     \end{itemize}

%%   \item OK, use a smart pointer to solve the ownership problem.
%%     \begin{itemize}
%%       \item \texttt{std::shared\_ptr} not available on pre-\CC11
%%         compilers
%%       \item \texttt{boost::shared\_ptr} is available for \CC98
%%         compilers... but Boost hasn't been ported to VxWorks.
%%       \item Could be ported to our platform, but that takes time
%%         (shared pointers are notoriously
%%         harder-to-write-and-test-than-expected).
%%       \item Still fragments the heap, etc.
%%     \end{itemize}

%%    \item OK, wrap a value and a boolean into a class...
%% \end{itemize}
%% \end{itemize}
%% \end{frame}


%% %%%%%%%%%%%%%%%%%%%%%%%%%%%%%%%%%%%%%%%%%%%%%%%%%%%%%%%%%%%%%%%%%%%%%%
%% \begin{frame}[fragile,t]
%% \frametitle{Real Example (from last week): NullableValue<T>}
%% {\scriptsize\begin{verbatim}
%% template<typename T> class NullableValue
%% {
%% public:

%%    NullableValue()           ;
%%    NullableValue(const T& t) ; // <- Problem

%%    ...

%%    operator bool() const { return !_isnull; }
%%    bool operator! () const { return _isnull; }

%% private:

%%    bool _isnull;
%%    T    _val;

%% };
%% \end{verbatim}
%% }
%% \begin{itemize}[<+->]
%% \item Note the second constructor -- in \CC11 we \emph{should} have
%% {\scriptsize\begin{verbatim}
%%  NullableValue(const T&  t); // copy ctor
%%  NullableValue(      T&& t); // move ctor    
%% \end{verbatim}
%% }
%% \end{itemize}


%% \end{frame}
