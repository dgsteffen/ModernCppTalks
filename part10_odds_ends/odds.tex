
\subsection{noexcept}
%%%%%%%%%%%%%%%%%%%%%%%%%%%%%%%%%%%%%%%%%%%%%%%%%%%%%%%%%%%%%%%%%%%%%%
\begin{frame}[fragile,t]
\frametitle{noexcept}
\begin{itemize}[<+->]
\item If a function cannot throw an exception or if the program isn't
  written to handle exceptions thrown by a function, that function can
  be declared noexcept.
{\scriptsize\begin{verbatim}
 extern "C" double sqrt(double) noexcept;    // will never throw

 // I'm not prepared to handle memory exhaustion
 vector<double> my_computation(const vector<double>& v) noexcept
        {
            vector<double> res(v.size());   // might throw
            for(int i; i<v.size(); ++i) res[i] = sqrt(v[i]);
            return res;
        }
\end{verbatim}
}
\item \texttt{noexcept} is a shorthand for \texttt{noexcept(true)}.
\item New operator \texttt{noexcept(expr)} : is true iff the
  expression can't throw (all calls are noexcept)
  \begin{itemize}
  \item Doesn't evaluate its argument (like sizeof()).
  \end{itemize}
\item Replaces the old dynamic \texttt{throw()} specification, which
  was never actually useful.

\end{itemize}
\end{frame}

%%%%%%%%%%%%%%%%%%%%%%%%%%%%%%%%%%%%%%%%%%%%%%%%%%%%%%%%%%%%%%%%%%%%%%
\begin{frame}[fragile,t]
\frametitle{noexcept, pt 2}
\begin{itemize}[<+->]

\item Functions can be made conditionally noexcept:
{\scriptsize\begin{verbatim}
template<class T>
void do_f(vector<T>& v) noexcept(noexcept(f(v.at(0))))
                                // can throw if f(v.at(0)) can
 {
   for(int i; i<v.size(); ++i)
   v.at(i) = f(v.at(i));
 }
\end{verbatim}
}

\item Dtors shouldn't throw; a generated destructor is implicitly
  noexcept (independently of what code is in its body) if all of the
  members of its class have noexcept destructors.

\item It is typically a bad idea to have a move operation throw, so
  declare those noexcept wherever possible. A generated copy or move
  operation is implicitly noexcept if all of the copy or move
  operations it uses on members of its class have noexcept
  destructors.

\item If a function marked \texttt{noexcept(true)} actually throws,
  \texttt{std::terminate} is immediately called... crash and burn.

\end{itemize}
\end{frame}

%%%%%%%%%%%%%%%%%%%%%%%%%%%%%%%%%%%%%%%%%%%%%%%%%%%%%%%%%%%%%%%%%%%%%%
\begin{frame}[fragile,t]
\frametitle{noexcept, pt 3}
\begin{itemize}[<+->]

\item Enables some important optimizations:

\item \texttt{std::vector::push\_back} may need to allocate new
  memory, and move existing elements to the new memory.
\begin{itemize}
  \item Ideally, it calls \texttt{std::move} on its elements to do
    this.
  \item But, \texttt{push\_back} has the strong exception guarantee --
    if an operation fails, the vector is left exactly as it was.
  \item But what happens if a move constructor throws?  It's already
    moved a bunch of elements out...
\end{itemize}
  \item Therefore, to maintain the strong exception guarantee,
    \texttt{std::vector::push\_back} \emph{cannot move elements} \Emph{unless
      their move ctors are noexcept}.

\vskip 6pt
\item Recommendation: put \texttt{noexcept()} everywhere you possibly
  can.
\item Optimization aside, it's an important semantic statement that
  goes a long way towards making code efficient and safe.

\end{itemize}
\end{frame}




\subsection{User-defined Literals}
%%%%%%%%%%%%%%%%%%%%%%%%%%%%%%%%%%%%%%%%%%%%%%%%%%%%%%%%%%%%%%%%%%%%%%
\begin{frame}[fragile,t]
\frametitle{User-defined Literals}
\framesubtitle{From isocpp.org}
\begin{itemize}[<+->]
\item C++ has always provided literals for a variety of built-in types
{\scriptsize\begin{verbatim}
 123  // int
 1.2  // double
 1.2F // float
 'a'  // char
 1ULL // unsigned long long
 0xD0 // hexadecimal unsigned
 "as" // string (null-terminated const(?) char*
\end{verbatim}
}
\item However, in C++98 there are no literals for user-defined types.
\item This can be a bother and also seen as a violation of the
  principle that user-defined types should be supported as well as
  built-in types are.
\item In particular, people have requested:
{\scriptsize\begin{verbatim}
"Hi!"s                                    // std::string, not zero-terminated array of char
1.2i                                      // imaginary
123.4567891234df                          // decimal floating point (IBM)
101010111000101b                          // binary
123s                                      // seconds
123.56km                                  // not miles! (units)
1234567890123456789012345678901234567890x // extended-precision
\end{verbatim}
}

\end{itemize}
\end{frame}

%%%%%%%%%%%%%%%%%%%%%%%%%%%%%%%%%%%%%%%%%%%%%%%%%%%%%%%%%%%%%%%%%%%%%%
\begin{frame}[fragile,t]
\frametitle{User-defined Literals (pt 2}
\begin{itemize}[<+->]
\item C++11 supports ``user-defined literals'' via literal operators
  that map literals with a given suffix into a desired type.
{\scriptsize\begin{verbatim}
constexpr complex<double> operator "" i(long double d)  // imaginary literal
    {
        return {0,d};   // complex is a literal type
    }

std::string operator""s (const char* p, size_t n)   // std::string literal
    {
        return string(p,n); // requires free store allocation
    }
\end{verbatim}
}

\item Note the use of constexpr to enable compile-time evaluation. Given those, we can write
{\scriptsize\begin{verbatim}
 template<class T> void f(const T&);

 f("Hello");    // pass pointer to char*
 f("Hello"s);   // pass (5-character) string object
 f("Hello\n"s); // pass (6-character) string object
 auto z = 2+1i; // complex(2,1)
\end{verbatim}
}
\end{itemize}

\vskip 12pt
\pause
See \url{https://isocpp.org/wiki/faq/cpp11-language#udls} for all the details.

\end{frame}



\subsection{Template Aliases}
%%%%%%%%%%%%%%%%%%%%%%%%%%%%%%%%%%%%%%%%%%%%%%%%%%%%%%%%%%%%%%%%%%%%%%
\begin{frame}[fragile,t]
\frametitle{Template Aliases}
\framesubtitle{From isocpp.org}
\begin{itemize}[<+->]
\item We have long wanted something like ``template typedefs'':
{\scriptsize\begin{verbatim}
// My own storage allocator
template<class T> struct MyAllocator {...}

// Make a ``typename'' for MyAllocator vectors...
template<typename T>
typedef std::vector<T, MyAllocator<T>> MyVector;
\end{verbatim}
}
\item We now have that, but with a different syntax (reusing
  \texttt{typedef} turned out to be problematic)
{\scriptsize\begin{verbatim}
template<typename T>
using MyVector = std::vector<T, MyAllocator<T>>;
\end{verbatim}
}
\item This syntax can be used as a better$^*$ syntax for ordinary
  typedefs:
{\scriptsize\begin{verbatim}
    typedef void (*PFD)(double);  // C-style

    using PF = void (*)(double);  // using with C-style type

    using P = [](double)->void;   // using with suffix return type
\end{verbatim}
}
\vskip 6pt
\item \Emph{Recommendation: generally prefer \texttt{using} to \texttt{typedef}.}
\end{itemize}
\pause
$^*$ Better in most people's opinions.
\end{frame}


\subsection{Raw string literals}
%%%%%%%%%%%%%%%%%%%%%%%%%%%%%%%%%%%%%%%%%%%%%%%%%%%%%%%%%%%%%%%%%%%%%%
\begin{frame}[fragile,t]
\frametitle{Raw string literals}
\framesubtitle{From isocpp.org}
\begin{itemize}[<+->]
\item Sometimes, you want a string to just be a string.  Consider the
  regex for two words separated by a backslash (\textbackslash
  w\textbackslash \textbackslash \textbackslash w):
{\scriptsize\begin{verbatim}
string s = "\\w\\\\\\w";    // I hope I got that right
\end{verbatim}
}
\item A ``raw string literal'' is a string literal where ``\" isn't an
  excape character:
{\scriptsize\begin{verbatim}
string s = R"(\w\\\w)"; // I'm pretty sure I got that right
\end{verbatim}
}
\item The original proposal for raw strings presents thismotivating example
{\scriptsize\begin{verbatim}
"('(?:[^\\\\']|\\\\.)*'|\"(?:[^\\\\\"]|\\\\.)*\")|"
// Are the five backslashes correct or not? Even experts become easily confused.
\end{verbatim}
}
\item The notation is \texttt{R"(...)"}.
{\scriptsize\begin{verbatim}
R"("quoted string")"    // the string is "quoted string"
\end{verbatim}
}
\item If absolutely necessary, custom delimiters can be added:
{\scriptsize\begin{verbatim}
 R"***("quoted string containing the usual terminator (")")***"
// the string is "quoted string containing the usual terminator (")"
\end{verbatim}
}
\vskip 6pt
\item See
  \url{https://isocpp.org/wiki/faq/cpp11-language-misc#raw-strings}
\end{itemize}
\end{frame}


\subsection{Attributes}
%%%%%%%%%%%%%%%%%%%%%%%%%%%%%%%%%%%%%%%%%%%%%%%%%%%%%%%%%%%%%%%%%%%%%%
\begin{frame}[fragile,t]
\frametitle{Attributes}
\framesubtitle{From isocpp.org}
\begin{itemize}[<+->]
\item A new standard syntax aimed at providing some order in the mess
  of facilities for adding optional and/or vendor specific information
  into source code (e.g. \_\_attribute\_\_, \_\_declspec, and \#pragma)
\item Always relate to the immediately preceding syntactic entity.
{\scriptsize\begin{verbatim}
void f [ [ noreturn ] ] ()  // f() will never return
    {
        throw "error";  // OK
    }

struct foo* f [ [ carries_dependency ] ] (int i);   // hint to optimizer

int* g(int* x, int* y [ [ carries_dependency ] ] );
\end{verbatim}
}
\end{itemize}
\end{frame}

%%%%%%%%%%%%%%%%%%%%%%%%%%%%%%%%%%%%%%%%%%%%%%%%%%%%%%%%%%%%%%%%%%%%%%
\begin{frame}[fragile,t]
\frametitle{Standard Attributes}
\framesubtitle{From isocpp.org}
\begin{itemize}[<+->]
  \item \texttt{noreturn} : Indicates that the function does not return.
  \item \texttt{carries\_dependency} : Hint to opimizer
  \item C++14 adds these:
  \item \texttt{deprecated} and \texttt{deprecated("reason")} : using the name
    or entity is allowed but discouraged
  \item \texttt{fallthrough} in a \texttt{switch} statement on a line of
    its own indicates that the fallthrough is intentional. (Can shut
    up annoying compiler warnings.)
  \item \texttt{nodiscard} The compiler should warn if the function's
    return value (or a value of this type) is ignored.
  \item \texttt{maybe\_unused} : suppress compiler warnings
  \item \texttt{optimize\_for\_synchronized} : optimizer hint for
    synchronized execution.
\end{itemize}
\pause
Some of these are fairly obscure... check the documentation, your
mileage may vary.

\end{frame}

\subsection{Flotsam and Jetsam}
%%%%%%%%%%%%%%%%%%%%%%%%%%%%%%%%%%%%%%%%%%%%%%%%%%%%%%%%%%%%%%%%%%%%%%
\begin{frame}[fragile,t]
\frametitle{Things I Won't Talk About}
\begin{itemize}[<+->]
\item \texttt{alignas} for alignment control
\item \texttt{long long} (we \emph{officially} have it now)
\item Copying and rethrowing exceptions (\Emph{exception safe code is
  an entire series of talks in itself})
\item extended integral types (from C99)
\item generalized unions
\item generalised PODs (``Plain Old Data'')
\vskip 12pt
\item Concurrency: this is a topic that will also need several
  talks...
\begin{itemize}
  \item and I understand absolutely none of it.
  \item We can do one eventually, but don't hold your breath
  \item Volunteers needed!
\end{itemize}
\end{itemize}
\end{frame}
%% \subsection{Inline Namespaces}
%% %%%%%%%%%%%%%%%%%%%%%%%%%%%%%%%%%%%%%%%%%%%%%%%%%%%%%%%%%%%%%%%%%%%%%%
%% \begin{frame}[fragile,t]
%% \frametitle{Inline Namespaces}
%% \begin{itemize}[<+->]
%% \end{itemize}
%% \end{frame}
