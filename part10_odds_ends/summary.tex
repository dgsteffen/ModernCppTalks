\subsection{\CC\ Standards}
%%%%%%%%%%%%%%%%%%%%%%%%%%%%%%%%%%%%%%%%%%%%%%%%%%%%%%%%%%%%%%%%%%%%%%
\begin{frame}[fragile,t]
\frametitle{Summary: \CC11}
\begin{columns}[t]

\column{.3\textwidth}
{\scriptsize
C99 features

Inline namespace

\_\_cplusplus

alignments

atomic operations

auto

default \& delete member fns

default move \& copy

decltype

delegating constructors

inherited constructors

override controls: final, override

explicit conversion operators

extended integer types

extern templates

lambdas

null pointer (nullptr)

rvalue references / move ops

}
\column{.5\textwidth}
{\scriptsize
attributes: {[[carries\_dependency]]}, [[noreturn]]

Dynamic Initialization/Destruction with Concurrency

Uniform initialization syntax and semantics

constant expressions (constexpr)

copying and rethrowing exceptions

enum class (scoped and strongly typed enums)

exception propagation (preventing it; noexcept)

in-class member initializers

initializer lists (uniform and general initialization)

local classes as template arguments

long long integers (at least 64 bits)

narrowing (how to prevent it)

static (compile-time) assertions (static\_assert)

suffix return type syntax

thread-local storage (thread\_local)

Default template args for function templates

}

\column{.2\textwidth}
{\scriptsize
PODs (generalized)

Simple SFINAE rule

memory model

range-for statement

raw string literals

right-angle brackets

template alias

unicode characters

unions (generalized)

user-defined literals

variadic templates

\vskip 12pt

smart pointers

}
\end{columns}

\end{frame}

%%%%%%%%%%%%%%%%%%%%%%%%%%%%%%%%%%%%%%%%%%%%%%%%%%%%%%%%%%%%%%%%%%%%%%
\begin{frame}[fragile,t]
\frametitle{Summary: \CC14}
\begin{itemize}

\item Binary literals

\item Generalized return type deduction

\item \texttt{decltype(auto)}

\item Generalized lambda captures

\item Generic lambdas

\item Variable templates

\item \Emph{Extended \texttt{constexpr}}

\item \Emph{The \texttt{deprecated} attribute}

\item Digit separators

\end{itemize}

\end{frame}


%%%%%%%%%%%%%%%%%%%%%%%%%%%%%%%%%%%%%%%%%%%%%%%%%%%%%%%%%%%%%%%%%%%%%%
\begin{frame}[fragile,t]
\frametitle{Summary: \CC17 (\Emph{!})}
\begin{columns}[t]

\column{.5\textwidth}
{\scriptsize
New auto rules for direct-list-initialization

static\_assert with no message

typename in a template template parameter

Removing trigraphs

Nested namespace definition

Attributes for namespaces and enumerators

u8 character literals

Constant evaluation for all non-type template arguments

Fold Expressions

Unary fold expressions and empty parameter packs

Remove Deprecated Use of the register Keyword

Remove Deprecated operator++(bool)

Removing Deprecated Exception Specifications from C++17

Make exception specifications part of the type system

Aggregate initialization of classes with base classes

Lambda capture of *this

Using attribute namespaces without repetition

Dynamic memory allocation for over-aligned data

\_\_has\_include in preprocessor conditionals

Template argument deduction for class templates

Non-type template parameters with auto type

Guaranteed copy elision


}
\column{.5\textwidth}
{\scriptsize
New specification for inheriting constructors

Direct-list-initialization of enumerations

Stricter expression evaluation order

constexpr lambda expressions

Different begin and end types in range-based for

Attributes fallthrough, nodiscard, maybe\_unused

Ignore unknown attributes

Pack expansions in using-declarations

Decomposition declarations

Hexadecimal floating-point literals

init-statements for if and switch

Inline variables

DR: Matching of template template-arguments excludes compatible templates

std::uncaught\_exceptions()

constexpr if-statements


\vskip 12pt
\Emph{Partial support in GCC 6, full(?) in GCC 7}
\url{https://gcc.gnu.org/projects/cxx-status.html#cxx1z}



}

\end{columns}


\end{frame}


\subsection{Advice}
%%%%%%%%%%%%%%%%%%%%%%%%%%%%%%%%%%%%%%%%%%%%%%%%%%%%%%%%%%%%%%%%%%%%%%
\begin{frame}[fragile,t]
\frametitle{Advice}
\begin{itemize}[<+->]
\item Almost Always Auto (but tell us how it works out)
\item Suffix notation for return types: always use?
{\scriptsize\begin{verbatim}
int  foo (double d);        // normal function
auto foo (double d) -> int; // the same thing
\end{verbatim}
}
\item Uniform initialization with \{\}: use unless there's a reason not to
  \begin{itemize}
  \item int i = 42; // too simple and obvious to change
  \item Unless you need to construct with args that would be a valid initializer list:
\begin{verbatim}
vector<int> v(100); // 100 ints
vector<int> v{100}; // one element of value 100
\end{verbatim}
\end{itemize}
\item Range-based for loops: prefer everywhere
\item Prefer free-function begin/end, they're more adaptable
\item nullptr: use everywhere (but watch out for bugs)

\end{itemize}
\end{frame}

%%%%%%%%%%%%%%%%%%%%%%%%%%%%%%%%%%%%%%%%%%%%%%%%%%%%%%%%%%%%%%%%%%%%%%
\begin{frame}[fragile,t]
\frametitle{Advice}
\begin{itemize}[<+->]
\item Delegating and inheriting constructors: use without reservation
\item Virtual function controls (\texttt{override/final}):
  recommended, but \texttt{final} frequently indicates design errors.
\item Member function control (\texttt{delete/default} member
  functions): recommended
\item Explicit conversion operators:  use is probably mandatory
\item In-class initializers: recommended
\item Enum classes: recommended
\item Declare everything \texttt{noexcept} if at all possible (it
  should be the default).
\item Make user-defined literals for classes where it's appropriate
\item Prefer ``using'' rather than ``typedef'' since it's more general
  (``using'' works with templates)
\item Use raw string literals (more useful than you'd think)
\item use $[[$\texttt{noreturn}$]]$ and other attributes to silence
  compiler warnings.
\end{itemize}
\end{frame}

%%%%%%%%%%%%%%%%%%%%%%%%%%%%%%%%%%%%%%%%%%%%%%%%%%%%%%%%%%%%%%%%%%%%%%
\begin{frame}[fragile,t]
\frametitle{Lambdas and Function Declaration Syntax}
\begin{itemize}
\item Original synax:
{\scriptsize\begin{verbatim}
Widget WidgetFactory (int, double, Gadget&) {...}
\end{verbatim}
}
\pause{}
\vskip 6pt
\item Suffix return type:
{\scriptsize\begin{verbatim}
auto WidgetFactory (int, double, Gadget&) -> Widget {...}
\end{verbatim}
}
\pause{}
\vskip 6pt
\item Lambda:
{\scriptsize\begin{verbatim}
auto WidgetFactory = [](int, double, Gadget&) -> Widget {...}
\end{verbatim}
}
\pause{}
\vskip 6pt
\item C++14 extends deduced return types to ordinary functions:
{\scriptsize\begin{verbatim}
auto WidgetFactory (int, double, Gadget&) {...}

auto WidgetFactory = [](int, double, Gadget&) {...}
\end{verbatim}
}
\vskip 12pt
\pause
\item Lambdas: recommended, \emph{particularly} with standard
  algorithms.
  \begin{itemize}
    \item The community may be moving towards writing \emph{all}
      functions as lambdas
    \item Gotcha: lambdas that capture local variables by reference
      shouldn't leave that scope
  \end{itemize}



\end{itemize}
\end{frame}


%%%%%%%%%%%%%%%%%%%%%%%%%%%%%%%%%%%%%%%%%%%%%%%%%%%%%%%%%%%%%%%%%%%%%%
\begin{frame}[fragile,t]
\frametitle{Advice}
\begin{itemize}[<+->]
\item Prefer algorithms to loops
\begin{itemize}
  \item Remember Sean Parent's talk, \url{https://www.youtube.com/watch?v=IzNtM038JuI&list=PLHxtyCq_WDLXFAEA-lYoRNQIezL_vaSX-&index=9}
(Search YouTube for ``Programming Conversations Lecture 5 part 1'')
\begin{center}
\Emph{Design Goal: No Raw Loops}
\end{center}
\end{itemize}
\item Containers:
  \begin{itemize}
    \item Avoid \texttt{vector<bool>}
    \item Use \texttt{bitset<N>} as a compile-time-sized alternative
    \item Use \texttt{std::set, std::map} for convenience and
      maintainability
    \item use \texttt{std::vector} and \texttt{std::array<N,T>} for
        everything else
      \item New continers (\texttt{unordered\_*}) not generally useful
        \item Remember, Big-O notation is a lie (until you know what
          ``big'' means)
  \end{itemize}
\end{itemize}
\end{frame}


%%%%%%%%%%%%%%%%%%%%%%%%%%%%%%%%%%%%%%%%%%%%%%%%%%%%%%%%%%%%%%%%%%%%%%
\begin{frame}[fragile,t]
\frametitle{Advice: Move Semantics}
\begin{itemize}[<+->]
\item Write to enable copy elision and RVO when possible
\begin{itemize}
  \item If you want a copy of it, take by value
  \item When possible, arrange for all return statements to return the
    same object
\end{itemize}

\item Support move semantics in classes (when it makes sense)
\begin{itemize}
  \item Moved-from objects can be in an inconsistent state; they must
    be safely destructable and assignable.
    \item Don't move in return types; just return objects.
\end{itemize}
\item Use Universl References to enable move semantics and perfect
  forwarding
\item \Emph{rvalue-references are lvalues!}
\item Use \texttt{std::move} to make an l-value into an r-value (organ donor)
\begin{itemize}
  \item Never use an object after it has been moved from
\end{itemize}
\end{itemize}
\end{frame}

%%%%%%%%%%%%%%%%%%%%%%%%%%%%%%%%%%%%%%%%%%%%%%%%%%%%%%%%%%%%%%%%%%%%%%
\begin{frame}[fragile,t]
\frametitle{Advice: RAII and Smart Pointers}
\begin{itemize}[<+->]
\item ``SESE'' is outdated; don't warp code to avoid multiple return
  statements
\item Always use RAII (Resource Aquisition Is Initialization) to
  handle resource management
\item Use \texttt{std::unique\_ptr} for RAII of memory by default.
\item Use \texttt{std::shared\_ptr} for RAII when needed
\item Use \texttt{std::weak\_ptr} to break cycles in
  \texttt{shared\_ptr}s
\item Don't use \texttt{auto\_ptr} except in extreme, dire, or
  extraordinary situations
\item Use \texttt{std::make\_shared} and \texttt{std::make\_unique} to
  aquire memory directly into smart pointers
\item ``new'' and ``delete'' are banished from application code.
\item We have effectively ``perfect'' garbage collection, so use it.
\end{itemize}

\end{frame}


%%%%%%%%%%%%%%%%%%%%%%%%%%%%%%%%%%%%%%%%%%%%%%%%%%%%%%%%%%%%%%%%%%%%%%
\begin{frame}[fragile,t]
\frametitle{Advice: Templates, Metaprogramming, Oh My}
\begin{itemize}[<+->]
\item Understand templates, SFINAE, etc.  Don't be afraid to use them
  when appropriate.
\item Use \texttt{constexpr} for compile-time values. (Prefer over
  old-fashioned template metaprogramming)
\item Use standard type traits when possible
  \begin{itemize}
  \item \texttt{std::enable\_if} is particularly useful for SFINAE
  \end{itemize}
\item Use template metaprogramming for type computation when necessary
\item Use variadic templates when necessary or useful
\item Remember to use Universal References
\vskip 12pt
\item \Emph{Think compile time!  These techniques are trememdously
  useful in embedded systems.}
\end{itemize}
\end{frame}

%%%%%%%%%%%%%%%%%%%%%%%%%%%%%%%%%%%%%%%%%%%%%%%%%%%%%%%%%%%%%%%%%%%%%%
\begin{frame}[fragile,t]
\frametitle{Advice: When you have time...}
Watch these videos:
\begin{itemize}[<+->]
\item Sean Parent (mentioned earlier)
%\item Scott Meyers, Universal References:
%{\tiny \url{https://channel9.msdn.com/Shows/Going+Deep/Cpp-and-Beyond-2012-Scott-Meyers-Universal-References-in-Cpp11}}
\item Titus Winters, Lessons in Sustainability: \url{https://www.youtube.com/watch?v=zW-i9eVGU_k}
\item Winters \& Wright, Testing: \url{https://www.youtube.com/watch?v=u5senBJUkPc}
\item Chandler Carruth, Undefined Behavior:
   \url{https://youtu.be/yG1OZ69H_-o}
\item Chander Carruth, Efficiency: \url{https://www.youtube.com/watch?v=fHNmRkzxHWs}
\vskip 6pt
\item Why are we using this language? Scott Meyers, ``Why \CC Sails
  When the Vasa Sank''\url{https://youtu.be/ltCgzYcpFUI}
\item Of course, Stroustrup: keynotes at any CppCon, GoingNative, and
  anywhere else you can find him.
\end{itemize}

\end{frame}
