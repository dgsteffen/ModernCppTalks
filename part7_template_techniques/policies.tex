\subsection{Policies}
%%%%%%%%%%%%%%%%%%%%%%%%%%%%%%%%%%%%%%%%%%%%%%%%%%%%%%%%%%%%%%%%%%%%%%
\begin{frame}[fragile,t]
\frametitle{Composing Types}
\framesubtitle{see Alexandrescu!}
How to we compose a type out of available parts?
\pause
\begin{itemize}[<+->]
\item Good old OO
{\scriptsize
\begin{verbatim}
class Employee                             {...};

class Secretary          : public Employee {...};
class TempWorker         : public Employee {...};

class TemporarySecretary : public Secretary, 
                           public TempWorker {...};
\end{verbatim}
}
\item OK, Multiple Inheritance has its quirks, but then so does
  everything else.  This is just fine ($\pm$ some virtual inheritance)
\item You don't have to look far to find places where straight-forward
  inheritance causes problems.
\begin{itemize}
\item Can't enforce architectural designs (E.G. Singletons)
\item Combinatorial explosion of types must be manually
  created. (\texttt{HourlyTemporarySecretary, SalariedTemporaryCoder...}) 
\item Bad combinations not forbidden (Maybe all Temps are hourly,
  but how do you keep someone from making a SalariedTemp?)
\end{itemize}

\end{itemize}

\end{frame}

%%%%%%%%%%%%%%%%%%%%%%%%%%%%%%%%%%%%%%%%%%%%%%%%%%%%%%%%%%%%%%%%%%%%%%
\begin{frame}[fragile,t]
\frametitle{Composing Types}
\framesubtitle{Problems with inheritance}
\begin{itemize}[<+->]
\item Reasons to Inherit:
  \begin{itemize}
  \item Code Reuse
  \item Customization of existing components
  \item Run-time polymorphism (virtual functions)
\end{itemize}

\item Virtual functions are awesome: run-time polymorphism is what OO
  is all about, right?
  \begin{itemize}  
    \item Well worth the cost if you need it.
    \item But when you don't need it, you still pay space and time
      costs.
  \end{itemize}
\item The Liskov Substitution Principle is a two-edged sword.
  
  \begin{itemize}
  \item Run-time polymorphism is dangerous without it.  (Compare
      Python's Duck-Type system, where substitution failures can't be
      diagnosed until \emph{run time}! )
    \item But, not all relationships are hierarchical.  Does square
      inherit from rectangle or the other way around?
    \item Scott Meyers' infamous \Emph{All
      birds can fly, penguins are birds, penguins can't fly, uh-oh} problem.
\end{itemize}

\item Liskov is \emph{strong behavioral subtyping} which isn't always
  what we want
\item And it always comes with virtual functions and run-time polymorphism

\end{itemize}

\end{frame}



%%%%%%%%%%%%%%%%%%%%%%%%%%%%%%%%%%%%%%%%%%%%%%%%%%%%%%%%%%%%%%%%%%%%%%
\begin{frame}[fragile,t]
\frametitle{CommonComponent: WorkflowEngine}
\framesubtitle{Written two weeks ago!}
\begin{center}That's all very abstract... here's a real example: the
  SmartBarcoding WorkflowEngine.\end{center}
\begin{itemize}[<+->]
\item It's a common component, being built by ``The Reusables''
\item Takes an interpreted barcode, updates the Workflow, and sends an
  event to the GUI (or wherever)...
{\scriptsize
\begin{verbatim}
struct BarcodeEvent { 
  string scan; 
  string pattern; 
  BarcodeCategory category;
};

class WorkflowEngine
{
public:
   WorkflowEngine& Instance() ; // its a singleton
   BarcodeEvent processBarcodeScan (const string& scan);

private:
   WorkflowEngine(); // private ctor to enforce singleton
};
\end{verbatim}
}
\item WorkflowEngine configured via messages from ``The App''
  (Vista, Tomes, ...)
\end{itemize}
\end{frame}

%%%%%%%%%%%%%%%%%%%%%%%%%%%%%%%%%%%%%%%%%%%%%%%%%%%%%%%%%%%%%%%%%%%%%%
\begin{frame}[fragile,t]
\frametitle{CommonComponent: WorkflowEngine}
\framesubtitle{part 2}
\begin{itemize}[<+->]
\item Problem: This is an asynchronous process, how do we communicate
  results (send events) to the rest of the system?
\begin{itemize}
  \item Trima: Messages?  Maybe
  \item Trima: Callbacks?  Maybe
  \item Vista: ... there's some triple-buffer thing I've heard of...
  \item Novaris:  ???
\end{itemize}
\item Every product uses a different scheme for inter-process communication.
\item As a CommonComponent designer, I have no way of knowing
  \emph{even in principle} how to send barcode events to other tasks/threads
\item Arguably, it's not my job anyway; the Trima/Optia/Novartis guys
  should do that.
\end{itemize}
\end{frame}

%%%%%%%%%%%%%%%%%%%%%%%%%%%%%%%%%%%%%%%%%%%%%%%%%%%%%%%%%%%%%%%%%%%%%%
\begin{frame}[fragile,t]
\frametitle{CommonComponent: WorkflowEngine}
\framesubtitle{The obvious solution}

\begin{itemize}[<+->]
\item Obvious solution: make it a base class...
{\scriptsize
\begin{verbatim}
class WorkflowEngineBase
{
public:

   void processBarcode (const Barcode& barcode);

protected:

  virtual void fireBarcodeEvent(BarcodeEvent& bc) = 0;
};

void WorkflowEngineBase::processBarcode (const Barcode& barcode)
{
  BarcodeEvent event;

  ... process, process, process

  fireBarcodeEvent(event);
}
\end{verbatim}
}
\end{itemize}

\end{frame}

%%%%%%%%%%%%%%%%%%%%%%%%%%%%%%%%%%%%%%%%%%%%%%%%%%%%%%%%%%%%%%%%%%%%%%
\begin{frame}[fragile,t]
\frametitle{Trima WorkflowEngine, with messages}
...And let the device teams inherit and customize as necessary:
{\scriptsize
\begin{verbatim}
// Trima Message-based handler

class TrimaWorkflowEngine : public WorkflowEngineBase
{
public:

 TrimaWorkflowEngine& Instance() ;    // still a singleton

protected:

  TrimaWorkflowEngine() : msg (MessageBase::SEND_LOCAL) {}

    void fireBarcodeEvent(BarcodeEvent& bc) {
       msg.send(bc);
    }

private:

    Message<BarcodeEvent> msg;
};


\end{verbatim}
}
\end{frame}

%%%%%%%%%%%%%%%%%%%%%%%%%%%%%%%%%%%%%%%%%%%%%%%%%%%%%%%%%%%%%%%%%%%%%%
\begin{frame}[fragile,t]
\frametitle{Trima WorkflowEngine, callbacks...}
Or, use callbacks
{\scriptsize
\begin{verbatim}
// Trima Message-based handler

class TrimaWorkflowEngine : public WorkflowEngineBase
{
public:

  TrimaWorkflowEngine& Instance() ;    // still a singleton


  void setEventCallback (const CallbackBase& c) { _eventCallback = c ; }  

protected:

    void fireBarcodeEvent(BarcodeEvent& bc) {
       _eventCallback(bc);
    }

private:

    CallbackBase _eventCallback ;

};


\end{verbatim}
}
\end{frame}

%%%%%%%%%%%%%%%%%%%%%%%%%%%%%%%%%%%%%%%%%%%%%%%%%%%%%%%%%%%%%%%%%%%%%%
\begin{frame}[fragile,t]
\frametitle{Problems...}
\begin{itemize}[<+->]
\item Note: ``Singleton'' can't be enforced (making the base class a
  singleton doesn't help!)
  \begin{itemize}
    \item And, each team has to implement Instance themselves, which
      is silly, they're all exactly the same
  \end{itemize}
\item We get polymorphism, but don't want it:
  \begin{itemize}
    \item They're singletons, we never have more than one, and we know
      at compile time which one it is.
    \item But consider a public virtual (say, \texttt{virtual void
      incomingBarcode(Barcode\&)}):
{\scriptsize\begin{verbatim}
TrimaWorkflowEngine& twe = 
       TrimaWorkflowEngine::Instace(); // ref to instance

twe.incomingBarcode(...);   // device-specific input
\end{verbatim}
}

  \end{itemize}


\item This invokes virtual function behavior (vtable lookup,
  call-through- function-pointer) although \Emph{we know what function
    to call at compile-time}!
\item Are users going to be tempted to acess this thing through
  pointers-to-base-class?
\end{itemize}

\end{frame}

%%%%%%%%%%%%%%%%%%%%%%%%%%%%%%%%%%%%%%%%%%%%%%%%%%%%%%%%%%%%%%%%%%%%%%
\begin{frame}[fragile,t]
\frametitle{Problems...}
\begin{itemize}[<+->]
\item \Emph{We don't want run-time polymorphism.}
\item \Emph{We don't want IS-A.}
\item We just want a TrimaWorkflowEngine with some (easily!)
  customized (at compile time!) behavior.
\end{itemize}

\pause
\begin{center}
\Emph{Enter Policies!}
\end{center}

\end{frame}

%%%%%%%%%%%%%%%%%%%%%%%%%%%%%%%%%%%%%%%%%%%%%%%%%%%%%%%%%%%%%%%%%%%%%%
\begin{frame}[fragile,t]
\frametitle{Policy-Based Design}
\framesubtitle{Alexandrescu, 2001}
\begin{itemize}[<+->]
  \item \Emph{Compile-time polymorphism}.
  \item Policies are classes that specialize behavior
  \item The ``Host'' class uses policies specified at compile-time.
{\scriptsize\begin{verbatim}
class Policy1 {...};
class Policy2 {...};

template<class P> class Host : public P {...};

typedef Host<Policy1> Host1;
typedef Host<Policy2> Host2;

\end{verbatim}
}
\item We now have two customized Host classes;
\item Nobody is tempted to use them polymorphically, no virtual
  functions (as we will see)
\item Architectural constraints reside in the Host
\item Sometimes called ``Upside-down inheritance'' since we derive
  \emph{from} the customization, instead of the other way around.

\end{itemize}

\pause
\end{frame}


%%%%%%%%%%%%%%%%%%%%%%%%%%%%%%%%%%%%%%%%%%%%%%%%%%%%%%%%%%%%%%%%%%%%%%
\begin{frame}[fragile,t]
\frametitle{Revised WorkflowEngine}
{\scriptsize
\begin{verbatim}
template<typename EventPolicy>
class WorkflowEngineHost : public EventPolicy   {
public:
  WorkflowEngineHost& Instance() ;    // <-- its a singleton... enforced!

  void processBarcode (const Barcode& barcode)   {
    BarcodeEvent event;
    ...
    this->fireBarcodeEvent(event);    // presumably inherited from EventPolicy
                                      // (and remember the this-> !)
  }
};
\end{verbatim}
}
\pause
\Emph{And now the customized bit:}
{\scriptsize\begin{verbatim}
class MessageEventPolicy {   // note: no inheritance
public:
  MessageEventPolicy () : msg (MessageBase::SEND_LOCAL) {}

  void fireBarcodeEvent(BarcodeEvent& bc) {msg.send(bc);}  // note: not virtual!

private:
    Message<BarcodeEvent> msg;
};

typedef WorkflowEngineHost<MessageEventPolicy> TrimaWorkflowEngine;
\end{verbatim}
}
\end{frame}


%%%%%%%%%%%%%%%%%%%%%%%%%%%%%%%%%%%%%%%%%%%%%%%%%%%%%%%%%%%%%%%%%%%%%%
\begin{frame}[fragile,t]
\frametitle{Revised WorkflowEngine, pt 2}
\begin{itemize}[<+->]
\item Note: no virtual functions to be seen.
\item In fact, \texttt{fireBarcodeEvent} will be inlined.
  \begin{itemize}
    \item We've gone from a vthunk + indirect function call to
      \emph{no function call at all}.
  \end{itemize}
\item Note that we use Duck-Typing here; all the policy needs to do is
  provide a \texttt{fireBarcodeEvent} method with the right signature.
\end{itemize}
\pause
{\scriptsize
\begin{verbatim}
class CallbackPolicy     // Nothing to do with the MessageEventPolicy
{
public:
  CallbackPolicy();

  void setEventCallback (const CallbackBase& c) { _eventCallback = c ; }  

  void fireBarcodEvent(BarcodeEvent& bc) {
    _eventCallback(bc); 
  }

private:
    CallbackBase _eventCallback ;
};
\end{verbatim}
}

\begin{itemize}[<+->]
\item Note that the policies give the host \emph{different interfaces}.
\end{itemize}
\end{frame}

%%%%%%%%%%%%%%%%%%%%%%%%%%%%%%%%%%%%%%%%%%%%%%%%%%%%%%%%%%%%%%%%%%%%%%
\begin{frame}[fragile,t]
\frametitle{Other options...}
\begin{itemize}[<+->]
\item Hosts can also use the policy as a datamember:
{\scriptsize\begin{verbatim}
template<typename Policy>
class Host {
  ...
private:

  Policy p;
};
\end{verbatim}
}
\item In this case, the policy can't enhance the Host's
  interface... but it's also a lot simpler.
\item Particularly useful if the policy just acts like a function.
\end{itemize}
\end{frame}


%%%%%%%%%%%%%%%%%%%%%%%%%%%%%%%%%%%%%%%%%%%%%%%%%%%%%%%%%%%%%%%%%%%%%%
\begin{frame}[fragile,t]
\frametitle{So what do we have?}
\begin{itemize}[<+->]

\item We compose at compile-time, not run-time.
\item Significant run-time savings.
\item Errors are caught at compile-time
\item Design constraints enforced by the host class. (The ``base
  class'' can be a Singleton now.)
\item Composed class' interface enhanced as necessary by the policy.
\item We still have an IS-A relationship... but who is going to try to
  use a TrimaWorkflowEngine AS-A CallbackPolicy?
  \begin{itemize}
    \item Note: we can prevent users from trying to do so by declaring
      the policy's ctor and dtor \texttt{protected}.  Nobody but the
      Host can make one.
  \end{itemize}

\end{itemize}
\end{frame}
