
\subsection{Traits}
%%%%%%%%%%%%%%%%%%%%%%%%%%%%%%%%%%%%%%%%%%%%%%%%%%%%%%%%%%%%%%%%%%%%%%
\begin{frame}[fragile,t]
\frametitle{Type Traits}

\begin{itemize}[<+->]
\item Traits classes map types to compile-time things:
\[
  \mathrm{type} \xrightarrow{\quad\mathrm{traits}\quad}%\stackrel{\mathrm{trait}}{\longmapsto}
    \left\{
      \begin{matrix}
        \mathrm{type} \\
        \mathrm{value} \\
        \mathrm{behavior}
        \end{matrix}
      \right.
\]
\vskip 12pt
\item Allows compile-time decisions to be made based on specific
  types, or on generic types within other templates.

\vskip 12pt
\item The proverbial ``extra level of indirection'', available at
  compile time.

\end{itemize}

\end{frame}

%%%%%%%%%%%%%%%%%%%%%%%%%%%%%%%%%%%%%%%%%%%%%%%%%%%%%%%%%%%%%%%%%%%%%%
\begin{frame}[fragile,t]
\frametitle{Example: Summing a Sequence}
\framesubtitle{Type-to-type mapping}

\begin{itemize}[<+->]
\item A simple example: summing a series of values:
{\scriptsize\begin{verbatim}
template<typename T> inline
T accum(const T* begin, const T* end) {
  T total = 0;
  while (begin != end) total += *begin++;
  return total;
}

int num[] = { 1, 2, 3, 4, 5 };

const char* txt = "Breakfast Meetings Rock";

cout << accum(num, num+5) << endl;   // produces 15

cout << (int)accum(txt, txt+strlen(txt)) // produces -98
     << endl;
\end{verbatim}
}
\vskip 12pt
\item \Emph{Clearly, the \texttt{char} version overflowed.}
\end{itemize}
\end{frame}

%%%%%%%%%%%%%%%%%%%%%%%%%%%%%%%%%%%%%%%%%%%%%%%%%%%%%%%%%%%%%%%%%%%%%%
\begin{frame}[fragile,t]
\frametitle{Example: Summing a Sequence}
\framesubtitle{continued}
\begin{itemize}[<+->]
\item We need to specify the types of the sum and the summands
  independently.  OK, that's easy...
{\scriptsize\begin{verbatim}
template<typename S, typename V> inline
S accum(const V* begin, const V* end) { ... }

unsigned sum;  // unsigned to make things interesting

sum = accum<int> (num, num+5); // we now have to specify
                               // the return type,

sum = accum<int> (txt,         // which is annoying...
                  txt+23);     // but correct (sum == 2206)

sum = accum<char>(txt,         // or deadly
                  txt+23);     // (sum == 4294967198)
\end{verbatim}
}
\vskip 12pt
\item Can't we choose the return type automatically?
\end{itemize}
\end{frame}



%%%%%%%%%%%%%%%%%%%%%%%%%%%%%%%%%%%%%%%%%%%%%%%%%%%%%%%%%%%%%%%%%%%%%%
\begin{frame}[fragile,t]
\frametitle{Example: Summing a Sequence}
\framesubtitle{still continued}
\begin{itemize}[<+->]
\item Traits to the rescue!
{\scriptsize\begin{verbatim}

template<typename T> struct SumTraits { typedef T type};  // default

template<> struct SumTraits<char> { typedef int type; }:  // specialize

template<T> inline
typename SumTraits<T>::type accum(const T* begin, const T* end) 
{
  typename SymTraits<T>::type total = 0;
  while (begin != end) total += *begin++;
  return total;
}

cout << accum(num, num+5) << endl;   // produces 15

cout << accum(txt, txt+23)) << endl; // produces 2206
\end{verbatim}
}

\vskip 12pt
\item We now 'hard code' the correct return type; the user can't mess
  it up.
\vskip 12pt
\item (Yeah, the syntax gets a little ugly...)
\end{itemize}
\end{frame}

%%%%%%%%%%%%%%%%%%%%%%%%%%%%%%%%%%%%%%%%%%%%%%%%%%%%%%%%%%%%%%%%%%%%%%
\begin{frame}[fragile,t]
\frametitle{Example: Summing a Sequence}
\framesubtitle{still still continued}
\begin{itemize}[<+->]
\item Overflow can happen to any type, so maybe...
{\scriptsize\begin{verbatim}
template<> struct SumTraits< char  > { typedef short     type; };
template<> struct SumTraits< short > { typedef int       type; };
template<> struct SumTraits< int   > { typedef long      type; };
template<> struct SumTraits< long  > { typedef long long type; };
\end{verbatim}
}
\item Now, all integral types sum to the next-largest type
\item And, we can customize:
{\scriptsize\begin{verbatim}
template<> struct SumTraits< Widget> { typedef Wocket type; };
\end{verbatim}
}
\item And all other types have a good default.

\item Can we get the compiler to produce the integer traits for us?
  \begin{itemize}
    \item Yes, but that's arguably metaprogramming, so we'll tackle
      that next time
  \end{itemize}
\end{itemize}
\vskip 12pt
\pause
\begin{center}
\Emph{Type traits provide a compile-time customization point.}
\end{center}

\end{frame}


%%%%%%%%%%%%%%%%%%%%%%%%%%%%%%%%%%%%%%%%%%%%%%%%%%%%%%%%%%%%%%%%%%%%%%
\begin{frame}[fragile,t]
\frametitle{Mapping Types to Values}
\framesubtitle{JETS input}
Another example: my JETS library (written at TerumoBCT), which serializes objects to and
from JSON text.

\begin{itemize}[<+->]
\item JSON supports a set number of types in an N-ary tree of
  key/value pairs
\item JSON has no type safety:
{\scriptsize
\begin{verbatim}
{"$type":"message", "valid":true, "Contents" : {"value":42, "pi":3.14}}
\end{verbatim}
}
\begin{itemize}
  \item To read, you first query to find out if the value can be
    interpreted as what you want, \emph{then} you read it.
{\scriptsize\begin{verbatim}
  if (node.isBool())
     myobj.valid = node.getBool();
\end{verbatim}
}
\end{itemize}
\item the JETS library (``JSON Extended Type Safety'') uses the type
  of the variable being read into to determine how to read the JSON
  stream.
\item ... and we'd like a good error message when it's not the
  expected type.
\item So we need a way to map types to strings.

\end{itemize}
\end{frame}

%%%%%%%%%%%%%%%%%%%%%%%%%%%%%%%%%%%%%%%%%%%%%%%%%%%%%%%%%%%%%%%%%%%%%%
\begin{frame}[fragile,t]
\frametitle{JETS error message traits}

So, define traits classes that map types to strings:

{\scriptsize\begin{verbatim}
template<typename T> struct fieldtraits { static const char* tname () { return "INVALID"; }};

// field traits for built in types
template<> struct fieldtraits<bool>   { static const char* tname () { return "bool"  ; }};
template<> struct fieldtraits<int >   { static const char* tname () { return "int"   ; }};
template<> struct fieldtraits<string> { static const char* tname () { return "string"; }};

template<typename T> struct fieldtraits<vector<T> > 
     { static const char* tname () { return "Array"; }};
\end{verbatim}
}
\pause
And we use it deep in the guts of the parsing....
{\scriptsize\begin{verbatim}
template<typename T> IValue& operator|| (const Field<T>& f) {
 ... much grunge in here ...
if (!ok)
    {
       ebuff_ << "Struct error, field \"" << f.name
              << "\" is not of type " << fieldtraits<T>::tname();
    }
\end{verbatim}
}
\pause
\begin{center}
\Emph{This is extensible; if you have your own type, you can
  specialize \texttt{fieldtraits} for it, and have the correct text
  show up in the error messages}
\end{center}
\end{frame}



%%%%%%%%%%%%%%%%%%%%%%%%%%%%%%%%%%%%%%%%%%%%%%%%%%%%%%%%%%%%%%%%%%%%%%
\begin{frame}[fragile,t]
\frametitle{std::numeric\_limits}
\begin{itemize}[<+->]
\item Another example: \texttt{std::numeric\_limits}, which is a
  template that has been specialized for arithmetic types:
{\scriptsize\begin{verbatim}
  std::numeric_limits<int>::is_integer == true;  // static const bool value

  numeric_limits<double>::has_infinity == true; // ditto
  
  std::numeric_limits<unsigned short>::max() == 65535 // static fn
  
\end{verbatim}
}

\item In this case, the base (non-specialized) case exists;
  \texttt{numeric\_limits<Widget>} compiles and gives reasonable results.
\item The other approach is to leave the base case undefined, in which
  case \texttt{numeric\_limits<Widget>} wouldn't compile.
\item And, you can write the specialized version for your classes if
  you want.
\end{itemize}
\pause
\begin{center}
Existence of the base case determines if types that don't have a
specialized Trait can access default behavior, or if trying to do so
is a compile-time error.
\end{center}


\end{frame}
