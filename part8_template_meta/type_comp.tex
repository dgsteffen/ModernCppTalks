\subsection{Metaprogramming Traits}
%%%%%%%%%%%%%%%%%%%%%%%%%%%%%%%%%%%%%%%%%%%%%%%%%%%%%%%%%%%%%%%%%%%%%%
\begin{frame}[fragile,t]
\frametitle{Type Traits}
\begin{itemize}[<+->]
\item Recall the earlier example from the JETS code, serializing a
  struct to JSON.
\item When we write a field, we create a new node in the DOM and then
  either write a value, or start a new subtree \Emph{(Pseudocode)}
{\scriptsize\begin{verbatim}
WriteNode(node n, int i)   { n.SetInt(i); }
WriteNode(node n, bool b)  ( n.setBool(b); }
..


template<typename T> WriteValue( Node parent, const T& val)
{
  create newnode

  if (leaf node)
    newnode.WriteNode(val)
  else // it's a root node for a subtree
    newnode.setObject();
    // Recurse
    for (member in val)
      WriteValue(newnode, member)
}
\end{verbatim}
} 
\item So, we need to know: is \texttt{T} a built-in type or a
  user-defined type (class or struct)?
\end{itemize}

\end{frame}


%%%%%%%%%%%%%%%%%%%%%%%%%%%%%%%%%%%%%%%%%%%%%%%%%%%%%%%%%%%%%%%%%%%%%%
\begin{frame}[fragile,t]
\frametitle{Type Traits}
\begin{itemize}[<+->]
\item We could define traits classes for all known built-ins (bool,
  int, short, etc) like last time.

\item Or.... we could let the compiler figure it out!

\vskip 12pt

\item The \texttt{is\_class} template class has an enum that is 1 if
  the template parameter is a class and 0 if it's not.
\item \Emph{How is it done?}
\item \texttt{Common\_6.9/jets/type\_traits.h}, it's straight out of
  Vandevoorde and Josuttis.
\vskip 12pt
\item \Emph{Warning: entering several dark and dusty corners of the
  language.  Bring a flashlight.}
\end{itemize}


\end{frame}


%%%%%%%%%%%%%%%%%%%%%%%%%%%%%%%%%%%%%%%%%%%%%%%%%%%%%%%%%%%%%%%%%%%%%%
\begin{frame}[fragile,t]
\frametitle{is\_class}
\framesubtitle{1/N}
{\scriptsize\begin{verbatim}

template<typename T>
class is_class
{
private:

   typedef char One;                // a type with size 1
   typedef struct {char a[2];} Two; // a type with size 2









public:

   enum { value = ... magic in here ... };

};

\end{verbatim}
}

\end{frame}


%%%%%%%%%%%%%%%%%%%%%%%%%%%%%%%%%%%%%%%%%%%%%%%%%%%%%%%%%%%%%%%%%%%%%%
\begin{frame}[fragile,t]
\frametitle{is\_class}
\framesubtitle{1/N}
{\scriptsize\begin{verbatim}

template<typename T>
class is_class
{
private:

   typedef char One;                // a type with size 1
   typedef struct {char a[2];} Two; // a type with size 2

   // Now for some template functions that will never get called, but have distinct return types.

   // This one takes a pointer-to-member of C, and returns a One.
   template<typename C> static One test(int C::*);

   // int C::* is a pointer-to-member of C.  This is a valid construct
   //  **even if C doesn't have any int members **

public:

   enum { value = ... magic still coming... };

};

\end{verbatim}
}

\end{frame}

%%%%%%%%%%%%%%%%%%%%%%%%%%%%%%%%%%%%%%%%%%%%%%%%%%%%%%%%%%%%%%%%%%%%%%
\begin{frame}[fragile,t]
\frametitle{is\_class}
\framesubtitle{1/N}
{\scriptsize\begin{verbatim}

template<typename T>
class is_class
{
private:

   typedef char One;                // a type with size 1
   typedef struct {char a[2];} Two; // a type with size 2

   // Now for some template functions that will never get called, but have distinct return types.

   // This one takes a pointer-to-member of C, and returns a One.
   template<typename C> static One test(int C::*);

   // This one will match anything else...
   template<typename C> static Two test(...);

public:

   enum { value = ... almost there ... };

};

\end{verbatim}
}

\end{frame}


%%%%%%%%%%%%%%%%%%%%%%%%%%%%%%%%%%%%%%%%%%%%%%%%%%%%%%%%%%%%%%%%%%%%%%
\begin{frame}[fragile,t]
\frametitle{is\_class}
\framesubtitle{1/N}
{\scriptsize\begin{verbatim}

template<typename T>
class is_class
{
private:

   typedef char One;                // a type with size 1
   typedef struct {char a[2];} Two; // a type with size 2

   // Now for some template functions that will never get called, but have distinct return types.

   // This one takes a pointer-to-member of C, and returns a One.
   template<typename C> static One test(int C::*);   // #1

   // This one will match anything else...
   template<typename C> static Two test(...);        // #2

public:

   enum { value = (sizeof(test<T>(0)) == 1) };

};

\end{verbatim}
}

\end{frame}


%%%%%%%%%%%%%%%%%%%%%%%%%%%%%%%%%%%%%%%%%%%%%%%%%%%%%%%%%%%%%%%%%%%%%%
\begin{frame}[fragile,t]
\frametitle{is\_class}
\framesubtitle{1/N}
\Emph{My head hurts.  One more time?}
\begin{itemize}[<+->]
\item We define two types that have different sizes
\item We define two overloaded functions.
\begin{itemize}
  \item The first takes a pointer-to-member and returns something with sizeof 1
  \item The second takes anything and returns something with sizeof 2
\end{itemize}
\item Arcane fact \#2: in overload resolution, vararg functions are
  always the \emph{last} choice
\item Arcane fact \#3: \texttt{int C::*} is 
\begin{itemize}
  \item a pointer-to-member-of C of type int.
  \item a valid construct \emph{even if C doesn't have any int members}.
  \item an invalid construct for things that can't have members.
\end{itemize}
\item If \texttt{T} is a class, struct, or union, both functions are
  valid and are instantiated; the first is chosen.
\item Otherwise, the first is invalid and SFINAE'd out of existence,
  the other is the only function instantiated
\item Checking \texttt{sizeof} the return type tells you which one the
  compiler prefers

\end{itemize}


\end{frame}

%%%%%%%%%%%%%%%%%%%%%%%%%%%%%%%%%%%%%%%%%%%%%%%%%%%%%%%%%%%%%%%%%%%%%%
\begin{frame}[fragile,t]
\frametitle{is\_class}
\framesubtitle{1/N}
{\scriptsize\begin{verbatim}

template<typename T>
class is_class
{
private:

   typedef char One;                // a type with size 1
   typedef struct {char a[2];} Two; // a type with size 2

   // Now for some template functions that will never get called, but have distinct return types.

   // This one takes a pointer-to-member of C, and returns a One.
   template<typename C> static One test(int C::*);  // #1

   // This one will match anything else...
   template<typename C> static Two test(...);       // #2

public:

   enum { value = (sizeof(test<T>(0)) == 1) };

};

\end{verbatim}
}

\end{frame}

%%%%%%%%%%%%%%%%%%%%%%%%%%%%%%%%%%%%%%%%%%%%%%%%%%%%%%%%%%%%%%%%%%%%%%
\begin{frame}[fragile,t]
\frametitle{Using is\_class}
\begin{itemize}[<+->]
\item Now our JETS output pseudocode looks like this:
{\scriptsize\begin{verbatim}
WriteNode(node n, int i)   { n.SetInt(i); }
WriteNode(node n, bool b)  ( n.setBool(b); }
...

template<typename T> WriteValue( Node parent, const T& val)
{
  create newnode

  if ( is_class<T>::value )    // <-- conditional
    newnode.setObject();
    for (member in val)
      WriteValue(newnode, member)

  else 
    newnode.WriteNode(val)
}
\end{verbatim}
}
\item Note that the conditional reads \texttt{if (1)} or \texttt{if (0)} upon instantiation.
\item If the compiler is doing any optimization at all, the false
  branch will be eliminated completely.

\end{itemize}


\end{frame}

%%%%%%%%%%%%%%%%%%%%%%%%%%%%%%%%%%%%%%%%%%%%%%%%%%%%%%%%%%%%%%%%%%%%%%
\begin{frame}[fragile,t]
\frametitle{Using is\_class}
\framesubtitle{continued}
\begin{itemize}[<+->]

\item If T is an int we get
{\scriptsize\begin{verbatim}
template<> WriteValue<int>( Node parent, const int& val)
{
  create newnode
  newnode.WriteNode(val)
}
\end{verbatim}
}
\item If it's a widget, we get
{\scriptsize\begin{verbatim}
template<> WriteValue<widget>( Node parent, const widget& val)
{
  create newnode
  newnode.setObject();
    for (member in val)
      WriteValue(newnode, member)
}
\end{verbatim}
}
\item Effectively, \Emph{the compiler writes the function for us}.
\end{itemize}


\end{frame}


%%%%%%%%%%%%%%%%%%%%%%%%%%%%%%%%%%%%%%%%%%%%%%%%%%%%%%%%%%%%%%%%%%%%%%
\begin{frame}[fragile,t]
\frametitle{Traits Metaprogramming}
\begin{itemize}[<+->]
\item This technique can be generalized to detecting, at compile time:
\begin{itemize}[<+->]
\item If a class has a specific member function or not.  (Introspection!)
\item If a member function is virtual
\item If A inherits from B
\item If A and B both inherit from C.
\item If a function parameter is a reference, or const, or a
  const-reference, or...
\end{itemize}
\item If A is-convertable-to B
\item ...
\end{itemize}
\end{frame}
