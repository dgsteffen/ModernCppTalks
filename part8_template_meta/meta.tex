\subsection{The First Metaprogram}
%%%%%%%%%%%%%%%%%%%%%%%%%%%%%%%%%%%%%%%%%%%%%%%%%%%%%%%%%%%%%%%%%%%%%%
\begin{frame}[fragile,t]
\frametitle{The ``Unruh Primes'' program}
\begin{itemize}[<+->]
\item At a C++ standardization meeting in 1994, Erwin Unruh wrote the
  first template metaprogram.
\item It didn't compile...
\item but trying to compile it produced \Emph{lists of prime numbers}
  in the error messages.
\item It was discovered that templates are Turing-complete.
  \begin{itemize}
  \item Compilers can be fed code that forces them, \emph{at compile
    time}, to perform arbitrary computations.
  \item In real life, there are hard limits on what you can really
    do (compilers aren't really build for this).
    \end{itemize}
\item Alas, Unruh's progream doesn't really work on modern compilers, becuase their error messages
  have been improved
\end{itemize}

\end{frame}


\subsection{Compile-time factorial}
%%%%%%%%%%%%%%%%%%%%%%%%%%%%%%%%%%%%%%%%%%%%%%%%%%%%%%%%%%%%%%%%%%%%%%

\begin{frame}[fragile,t]
\frametitle{Simple example: compile-time factorial}
\begin{itemize}[<+->]
\item Everyone's favorite example of recursion...
\item Start with the general case:
{\scriptsize\begin{verbatim}
template<int N> struct Factorial
{
   enum { result = N * Factorial<N-1>::result };
};

\end{verbatim}
}
\item Remember the base case (so we don't recurse forever)
{\scriptsize\begin{verbatim}
template<> struct Factorial<0> { enum {result = 1}; };

\end{verbatim}
}
\item Then use it...
{\scriptsize\begin{verbatim}
int main() {  return Factorial<5>::result;  }

\end{verbatim}
}

\item The code produced is effectively as if we had written:
{\scriptsize\begin{verbatim}
int main() { return 120; }

\end{verbatim}
}

\item Exactly zero run-time instructions are produced.


\end{itemize}
\end{frame}

%%%%%%%%%%%%%%%%%%%%%%%%%%%%%%%%%%%%%%%%%%%%%%%%%%%%%%%%%%%%%%%%%%%%%%

\begin{frame}[fragile,t]
\frametitle{Why Enums?}
\begin{itemize}[<+->]
  \item In old compilers, enums were the only way to have compile-time
    integer constants.
  \item Now we also have static consts and in-class inializers.  We
    could have just as easily written:
{\scriptsize\begin{verbatim}
template<int N> struct Factorial
{
   static int const result = N * Factorial<N-1>::result;
};

template<> struct Factorial<0> { static const int result = 1; };

\end{verbatim}
}
\item Enums are still preferred.
\begin{itemize}
  \item Static constant members are still l-values...
  \item Pass one to \texttt{void foo(const int\&)} forces the compiler
    to instantiate it and stick it somewhere
  \item On the other hand, enums are technically constant r-values (or
    some such) and act like integer literals
\end{itemize}

\end{itemize}
\end{frame}

%%%%%%%%%%%%%%%%%%%%%%%%%%%%%%%%%%%%%%%%%%%%%%%%%%%%%%%%%%%%%%%%%%%%%
\subsection{Compile-time loops}

\begin{frame}[fragile,t]
\frametitle{Compile-time loops}
\begin{itemize}[<+->]
\item As a second example, let's loop from 10 down to 1 at compile
  time:
{\scriptsize\begin{verbatim}
template<int N> struct Looper
{
   static void exec() {
     cout << "At " << N << endl;
     Looper<N-1>::exec();
   }
};

template<> struct Looper<0>
{
   static void exec() {}
};

int main()
{
   Looper<10>::exec();
}

\end{verbatim}
}

\item Note the same recursive template technique

\end{itemize}

\end{frame}


%%%%%%%%%%%%%%%%%%%%%%%%%%%%%%%%%%%%%%%%%%%%%%%%%%%%%%%%%%%%%%%%%%%%%

\begin{frame}[fragile,t]
\frametitle{Compile-time loops}
\begin{itemize}[<+->]
\item Swap the order of the statements to count up from 1 to 10:
{\scriptsize\begin{verbatim}
template<int N> struct Looper
{
   static void exec() {
     Looper<N-1>::exec();
     cout << "At " << N << endl;
   }
};
\end{verbatim}
}

\item Metaprogramming is much closer to \Emph{functional} programming
  \begin{itemize}[<+->]
  \item All values are immutable
  \item Recursion rather than iteration
  \item Apparently learning Haskell makes this much easier...
  \end{itemize}
\vskip 12pt
\item Keep in mind that there are real limits to how much template
  recursion a compiler is willing to put up with
\item GCC provides a command-line flag \texttt{-ftemplate-depth=n} to
  set the max recusion depth; defaults to 900
\item Compile times and memory use can get really big.
\end{itemize}

\end{frame}
