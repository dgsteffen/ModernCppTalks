\subsection{Basics of Variadic Tempates}
%%%%%%%%%%%%%%%%%%%%%%%%%%%%%%%%%%%%%%%%%%%%%%%%%%%%%%%%%%%%%%%%%%%%%%
\begin{frame}[fragile,t]
\frametitle{Variadic Templates}
\begin{itemize}[<+->]
\item The most mind-blowing new feature in \CC11
\item The definitive talk: Andrei Alexandrescu, ``Variadic Templates
  are Funadic'', GoingNative 2012:
  \url{https://channel9.msdn.com/Events/GoingNative/GoingNative-2012/Variadic-Templates-are-Funadic}
\item The following slides are directory stolen from Andrei.  Maybe
  we'll just watch him instead...
\item Peter Sommerlad, ``Variadic Templates in \CC11 / \CC14 - An
  Introduction'', CppCon 2015: \url{https://www.youtube.com/watch?v=R1G3P5SRXCw}
\item Micha\l{} Dominiak, ``Variadic expansion in examples'', CppCon 2016 \url{https://youtu.be/Os5YLB5D2BU}
\item Jason Turner (again) has some nice examples of variadic expansion.
\end{itemize}
\end{frame}


%%%%%%%%%%%%%%%%%%%%%%%%%%%%%%%%%%%%%%%%%%%%%%%%%%%%%%%%%%%%%%%%%%%%%%
\begin{frame}[fragile,t]
\frametitle{Motivation}
\begin{itemize}[<+->]
\item Define typesafe variadic functions
  \begin{itemize}
    \item C99 macros safer than \CC03 variadic functions?!
    \item Forwarding with before/after hooks
    \begin{itemize}  \item \emph{Uses ``universal references'', about
        which more later}
  \end{itemize}
  \end{itemize}
\item Define algebraic types without contortions
  \begin{itemize}[<+->]
  \item Sum types (\texttt{variant})
    \item Example: \texttt{NullableType<T>} in the Terminal: it should
      have a ctor that matches T's ctor, but how to write?
  \item Product types (\texttt{tuple})
  \end{itemize}
\item Specify settings and parameters in policy-based designs
\end{itemize}
\end{frame}


%%%%%%%%%%%%%%%%%%%%%%%%%%%%%%%%%%%%%%%%%%%%%%%%%%%%%%%%%%%%%%%%%%%%%%
\begin{frame}[fragile,t]
\frametitle{Pre-\CC11}
\framesubtitle{Peter Sommerlad, CppCon 2015}
\begin{itemize}[<+->]
\item Alexandrescu, ``Modern C++ Design'', and Boost library had heavy
  infrastructure for Tuple and other things:
  \begin{itemize}
    \item \texttt{boost::bind()}
    \item \texttt{boost::tuple<...>}
    \item \texttt{Loki::Typelist<Head,Tail>}
  \end{itemize}
\item Requires tedious repetition, or
\item Serious macro magic: Boost has a macro metaprogramming library!
{\scriptsize\begin{verbatim}
#define BOOST_VARIANT_AUX_DECLARE_PARAMS \
  BOOST_PP_ENUM( \
    BOOST_VARIANT_LIMIT_TYPES \
  , BOOST_VARIANT_AUX_DECLARE_PARAMS_IMPL \
  , T \
  ) \
/**/
\end{verbatim}
}
\item (yes, those are \emph{recursive macros!} Chthulu shall rise...)
\item Both cases have a built-in maximum.
\end{itemize}
\end{frame}



%%%%%%%%%%%%%%%%%%%%%%%%%%%%%%%%%%%%%%%%%%%%%%%%%%%%%%%%%%%%%%%%%%%%%%
\begin{frame}[fragile,t]
\frametitle{Fundamentals}
{\scriptsize\begin{verbatim}
template <typename... Ts>  // Ts is "template parameter pack"
class C { ... };

template <typename... Ts>
void fun (const Ts&... vs> // vs is "value parameter pack"
{ ... } 
\end{verbatim}
}
\begin{itemize}[<+->]

\item These are new kinds; they are unlike any other construct in the language.

\item \texttt{Ts} is not a type; \texttt{vs} is not a value!
{\scriptsize\begin{verbatim}
typedef Ts MyList; // error!
Ts var;            // error
auto copy = cs;    // error
\end{verbatim}
}
\item \texttt{Ts} is an alias for a list of types
\item \texttt{vs} is an alias for a list of values
\item Either list may be potentially empty (!)
\end{itemize}
\end{frame}


%%%%%%%%%%%%%%%%%%%%%%%%%%%%%%%%%%%%%%%%%%%%%%%%%%%%%%%%%%%%%%%%%%%%%%
\begin{frame}[fragile,t]
\frametitle{Names, Declarations, Expansion}
\begin{itemize}[<+->]
\item If the ... is in front of the name, {\bf define} the name as a parameter pack
{\scriptsize\begin{verbatim}
template < typename ...Ts >  // Ts is a parameter pack
\end{verbatim}
}
\item if the ... is after the name, {\bf expand} the name into a
  comma-separated list
{\scriptsize\begin{verbatim}
Ts... // produces: T1, T2, .... Tn
\end{verbatim}
}
\item In between two names, define second name as a list of
  parameters, given by the first (it's ``in front of'' the second one)
{\scriptsize\begin{verbatim}
template <typename ...Ts> // as above
void foo (const Ts& ... vs); // vs is a pack
\end{verbatim}
}
\item Think of it as ``smart'' textual expansion (it really isn't
  textual expansion, it's much deeper)
\end{itemize}

\end{frame}


%%%%%%%%%%%%%%%%%%%%%%%%%%%%%%%%%%%%%%%%%%%%%%%%%%%%%%%%%%%%%%%%%%%%%%
\begin{frame}[fragile,t]
\frametitle{Operations on Parameter Packs}
\begin{itemize}[<+->]
\item Can only be used in one of two ways
\item Apply the \Emph{new} \texttt{sizeof...} operator to it:
{\scriptsize\begin{verbatim}
size_t items = sizeof...(ts); // or vs
\end{verbatim}
}
\begin{itemize}
  \item ``Completely useless'' according to Alexandrescu, but we'll
    use it later
\end{itemize}
\item Expand the pack into a comma-separated list
{\scriptsize\begin{verbatim}
template<typename... Ts>
void foo(Ts const&... vs)
{
   bar (3.14, vs..., "hi");  // expand vs (safe commas)
}
\end{verbatim}
}
\item ... and nothing else!
\end{itemize}
\end{frame}


%%%%%%%%%%%%%%%%%%%%%%%%%%%%%%%%%%%%%%%%%%%%%%%%%%%%%%%%%%%%%%%%%%%%%%
\begin{frame}[fragile,t]
\frametitle{A quick example}
\begin{itemize}[<+->]
\item It's good to stay grounded, so: print any number of things of
  any type to a line.
{\scriptsize\begin{verbatim}

void print() { cout << endl; }   // base case

template<typename T, typename ...Ts>
void print (const T& t, const Ts& ...ts)
{
   cout << t << ' ';   // peel off first arg
   print(ts...);       // "recurse" on the rest
}


int main()
{
   print ("HI", 3, 1.4);
}

// produces this:
HI 3 1.4
\end{verbatim}
}
\end{itemize}
\end{frame}



%%%%%%%%%%%%%%%%%%%%%%%%%%%%%%%%%%%%%%%%%%%%%%%%%%%%%%%%%%%%%%%%%%%%%%
\begin{frame}[fragile,t]
\frametitle{Expansion Rules}

\begin{tabular}{ll}
\hline
Use & Expansion \\ \hline
\texttt{Ts...} & \texttt{T1, T2, ... Tn} \\
\texttt{Ts\&\&...}& \texttt{T1\&\&, T2\&\&, ... Tn\&\&} \\
\texttt{x<Ts,Y>::z...}& \texttt{X<T1,Y>::z, ... X<Tn,Y>::z} \\
\texttt{x<Ts\&, Us>...}& \texttt{x<T1\&, U1>, x<T2\&, U2>, ... x<Tn\&, Un>} \\
\texttt{func(5, vs)...}& \texttt{func(5, v1), func(5, v2), ... func(5, vn)} \\ \hline
\end{tabular}

\begin{itemize}[<+->]
\item Assume for demonstration purposes:
\begin{itemize}
  \item Ts is A, B, C
  \item Us is Q, R, S
  \item vs is x, y, z
\end{itemize}
\item Then:
    \begin{itemize}
      \item Ts.... $\to$ A, B, C
      \item X<Ts, Z>... $\to$ X<A,Z>, X<B,Z>, X<C,Z>
      \item x<Ts, Us>... $\to$ x<A,Q>, x<B,R>, x<C,S>
      \item func(5, vs)... $\to$ func(5,x), func(5,y), func(5,z)
      \item f(vs, vs)... $\to$ f(x,x), f(y,y), f(z,z)  
\end{itemize}
\item dot products, cross products, Cartesian products are all
  possible
\item Examples coming soon...
\end{itemize}


\end{frame}


%%%%%%%%%%%%%%%%%%%%%%%%%%%%%%%%%%%%%%%%%%%%%%%%%%%%%%%%%%%%%%%%%%%%%%
\begin{frame}[fragile,t]
\frametitle{Expansion locations}
\framesubtitle{Where can we expand?}
\begin{itemize}[<+->]
\item Initializer lists:
{\scriptsize\begin{verbatim}
any a[] = { vs... };
\end{verbatim}
}
\item Base specifiers:
{\scriptsize\begin{verbatim}
template< typename... Ts> struct C : public Ts... {};

template< typename... Ts> struct D : public Box<Ts>... {};
\end{verbatim}
}

\item Member initializer lists:
{\scriptsize\begin{verbatim}
// inside struct D...
template< typename... Us >
D(Us... vs) : Box<Ts>(vs)... {}
\end{verbatim}
}

\item Template argument lists:
{\scriptsize\begin{verbatim}
std::map<Ts...> m;
\end{verbatim}
}

\item Exception specifications (which are gone from the language)
\item Attribute lists (but never mind)
{\scriptsize\begin{verbatim}
struct [[ Ts... ]] IAmFromTheFuture {};
\end{verbatim}
}

\item Lambda capture lists
{\scriptsize\begin{verbatim}
template <class... Ts> void fun(Ts... vs) {
  auto g = [&vs...] { return foo(vs...); }
  g();
}
\end{verbatim}
}
\end{itemize}
\end{frame}

%%%%%%%%%%%%%%%%%%%%%%%%%%%%%%%%%%%%%%%%%%%%%%%%%%%%%%%%%%%%%%%%%%%%%%
\begin{frame}[fragile,t]
\frametitle{Where can't we expand?}
\begin{itemize}[<+->]
\item Declarations (to declare something for every element in a pack)
\item Statements (to execute a statement for every element in a pack)
\begin{itemize}
  \item Mainly because what happens if the pack is empty?

\end{itemize}
\item We will come back to this later.
\end{itemize}
\end{frame}


%% %%%%%%%%%%%%%%%%%%%%%%%%%%%%%%%%%%%%%%%%%%%%%%%%%%%%%%%%%%%%%%%%%%%%%%
%% \begin{frame}[fragile,t]
%% \frametitle{Multiple expansions}
%% \begin{itemize}[<+->]
%% \item Expansion proceeds outwards
%% \item These are all different expansions:
%% {\scriptsize\begin{verbatim}
%% template< class... Ts> void fun(Ts... vs) {
%%   gun(A<Ts...>::hun(vs)...); // A<T1, T2, T3>::hun(v1), A<T1
%%   gun(A<Ts...>::hun(vs...)); // A<T1, T2>::hun(v1, v2)
%%   gun(A<Ts>::hun(vs)...);    // A<T1>::hun(v1), A<T2>::hun(v2)
%% }
%% \end{verbatim}
%% }
%% \end{itemize}
%% \end{frame}

%% %%%%%%%%%%%%%%%%%%%%%%%%%%%%%%%%%%%%%%%%%%%%%%%%%%%%%%%%%%%%%%%%%%%%%%
%% \begin{frame}[fragile,t]
%% \frametitle{Andrei shows off: VVTTs}
%% {\scriptsize\begin{verbatim}
%% template <
%%    template T,
%%    template <
%%       template<class...> class... Policies
%%    >
%%  >
%% class Variadic_Variadic_Template_Templates;
%% \end{verbatim}
%% }
%% \begin{itemize}[<+->]
%% \item Yeah, it works.
%% \item Template: fixed parameter T,
%% \item parameterize with variable number of policies
%% \item Each of which takes a variable number of arguments
%% \item Don't ask me how it works...
%% \end{itemize}
%% \end{frame}

%%%%%%%%%%%%%%%%%%%%%%%%%%%%%%%%%%%%%%%%%%%%%%%%%%%%%%%%%%%%%%%%%%%%%%
\begin{frame}[fragile,t]
\frametitle{Example}
\begin{itemize}[<+->]
\item Pattern matching:
{\scriptsize\begin{verbatim}
template <class T1, class T2> bool isOneOf( const T1& a, const T2& b)
{  return a == b ; }

template <class T1, class T2, class... Ts>
bool isOneOf(const T1& a, const T2& b, const Ts&... vs)
{
   return a == b || isOneOf(a, vs...);
}

assert( isOneOf (1, 2, 3.5, 4, 1, 2 ) );
\end{verbatim}
}
\item Very fast, just comparisons and jumps.
\item Not recursion: each call is a function of the same name with fewer arguments
\end{itemize}
\end{frame}

\subsection{Variadic Expansion}
%%%%%%%%%%%%%%%%%%%%%%%%%%%%%%%%%%%%%%%%%%%%%%%%%%%%%%%%%%%%%%%%%%%%%%
\begin{frame}[fragile,t]
\frametitle{Expansion: calling a function for every argument}
\begin{itemize}[<+->]
\item Earlier we mentioned that you can't expand statements (to execute a statement for every element in a pack)
\item We'd like to do this:
{\scriptsize\begin{verbatim}
template <typename... Ts> void foo ( Ts... vs)
{
  bar(vs)...; // execute for each element -- doesn't work!
  // "expected expression"
  // "expected ; before ..."
  // "pack not expanded"
}
\end{verbatim}
}
\item A solution has been developed:  Use an initializer list
{\scriptsize\begin{verbatim}
template<typename... Ts>
void foo (const Ts& ... vs)
{
   (void) std::initializer_list<int>{ ( bar(vs), 0 )... };
}
\end{verbatim}
}
\item How does it work?
\end{itemize}
\end{frame}


%%%%%%%%%%%%%%%%%%%%%%%%%%%%%%%%%%%%%%%%%%%%%%%%%%%%%%%%%%%%%%%%%%%%%%
\begin{frame}[fragile,t]
\frametitle{Expansion: calling a function for every argument}
\begin{itemize}[<+->]
\item The interesting bit expands to:
{\scriptsize\begin{verbatim}
(void) std::initializer_list<int>{ (bar(v1),0) , (bar(v2),0) ... };
\end{verbatim}
}
\item Recall that the comma operator evaluates the first operand and
  discards the result, then evaluates the second and returns that
  value.
\item Each ``element'' in the \texttt{initializer\_list} is :
\begin{itemize}
  \item a function call that is exectuted, and the return value
    ignored;
  \item An integer constant '0', which is the result of the expression
\end{itemize}
\item The compiler is happy to make a list of zeroes and then ignore it.
\vskip 12pt
\item We can use this to explore the details of parameter pack expansion.

\end{itemize}
\end{frame}


%%%%%%%%%%%%%%%%%%%%%%%%%%%%%%%%%%%%%%%%%%%%%%%%%%%%%%%%%%%%%%%%%%%%%%
\begin{frame}[fragile,t]
\frametitle{Expansion: calling a function for every argument}
The setup:
{\scriptsize\begin{verbatim}
#include <iostream>
using namespace std;

template<typename T>
void print_impl(const T& t) { cout << t << ' '; }

template<typename ...Ts>
void print (const Ts& ...ts)
{
   (void)initializer_list<int> { (print_impl(ts), 0)... };
   cout << endl;
}

int main() {
  print ("HI", true, 42);
}

// produces

HI 1 42
\end{verbatim}
}

\end{frame}


%%%%%%%%%%%%%%%%%%%%%%%%%%%%%%%%%%%%%%%%%%%%%%%%%%%%%%%%%%%%%%%%%%%%%%
\begin{frame}[fragile,t]
\frametitle{Expansion: calling a function for every argument}
Now add an intermediatry to experiment with:
{\scriptsize\begin{verbatim}
template<typename T>
void print_impl(const T& t) { cout << t << ' '; }

template<typename ...Ts>
void print (const Ts& ...ts)
{
   (void)initializer_list<int> { (print_impl(ts), 0)... };
   cout << endl;
}

template<typename... Ts>
void expand(Ts... vs)
{
  // Fiddle with this line to get interesting results
   (void) initializer_list<int> { ( print(vs...), 0) }; 
}

int main() {
  expand ("HI", true, 42); // just like print("HI", true, 42)
}

// produces
HI 1 42

\end{verbatim}
}
\end{frame}

%%%%%%%%%%%%%%%%%%%%%%%%%%%%%%%%%%%%%%%%%%%%%%%%%%%%%%%%%%%%%%%%%%%%%%
\begin{frame}[fragile,t]
\frametitle{Expansion: Demo}
Go play with this:
{\tiny\begin{verbatim}
#include <iostream>
using namespace std;  // bad in real code, good for slides

template<typename T>
void print_impl(const T& t) { cout << t << ' '; }

template<typename ...Ts>
void print (const Ts& ...ts)
{
   (void)initializer_list<int> { (print_impl(ts), 0)... };
   cout << endl;
}

template<typename... Ts>
void expand(Ts... vs)
{
   (void) initializer_list<int> 
   { 
     ( print(vs...), 0)
      //( print(vs), 0)... 
      //( print(vs..., vs), 0)...
      //( print(vs, vs, vs), 0)...
      //( print(vs, vs..., vs), 0)...
   }; 
}

int main()
{
   expand ("HI", true, 42);
}
\end{verbatim}
}
\end{frame}

\subsection{Real-Life Example}

%%%%%%%%%%%%%%%%%%%%%%%%%%%%%%%%%%%%%%%%%%%%%%%%%%%%%%%%%%%%%%%%%%%%%%
\begin{frame}[fragile,t]
\frametitle{Example: what Terminal could be}
\framesubtitle{Motivation}
Problem statement: Terminal message processing.
\begin{itemize}[<+->]
\item A JSON message arrives, with a field ``\$type'' holding the name
  of a class
  \begin{itemize}
  \item Create an object of that type
  \item Deserialize the JSON message into it
  \item Execute some code to do something
  \end{itemize}
\item The problem is to turn a \Emph{runtime value} (the string in the \$type
  field) into an object of the correct type (a \Emph{compile-time
    thing}) and then run a specific function (also a
  \Emph{compile-time} thing).
\end{itemize}
\end{frame}

%%%%%%%%%%%%%%%%%%%%%%%%%%%%%%%%%%%%%%%%%%%%%%%%%%%%%%%%%%%%%%%%%%%%%%
\begin{frame}[fragile,t]
\frametitle{Terminal Message Handling}
\framesubtitle{The Obvious Solution}
\begin{itemize}[<+->]
\item The obvious solution:
{\scriptsize\begin{verbatim}
void OnIncomingMessage(const string& msg)

  IJetStream in;    
  in.Parse(msg);

  string type;      
  in || field("$type", type);

  if (type == "Camel")      { Camel c; in || c; ... Camel stuff }

  else if (type == "Horse") { Horse h; in || h; ... ride Horse  }

  else if (type == "Rhino") { Rhino r; in || r; ... ride Rhino? }
  }
\end{verbatim}
}

\item ... but there are 50 messages ...
\item ... many of which are handled by other Reusable modules ...
\item The Device devs have to edit and maintain this huge chunk of
  code.
\item Note: \Emph{every time this is touched} the whole thing should
  be revalidated.
\vskip 12pt Can we do better?  Sure!
\end{itemize}


\end{frame}




%%%%%%%%%%%%%%%%%%%%%%%%%%%%%%%%%%%%%%%%%%%%%%%%%%%%%%%%%%%%%%%%%%%%%%
\begin{frame}[fragile,t]
\frametitle{Terminal Message Handling}
\framesubtitle{Refactoring 1}
%\begin{center} \Emph{Refactor!}\hskip 1in \end{center}
\Emph{Refactor!}
\begin{columns}[t]
\pause
\column{.4\textwidth}
First, MessageHandler fns:
{\scriptsize\begin{verbatim}
  IJetStream in;
  in.Parse(msg);

  string type;
  in || field("$type", type);

  if (type == "Camel") {    
    Camel c; 
    in || c;
    HandleMsg(c);
  }
  else if (type == "Horse") { 
    Horse h; 
    in || h;
    HandleMsg(h);
  }
  else if (type == "Rhino") { 
    Rhino r; 
    in || r;
    HandleMsg(r);
  }
\end{verbatim}
}
\pause
\column{.6\textwidth}
Next, template the deserialization:
{\scriptsize\begin{verbatim}
template<typename T> void Dispatch(IJetStream& in)
{
  T t;
  in || t;
  HandleMsg(t);
}


  IJetStream in;
  in.Parse(msg);

  string type;
  in || field("$type", type);


  if      (type == "Camel") Dispatch<Camel>(in);
  else if (type == "Horse") Dispatch<Horse>(in); 
  else if (type == "Rhino") Dispatch<Rhino>(in);

\end{verbatim}
}

\end{columns}

\end{frame}


%%%%%%%%%%%%%%%%%%%%%%%%%%%%%%%%%%%%%%%%%%%%%%%%%%%%%%%%%%%%%%%%%%%%%%
\begin{frame}[fragile,t]
\frametitle{The Essence of Message Handling}
\begin{itemize}[<+->]
\item Now we are down to the essiential code: \Emph{Associating a
  string with a type}.
\item Much better, but we still have a giant block of \texttt{if
  ... else} code to maintain
\vskip 12pt
\item Let's get the compiler to write that if-block for us.
\vskip 12pt
\item For the sake of the example (and looking at assemby), let's
  simplify the example:
  \begin{itemize}
    \item Associate integers to types (rather than strings)
    \item ``MessageHandler'' returns an int
    \item When that type comes in, dispatch to the handler and see
      what comes back.
\end{itemize}
\item Oversimplified (associating integers to other integers at
  compile time is easy, just use traits) but it's a good example.
\end{itemize}
\end{frame}



%%%%%%%%%%%%%%%%%%%%%%%%%%%%%%%%%%%%%%%%%%%%%%%%%%%%%%%%%%%%%%%%%%%%%%
\begin{frame}[fragile,t]
\frametitle{Simplified Example}
\begin{itemize}[<+->]
\item Our example, excluding the JSON stuff, now amounts to this:
{\scriptsize\begin{verbatim}
struct Camel { static constexpr int type = 1; };
struct Horse { static constexpr int type = 2; };
struct Rhino { static constexpr int type = 3; };

inline int HandleMsg(Camel&) { return 13;}
inline int HandleMsg(Horse&) { return 17;}
inline int HandleMsg(Rhino&) { return 23;}

template<typename T> int Dispatch()
{
      T t;
      return Result(t);
} 

int main(int argc, char** argv)
{
   if      (argc == Camel::type) return Dispatch<Camel>();
   else if (argc == Horse::type) return Dispatch<Horse>();
   else if (argc == Rhino::type) return Dispatch<Rhino>();
   return -1;
}

\end{verbatim}
}

\item Note: value to compare with now in message type
\end{itemize}

\end{frame}


%%%%%%%%%%%%%%%%%%%%%%%%%%%%%%%%%%%%%%%%%%%%%%%%%%%%%%%%%%%%%%%%%%%%%%
\begin{frame}[fragile,t]
\frametitle{Variadic Message Handling}
\framesubtitle{Live Demo}
{\tiny\begin{verbatim}
// as above, except now...

template<typename ThisOne, typename... TheOthers>
struct Dispatcher
{
   static int eval(int input)
      {
         if (input == ThisOne::type)
            return Dispatch<ThisOne>();            
         else
            return Dispatcher<TheOthers...>::eval(input);
      }
};

template<typename T> struct Dispatcher<T> 
{ 
   static int eval(int input) 
      { 
         if (input == T::type) return Dispatch<T>();
         return -1;
      }
};

int main(int argc, char** argv)
{
   return Dispatcher<Camel, Horse, Rhino>::eval(argc);
}
\end{verbatim}
}

\end{frame}


%%%%%%%%%%%%%%%%%%%%%%%%%%%%%%%%%%%%%%%%%%%%%%%%%%%%%%%%%%%%%%%%%%%%%%
\begin{frame}[fragile,t]
\frametitle{Compile-time nested if/else statement}
\begin{itemize}[<+->]
\item The compiler now generates the big if/else block for us.
\item The resulting code is \Emph{exactly} the same as if we had
  written it by hand.
\item We could deliver this to the device teams, all they need to know
  is the list of message types, the library does the rest.

\vskip 12pt
\item The ``real'' terminal doesn't do this, because it has to work on
  VxWorks 6.9... the result is \emph{much} less efficient, has more stuff at
  runtime (or at least at program startup), and is generally more grungy.
\end{itemize}
\end{frame}


\subsection{Variadic Summary}
%%%%%%%%%%%%%%%%%%%%%%%%%%%%%%%%%%%%%%%%%%%%%%%%%%%%%%%%%%%%%%%%%%%%%%
\begin{frame}[fragile,t]
\frametitle{Variadic Summary...}
\begin{itemize}[<+->]
\item Variadic templates are template classes or functions that take
  arbitrary numbers of arguments
\item This treatment has barely scratched the surface
\item Along with ``Universal References'', allows perfect forwarding
  of function arguments
\begin{itemize}
  \item Consider how std::vector<T>::emplace\_back works...
{\scriptsize\begin{verbatim}
vector<int> vi   ;  vi.emplace_back(43);

vector<Widget> vw; vi.emplace_back(Wocket, Sprocket, Zoink);
\end{verbatim}
}
\end{itemize}
\item What are Universal References? I'm glad you asked...
\end{itemize}
\end{frame}

%% %%%%%%%%%%%%%%%%%%%%%%%%%%%%%%%%%%%%%%%%%%%%%%%%%%%%%%%%%%%%%%%%%%%%%%
%% \begin{frame}[fragile,t]
%% \frametitle{Motivation}
%% \end{frame}

%% %%%%%%%%%%%%%%%%%%%%%%%%%%%%%%%%%%%%%%%%%%%%%%%%%%%%%%%%%%%%%%%%%%%%%%
%% \begin{frame}[fragile,t]
%% \frametitle{Motivation}
%% \end{frame}

%% %%%%%%%%%%%%%%%%%%%%%%%%%%%%%%%%%%%%%%%%%%%%%%%%%%%%%%%%%%%%%%%%%%%%%%
%% \begin{frame}[fragile,t]
%% \frametitle{Motivation}
%% \end{frame}
