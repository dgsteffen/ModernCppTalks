\subsection{Constexpr Intro}

\newcommand{\cexpr}{\texttt{constexpr }}

%%%%%%%%%%%%%%%%%%%%%%%%%%%%%%%%%%%%%%%%%%%%%%%%%%%%%%%%%%%%%%%%%%%%%%
\begin{frame}[fragile,t]
\frametitle{Constexpr }
\begin{itemize}[<+->]
\item A \Emph{\cexpr} expression is a compile-time constant evaluated
  by the \emph{compiler} rather than the code.
\item Intended to replace many uses of template metaprogramming with
  something simpler (looks like ``normal'' \CC).
\item Advantages:
  \begin{itemize}
    \item No runtime cost, no execution time, minimal footprint
    \item Errors found at compile or link time
    \item No synchronization / startup concerns. (Compare to the
      order-of-global-statics-initialization fiasco.)
  \end{itemize}
\item The most important feature in \CC11?  Debatable
\item The biggest?  Probably.
  \begin{itemize}
  \item The Clang team reports: \cexpr took as much work as the rest
    of \CC11 put together
  \item Introduces a new dimension of compile-time programming.
  \end{itemize}

\item \Emph{Much} (\CC11) \Emph{Much} (\CC14) easier than template metaprogramming!
\end{itemize}
\end{frame}


%%%%%%%%%%%%%%%%%%%%%%%%%%%%%%%%%%%%%%%%%%%%%%%%%%%%%%%%%%%%%%%%%%%%%%
\begin{frame}[fragile,t]
\frametitle{\cexpr}
\begin{itemize}[<+->]
\item \Emph{new keyword}: \cexpr
  \begin{itemize}
  \item Chosen because it's unlikely to conflict with names in existing code
  \end{itemize}
\item implies \texttt{const}
\item \cexpr can be applied to:
\begin{itemize}
  \item values 
  \item functions
  \item constructors (constexpr user-defined types)
\end{itemize}
\end{itemize}
\end{frame}




%%%%%%%%%%%%%%%%%%%%%%%%%%%%%%%%%%%%%%%%%%%%%%%%%%%%%%%%%%%%%%%%%%%%%%
\begin{frame}[fragile,t]
\frametitle{Constexpr resources}
Oddly, I haven't found ``The'' \cexpr talk.  Here are some resources:
\begin{itemize}[<+->]
\item Scott Schurr, CppCon 2015: a very gentle intro.
  \url{https://www.youtube.com/watch?v=fZjYCQ8dzTc}
\item Scott Scurr, part 2.
  \url{https://www.youtube.com/watch?v=qO-9yiAOQqc}
\item Dietmar K\"{u}hl, ``Constant Fun'' CppCon 2016 (I like this one better)
  \url{https://www.youtube.com/watch?v=qO-9yiAOQqc}
\item Search youtube and channel9 for ``CppCon constexpr'', you'll
  find other talks on specific things.
\item Jason Turner's YouTube podcast series.
\end{itemize}

\end{frame}


%%%%%%%%%%%%%%%%%%%%%%%%%%%%%%%%%%%%%%%%%%%%%%%%%%%%%%%%%%%%%%%%%%%%%%
\begin{frame}[fragile,t]
\frametitle{Review of Variables and Constants}
\framesubtitle{Edouard from quasardb}
\begin{itemize}[<+->]
\item Normal variable
{\scriptsize\begin{verbatim}
int val = 3;
\end{verbatim}
}
\begin{itemize}[<+->]
  \item ``3'' is known at compile time
  \item \texttt{val} doesn't exist until runtime
  \item Value changes at runtime
  \item Could have been initialized to \texttt{foo()} just as easily
\end{itemize}
\vskip 24pt
\item Const variable
{\scriptsize\begin{verbatim}
const int val = foo(); // 'val' exists at runtime, value set at runtime,
                       // can't change thereafter

\end{verbatim}
}

\item[] Const variables have normal lifetime, they just can't be written
  to after construction.

\end{itemize}
\end{frame}

%%%%%%%%%%%%%%%%%%%%%%%%%%%%%%%%%%%%%%%%%%%%%%%%%%%%%%%%%%%%%%%%%%%%%%
\begin{frame}[fragile,t]
\frametitle{static consts (1)}
\begin{itemize}[<+->]

\item Static const variable (now we're getting somewhere)
{\scriptsize\begin{verbatim}
static const int a = 3;  // val is set at compile time, can't change
                         // but when does the object's lifetime start?

static const int b = a+2; // fine, if a's ctor has already run

int array[b];             // OK, b is a compile-time constant

static const int d = foo(..); // error - must initialized to 
                              // compile-time value

switch(val) {
  case 1 : ...
  case a : ... // allowed!
  case b : ... // also OK
}
\end{verbatim}
}
\item \Emph{However...}
\begin{itemize}
  \item The issue we're skirting here is the ``static initialization order fiasco''
  \item if \texttt{a} and \texttt{b} are defined in different
    translation units, the relative order of instatiaion is \Emph{undefined}
  \item This is the problem that Meyers-style singletons are intended
    to solve
\end{itemize}
\end{itemize}
\end{frame}


%%%%%%%%%%%%%%%%%%%%%%%%%%%%%%%%%%%%%%%%%%%%%%%%%%%%%%%%%%%%%%%%%%%%%%
\begin{frame}[fragile,t]
\frametitle{static consts}
\framesubtitle{more problems with}
{\scriptsize\begin{verbatim}
// in header file:
struct IHaveConstants {
  static const int i = 42;       // fine, allowed
  static const double d = 3.14   // error, can't initialize in the header!
  static const double pi;        // ... so we have to init later
  static const char* name;       // ditto
};

// in some .cpp file somewhere...
const double IHaveConstants::pi = 3.14; // initialize elsewhere
const char* IHaveConstants::name = "Camel";
\end{verbatim}
}
\begin{itemize}[<+->]
\item For class statics, only integers can be initialized in the class declaration.
\item Order-of-initialization still an issue
\item For all static const variables:
  \begin{itemize}
  \item Note: it's a compile-time constant, but \Emph{lifetime starts
    during program startup}.
  \item So the compiler can assume it knows what the value of the
    variable is, although the object won't exist until program startup
  \item And notice: \emph{there is an object}, it's an lvalue, it
    has an address...
  \end{itemize}
\end{itemize}
\pause
\begin{center}\Emph{\cexpr to the rescue}\end{center}
\end{frame}

\subsection{Constexpr Values}
%%%%%%%%%%%%%%%%%%%%%%%%%%%%%%%%%%%%%%%%%%%%%%%%%%%%%%%%%%%%%%%%%%%%%%
\begin{frame}[fragile,t]
\frametitle{constexpr values}
\begin{itemize}[<+->]
\item declaring variables as \cexpr:
  \begin{itemize}
  \item Initialization at compile-time, not program startup
  \item Can be specified in .h file (rather than declare in .h and
    define in .cpp)
  \item Not restricted to integral types.
  \end{itemize}
\pause
{\scriptsize\begin{verbatim}
// in header file, or anywhere

constexpr int answer = 42;

struct IHaveConstants {
  static constexpr int i = answer;        // fine, allowed
  static constexpr double pi = 3.14       // Aha!

  constexpr const char* name = "Hi";      // oddly, needs the const
};

\end{verbatim}
}
\item Note: \texttt{IHaveConstants::i} does not have
  order-of-initialization issues relative to where \texttt{answer} is
  defined; it's all compile time.
\item And can be of user-defined type, as we'll see later.  
\end{itemize}

\end{frame}




\subsection{constexpr functions}
%%%%%%%%%%%%%%%%%%%%%%%%%%%%%%%%%%%%%%%%%%%%%%%%%%%%%%%%%%%%%%%%%%%%%%
\begin{frame}[fragile,t]
\frametitle{\cexpr functions}
\begin{itemize}[<+->]
\item \cexpr functions: a much nicer alternative to
  recursive templates.

\item Recall: templates weren't designed for this kind of thing
\item Templates are Turing-complete... kind of by accident.
\item ... so the syntax can get nasty.
\item \cexpr functions are Turing-complete by design
\item .... so the syntax is much, much nicer.

%% \item Recall from our earlier example: 

%% {\scriptsize\begin{verbatim}
%% template<int N> struct Factorial
%% {
%%    enum { result = N * Factorial<N-1>::result };
%% };

%% template<> struct Factorial<0> { enum {result = 1}; };

%% \end{verbatim}
%% }

%% \item Recursive templates to compute factorial at compile time.
%%   Compare to....

\end{itemize}
\end{frame}



%%%%%%%%%%%%%%%%%%%%%%%%%%%%%%%%%%%%%%%%%%%%%%%%%%%%%%%%%%%%%%%%%%%%%%
\begin{frame}[fragile,t]
\frametitle{Comparison of compile-time Factorials}
\begin{itemize}[<+->]
\item In \CC11, \cexpr functions are allowed to have only one
  statement:
\vskip 12pt
\begin{columns}[t]
\column{.5\textwidth}
\Emph{\CC98, Templates}
{\scriptsize\begin{verbatim}
template<int N>
struct Factorial {
enum { val = N * Factorial<N-1>::val };
};

template<> struct Factorial<0> 
{  enum {val = 1};}
\end{verbatim}
}

\column{.5\textwidth}
\Emph{\CC11, Constexpr}
{\scriptsize\begin{verbatim}
constexpr int Factorial(N)
{
  return (N >= 0) 
         ? N*Factorial(N-1)
         : 1;
}
\end{verbatim}
}
\end{columns}
\vskip 6pt
\item \Emph{\CC14} removes most limitations on \cexpr functions:
{\scriptsize\begin{verbatim}
constexpr int factorial(N)
{
   int res = 1;
   for (int i = N; i > 0; --i)
     res *= i;

   return res;
}
\end{verbatim}
} 

\end{itemize}
\pause
\begin{center}
These days, if you have \CC11 you probably have \CC14,
\Emph{so we'll talk \CC14 \cexpr from here on out.}
\end{center}

\end{frame}

%%%%%%%%%%%%%%%%%%%%%%%%%%%%%%%%%%%%%%%%%%%%%%%%%%%%%%%%%%%%%%%%%%%%%%
\begin{frame}[fragile,t]
\frametitle{Is it compile time or not?}
\begin{itemize}[<+->]
\item \cexpr function can be called at runtime just like
  normal functions
\item In order to run at compile time, all inputs and internal
  function calls must also be \cexpr
{\scriptsize\begin{verbatim}
int i;
...
return Factorial(5); // compile-time
return Factorial(i); // run time
\end{verbatim}
}
\item Limits on \cexpr functions:
  \begin{itemize}
    \item All inputs must be literals
    \item All function calls must be \cexpr
    \item No virtual function calls, etc.
  \end{itemize}
\item Things you can't do in \CC14 \cexpr functions:
  \begin{itemize}
  \item asm, goto, labels, try-block, throw
    \item (labels in case statements are OK)
    \item local variables must be of literal type, non-static,
      non-thread-local, initialized, non-volatile, ...
  \end{itemize}
\end{itemize}
\end{frame}

%%%%%%%%%%%%%%%%%%%%%%%%%%%%%%%%%%%%%%%%%%%%%%%%%%%%%%%%%%%%%%%%%%%%%%
\begin{frame}[fragile,t]
\frametitle{constexpr functions in \CC14}
\pause
\begin{itemize}[<+->]

\item Nobody memorizes all of this; be reasonable and the compiler
  will gently correct you.
\item Think of it as a \CC\ interpreter that gets run at compile time
  to get the answer.
\item Limitiations basically to keep this interpreter simple.
\item \Emph{Note}: \cexpr functions can have any of these
  constructs, they just can't \Emph{execute} them.  
  \begin{itemize}
    \item If the execution path doesn't have a forbidden construct,
      it's just fine.
  \end{itemize}
\item This leads to interesting constructs (next slide)
\end{itemize}


\end{frame}



%%%%%%%%%%%%%%%%%%%%%%%%%%%%%%%%%%%%%%%%%%%%%%%%%%%%%%%%%%%%%%%%%%%%%%
\begin{frame}[fragile,t]
\frametitle{``Throw'' idiom}
\begin{itemize}[<+->]
\item Example: handling Factorial of negative integers.
{\scriptsize\begin{verbatim}
constexpr int factorial(N)
{
  if (N<0) throw "Range Error";
  int res = 1;
  for (int i = N; i > 0; --i)
    res *= i;

  return res;
}


factorial(5);    // fine, compile time
factorial(-1);   // compile time error -- 
                 // can't throw in constexpr function

factorial( foo() ); // run-time
                    // Throws if foo() returns negative
\end{verbatim}
}

\item \ [Note: GCC < 5.4 gets this wrong!]

\end{itemize}
\end{frame}


%%%%%%%%%%%%%%%%%%%%%%%%%%%%%%%%%%%%%%%%%%%%%%%%%%%%%%%%%%%%%%%%%%%%%%
\begin{frame}[fragile,t]
\frametitle{How far can it go?}
\begin{itemize}[<+->]
\item \cexpr is Turing Complete
\item This is a new dimension of compile-time programming (orthogonal
  to templates)
\item Jason Turner has compile-time random number generators \Emph{(!)}
\url{https://www.youtube.com/watch?v=rpn_5Mrrxf8}
\item \cexpr is still being explored
\item \cexpr is still being propagated through the standard library
\end{itemize}
\end{frame}



\subsection{Constexpr Objects}
%%%%%%%%%%%%%%%%%%%%%%%%%%%%%%%%%%%%%%%%%%%%%%%%%%%%%%%%%%%%%%%%%%%%%%
\begin{frame}[fragile,t]
\frametitle{Constexpr Objects}
\begin{itemize}[<+->]
\item To make a \cexpr object of a user defined type, make a 
  \cexpr constructor:
{\scriptsize\begin{Verbatim}
struct complex_num {
   constexpr complex_num(double r, double i) : re(r), im(i) {}

   constexpr double abs_squared() const { return re*re+im*im; }
  
   double re;
   double im;   
};

int main()
{
   constexpr complex_num c(3,4);
   return c.abs_squared();
}
\end{Verbatim}
}
\item (live demo \#2 : put this into the compiler explorer, note the
  extreme lack of assembly instructions.)
\item We have compile-time objects of arbitrary complexity.
\item Jason Turner has  compile-time maze generators and solvers \Emph{(!)}
\url{https://www.youtube.com/watch?v=3SXML1-Ty5U}
\item[] \ \CC\ Weekly with Jason Turner on YouTube is good stuff.
\end{itemize}
\end{frame}


%%%%%%%%%%%%%%%%%%%%%%%%%%%%%%%%%%%%%%%%%%%%%%%%%%%%%%%%%%%%%%%%%%%%%%
\begin{frame}[fragile,t]
\frametitle{Constexpr Summary}
\begin{itemize}[<+->]
\item What about \cexpr std::vectors, std::strings....
\item Coming -- they're going through the standard library right now
  and adding \cexpr where appropriate
\item This is a whole new dimension to the language
\begin{itemize}
  \item Orthogonal to templates, but not exclusive
  \item Template metaprogramming and \cexpr complement each other.
\end{itemize}
\vskip 6pt
\item Advice: use \cexpr whenever possible.
\end{itemize}
\end{frame}
