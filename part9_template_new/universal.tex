%% ... 





%%%%%%%%%%%%%%%%%%%%%%%%%%%%%%%%%%%%%%%%%%%%%%%%%%%%%%%%%%%%%%%%%%%%%%
\begin{frame}[fragile,t]
\frametitle{References}
The definitive talk: Scott Meyers at C++ And Beyond 2012:

\url{https://channel9.msdn.com/Shows/Going+Deep/Cpp-and-Beyond-2012-Scott-Meyers-Universal-References-in-Cpp11}

\begin{itemize}[<+->]
\item We'll watch a cut-down version of the talk, but you should go
  watch the whole thing.
\item For the record, this is roughly:
\begin{itemize}
\item 0 - 16:40
\item 17:13 - 20:30
\item 56:00 - 58:00
\item 59:30 - 1:03:30
\item 1:19:58 - 1:22:53
\end{itemize}

\end{itemize}







%% \begin{itemize}[<+->]
%% \item ``Universal Reference'' a term invented by Dr. Meyers, now in
%%   standard use.
%% \item Rvalue reference $\implies$ \texttt{type\&\&}
%% \item Therefore,  \texttt{type\&\&} $\implies$ rvalue reference, right?
%% \item No! \texttt{type\&\&} $\centernot\implies$ rvalue reference.
%% \item This is tremendously confusing.
%% \item This is also a lie.
%% \item But it's a really really really useful lie.
%% \item Come with me and we'll see how deep the rabbit hole goes.
%% \end{itemize}

\end{frame}

%% %%%%%%%%%%%%%%%%%%%%%%%%%%%%%%%%%%%%%%%%%%%%%%%%%%%%%%%%%%%%%%%%%%%%%%
%% \begin{frame}[fragile,t]
%% \frametitle{Recall Reference Binding Rules}
%% \newcommand{\ccheck}{{\textcolor{magenta}{\checkmark}}}

%% \center{

%% \begin{tabular}{cr|c|c|c|c|}
%% %\cline{3-6}
%% \multicolumn{2}{c}{} &  \multicolumn{4}{c}{\bf{Object Type}}\\ 
%% & \multicolumn{1}{c}{}& \multicolumn{1}{c}{ const Lvalue}
%% & \multicolumn{1}{c}{Lvalue} & \multicolumn{1}{c}{Rvalue}
%% & \multicolumn{1}{c}{const Rvalue} \\ \cline{3-6} 
%% \multirow{4}{*}{\begin{sideways}\bf{Ref Type}\end{sideways}} & const \& & \ccheck & \ccheck & \ccheck & \ccheck \\ \cline{3-6}
%%  & \& & & \ccheck & & \\ \cline{3-6}
%%  & \&\& & &  & \ccheck & \\ \cline{3-6}
%%  & const \&\& & & & & \ccheck \\ \cline{3-6}


%% \end{tabular}

%% \vskip 12pt

%% \begin{itemize}[<+->]
%% \item Everything  binds to \texttt{const \&}
%% \item All the rest of the binding rules are \emph{minimal}
%% \item \texttt{const \&\&} is effectively useless
%% \end{itemize}

%% }



%% \end{frame}



%% %%%%%%%%%%%%%%%%%%%%%%%%%%%%%%%%%%%%%%%%%%%%%%%%%%%%%%%%%%%%%%%%%%%%%%
%% \begin{frame}[fragile,t]
%% \frametitle{The double life of type\&\&}
%% \begin{itemize}[<+->]
%% \vskip 12pt
%% \item Shorthand for this talk:
%% \begin{itemize}
%%   \item LRref = Lvalue reference
%%   \item RRef = Rvalue reference
%%   \item URef = Universal reference
%% \end{itemize}
%% \end{itemize}


%% {\scriptsize\begin{verbatim}
%% void f( Widget&& param);   // rvalue reference

%% Widget&& var1 = Widget();  // rvalue reference

%% auto&&   var2 = Widget();  // NOT rvalue reference

%% template<typename T>
%% void f(std::vector<T>&& p); // rvalue reference

%% template<typename T>
%% void f(T&& p);              // not rvalue reference
%% \end{verbatim}
%% }
%% \pause
%% This is still a lie.
%% \end{frame}

%% %%%%%%%%%%%%%%%%%%%%%%%%%%%%%%%%%%%%%%%%%%%%%%%%%%%%%%%%%%%%%%%%%%%%%%
%% \begin{frame}[fragile,t]
%% \frametitle{The double life of type\&\&}
%% \framesubtitle{The plot thickens...}
%% In \texttt{type\&\&}, ``\&\&'' means either:
%% \begin{itemize}[<+->]
%% \item Rvalue reference
%%   \begin{itemize}
%%     \item Binds rvalues only
%%     \item Facilitates moves
%% \end{itemize}
%% \item \Emph{Universal reference}
%%   \begin{itemize}
%%     \item Rvalue reference \emph{or} lvalue reference
%%     \item \Emph{Syntactically} \texttt{type\&\&}, \Emph{semantically}
%%       \texttt{type\&\&} or \texttt{type\&}.
%%       \item Binds to lvalues \emph{and} rvalues, const, nonconst,
%%         \Emph{everything}
%%       \item May facilitate copies, may facilitate moves
%%   \end{itemize}
%% \end{itemize}
%% \end{frame}

%%%%%%%%%%%%%%%%%%%%%%%%%%%%%%%%%%%%%%%%%%%%%%%%%%%%%%%%%%%%%%%%%%%%%%
\begin{frame}[fragile,t]
\frametitle{Perfect Fowarding: The Problem}
\framesubtitle{part 1}
\begin{itemize}[<+->]
\item ``Rvalue References Explained'' by Thomas Becker, \url{http://thbecker.net/articles/rvalue_references/section_07.html}
\item Consider a factory function:
{\scriptsize\begin{verbatim}
template<typename T, typename A>
shated_ptr<T> Factory(A arg)
{
  return shared_ptr<T>{new T(arg);;
}
\end{verbatim}
}
\item Great, we just forward the argument to T's ctor.
\item But, we're always making a copy of Arg, which might be bad for
  many reasons.
\item OK, so take by reference:
{\scriptsize\begin{verbatim}
template<typename T, typename Arg> shated_ptr<T> Factory(Arg& arg) {...}
\end{verbatim}
}
\item ... and now it won't work for const\& or rvalue arguments.  OK,
  overloading is our friend:
{\scriptsize\begin{verbatim}
template<typename T, typename A> shared_ptr<T> Factory(      A&  arg) {...}
template<typename T, typename A> shared_ptr<T> Factory(const A&  arg) {...}
template<typename T, typename A> shared_ptr<T> Factory(      A&& arg) {...}
\end{verbatim}
}
\end{itemize}
\end{frame}

%%%%%%%%%%%%%%%%%%%%%%%%%%%%%%%%%%%%%%%%%%%%%%%%%%%%%%%%%%%%%%%%%%%%%%
\begin{frame}[fragile,t]
\frametitle{Perfect Fowarding: The Problem}
\framesubtitle{part 2}
\begin{itemize}[<+->]
\item But if we have two parameters, we have a combinatorial
  explosion: 
{\scriptsize\begin{verbatim}
template<typename T, typename A, typename B> shared_ptr<T> ...

... F(      A&  a, B& b ) {...}
... F(const A&  a, B& b ) {...}
... F(      A&& a, B& b ) {...}

... F(      A&  a, const B& b ) {...}
... F(const A&  a, const B& b ) {...}
... F(      A&& a, const B& b ) {...}

... F(      A&  a, B&& b ) {...}
... F(const A&  a, B&& b ) {...}
... F(      A&& a, B&& b ) {...}
\end{verbatim}
}
\end{itemize}
\pause
\Emph{Perfect fowarding to the rescue!}
\end{frame}

%%%%%%%%%%%%%%%%%%%%%%%%%%%%%%%%%%%%%%%%%%%%%%%%%%%%%%%%%%%%%%%%%%%%%%
\begin{frame}[fragile,t]
\frametitle{Perfect Fowarding}
\begin{itemize}[<+->]
\item One parameter
{\scriptsize\begin{verbatim}
template<typename T, typename A> shared_ptr<T> Factory(A&&  arg) {...}
\end{verbatim}
}
\item Two parameters
{\scriptsize\begin{verbatim}
template<typename T, typename A, typename B>
shared_ptr<T> Factory(A&& a, B&& b)          {...}

\end{verbatim}
}
\item arbitrary parameters
{\scriptsize\begin{verbatim}

template<typename T, typename ...As>
shared_ptr<T> Factory(As&&.... args) { ...}
\end{verbatim}
}

\vskip 12pt
\item So go back to our earlier examples of variatic templates, and
  put ``\&\&'' everywhere.
\end{itemize}
\end{frame}

