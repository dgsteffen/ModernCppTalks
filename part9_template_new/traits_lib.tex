
%%%%%%%%%%%%%%%%%%%%%%%%%%%%%%%%%%%%%%%%%%%%%%%%%%%%%%%%%%%%%%%%%%%%%%
\begin{frame}[fragile,t]
\frametitle{Type Traits Refresher}
Recall our ``Is Class'' trait...
{\scriptsize\begin{verbatim}

template<typename T>
class is_class
{
private:

   typedef char One;                // a type with size 1
   typedef struct {char a[2];} Two; // a type with size 2

   // Now for some template functions that will never get called, but have distinct return types.

   // This one takes a pointer-to-member of C, and returns a One.
   template<typename C> static One test(int C::*);   // #1

   // This one will match anything else...
   template<typename C> static Two test(...);        // #2

public:

   enum { value = (sizeof(test<T>(0)) == 1) };

};
\end{verbatim}
}

\pause
The \CC11 standard library defines this for us, and many many others...

\end{frame}



%%%%%%%%%%%%%%%%%%%%%%%%%%%%%%%%%%%%%%%%%%%%%%%%%%%%%%%%%%%%%%%%%%%%%%
\begin{frame}[fragile,t]
\frametitle{The type\_traits header}
{\ \CC11\ } standard library: type\_traits.h

\begin{columns}[t]

\column{.5\textwidth}
{\tiny
// 20.11.3, helper class:

template <class T, T v> struct integral\_constant;

typedef integral\_constant<bool, true> true\_type;

typedef integral\_constant<bool, false> false\_type;

// 20.11.4.1, primary type categories:

template <class T> struct is\_void;

template <class T> struct is\_integral;

template <class T> struct is\_floating\_point;

template <class T> struct is\_array;

template <class T> struct is\_pointer;

template <class T> struct is\_lvalue\_reference;

template <class T> struct is\_rvalue\_reference;

template <class T> struct is\_member\_object\_pointer;

template <class T> struct is\_member\_function\_pointer;

template <class T> struct is\_enum;

template <class T> struct is\_union;

template <class T> struct is\_class;

template <class T> struct is\_function;

// 20.11.4.2, composite type categories:

template <class T> struct is\_reference;

template <class T> struct is\_arithmetic;

template <class T> struct is\_fundamental;

template <class T> struct is\_object;

template <class T> struct is\_scalar;

template <class T> struct is\_compound;

template <class T> struct is\_member\_pointer;

// 20.11.4.3, type properties:

template <class T> struct is\_const;

template <class T> struct is\_volatile;

template <class T> struct is\_trivial;

template <class T> struct is\_trivially\_copyable;

template <class T> struct is\_standard\_layout;


}
\column{.5\textwidth}
{\tiny
template <class T> struct is\_pod;

template <class T> struct is\_literal\_type;

template <class T> struct is\_empty;

template <class T> struct is\_polymorphic;

template <class T> struct is\_abstract;

template <class T> struct is\_signed;

template <class T> struct is\_unsigned;

template <class T, class... Args> struct is\_constructible;

template <class T> struct is\_default\_constructible;

template <class T> struct is\_copy\_constructible;

template <class T> struct is\_move\_constructible;

template <class T, class U> struct is\_assignable;

template <class T> struct is\_copy\_assignable;

template <class T> struct is\_move\_assignable;

template <class T> struct is\_destructible;

template <class T, class... Args> struct is\_trivially\_constructible;

template <class T> struct is\_trivially\_default\_constructible;

template <class T> struct is\_trivially\_copy\_constructible;

template <class T> struct is\_trivially\_move\_constructible;

template <class T, class U> struct is\_trivially\_assignable;

template <class T> struct is\_trivially\_copy\_assignable;

template <class T> struct is\_trivially\_move\_assignable;

template <class T> struct is\_trivially\_destructible;

template <class T, class... Args> struct is\_nothrow\_constructible;

template <class T> struct is\_nothrow\_default\_constructible;

template <class T> struct is\_nothrow\_copy\_constructible;

template <class T> struct is\_nothrow\_move\_constructible;

template <class T, class U> struct is\_nothrow\_assignable;

template <class T> struct is\_nothrow\_copy\_assignable;

template <class T> struct is\_nothrow\_move\_assignable;

template <class T> struct is\_nothrow\_destructible;

template <class T> struct has\_virtual\_destructor;



}

\end{columns}


\end{frame}


%%%%%%%%%%%%%%%%%%%%%%%%%%%%%%%%%%%%%%%%%%%%%%%%%%%%%%%%%%%%%%%%%%%%%%
\begin{frame}[fragile,t]
\frametitle{Example: \texttt{is\_pointer}}
\begin{itemize}[<+->]
\item Go look at the \CC11 standard, section 20.11.2.
\item OK, what do you do with it?
{\scriptsize\begin{verbatim}
#include <type_traits>
using namespace std;

template<typename T> int foo(T t)
{
  if ( is_pointer<T>::value) 
    return 5;
  return 16;
}

int main() {
  	int * i;
  	return foo(*i);
}
\end{verbatim}
}


\item (this is live demo \#1; put this into Mat Godbolt's compiler
  explorer \url{https://godbolt.org/} and see what happens.)

\end{itemize}
\end{frame}

%%%%%%%%%%%%%%%%%%%%%%%%%%%%%%%%%%%%%%%%%%%%%%%%%%%%%%%%%%%%%%%%%%%%%%
\begin{frame}[fragile,t]
\frametitle{\texttt{std::enable\_if}}
\begin{itemize}[<+->]
\item A powerful trait for using SFINAE and other tricks:
\vskip 6pt
{\begin{verbatim}

template< bool B, class T = void > struct enable_if;
\end{verbatim}
}
\vskip 6pt
\item If B is true, std::enable\_if has a public member typedef \texttt{type}, equal to T; otherwise, there is no member typedef. 
\item Use SFINAE to specialize or constrain template
\item Common techniques:
  \begin{itemize}
    \item Use enable\_if to define the return type
    \item Use enable\_if to define an extra, defaulted template parameter
  \end{itemize}
\end{itemize}
\pause
\end{frame}


%%%%%%%%%%%%%%%%%%%%%%%%%%%%%%%%%%%%%%%%%%%%%%%%%%%%%%%%%%%%%%%%%%%%%%
\begin{frame}[fragile,t]
\frametitle{\texttt{std::enable\_if}}

\begin{columns}[t]

\column{.5\textwidth}
Use as the return type:
{\scriptsize\begin{verbatim}

// 1. the return type (bool) is only valid 
// if T is an integral type:

template <class T> 

typename enable_if<is_integral<T>::value,
                   bool>::type

is_odd (T i) {return bool(i%2);}

\end{verbatim}
}
\pause
\column{.5\textwidth}
Use as a default template arg:
{\scriptsize\begin{verbatim}

// 2. the second template argument is only 
// valid if T is an integral type:

template <class T, class NOTUSED = 
          typename enable_if
             <is_integral<T>::value>
           ::type>

bool is_even (T i) {return !bool(i%2);}
\end{verbatim}
}

\end{columns}
\vskip 12pt
\pause
Either way:
{\scriptsize\begin{verbatim}
int main() {

  short int i = 1;    // code does not compile if type of i is not integral

  std::cout << std::boolalpha;
  std::cout << "i is odd: " << is_odd(i) << std::endl;
  std::cout << "i is even: " << is_even(i) << std::endl;

  return 0;
}

\end{verbatim}
}


\end{frame}
