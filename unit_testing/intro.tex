\section{Introduction}

\begin{frame}[fragile]
\frametitle{References}
\framesubtitle{E.G., where Dave stole all this material from}

\begin{itemize}
\item CppCon 2015: T. Winters \& H. Wright ``All Your Tests are Terrible...''
  \url{https://www.youtube.com/watch?v=u5senBJUkPc}
  \Emph{The} talk to start from.  The examples are in \CC but the
  content applies to all languages.
\vskip 6pt
\item ``Test Smells and Fragrances Kevlin Henney''
  \url{https://youtu.be/3frRGQGiPuc} (bad audio
  quality, the talk is a little slow, but it's got good content)

%% (also 
%% Structure and Interpretation of Test Cases - Kevlin Henney
%% \url{https://www.youtube.com/watch?v=tWn8RA_DEic}


\item GOTO 2013: Roy Osherove ``JS Unit Testing Good Practices \& Horrible Mistakes''
\url{https://www.youtube.com/watch?v=iP0Vl-vU3XM}  (He has many talks,
I think this is the best)

\end{itemize}


\end{frame}

%%%%%%%%%%%%%%%%%%%%%%%%%%%%%%%%%%%%%%%%%%%%%%%%%%%%%%%%%%%%%%%%%%%%%%
%%%%%%%%%%%%%%%%%%%%%%%%%%%%%%%%%%%%%%%%%%%%%%%%%%%%%%%%%%%%%%%%%%%%%%

\begin{frame}[fragile]
\frametitle{Definitions}
\framesubtitle{}
``Unit Testing'' is testing at the ``smallest unit of
functionality''.  For software engineers, these 
``smallest units'' are
\begin{itemize}
\item functions (or sets of related functions)
\item member functions (or classes)
\end{itemize}

In both cases, we are testing at the individual ``chunk'' of
executable code, the function or member function level.
\vskip 6pt
The term can be used to mean different things in other areas of
engineering, but this definition is standard in software.



\end{frame}


\begin{frame}[fragile]
\frametitle{Frameworks}
\framesubtitle{}
A unit test framework is a library (or language facility) that
provides unit testing features. Some languages have a framework built
in (C\#, Python(?)), some have one standard framwork (Java uses
JUnit), \CC has many:
\begin{itemize}
\item Google Test (GTest), probably the most commonly used.
\item Boost.Test (part of the Boost library), also widely used.
\item Catch2 (``Catch''), newer ``more advanced'' framework, still
  young but rapidly gaining market share
\end{itemize}
\vskip 6pt
GTest is the standard framework in use at SciTec.  OCEANA is using
Catch2, SOFA will do a trade study.
\vskip 6pt
\begin{center}
\Emph{Unit testing frameworks make it easy to write unit tests; they reduce
the friction involved, provide consistent testing facilities, and
enforce consistent testing practices.}
\end{center}

\end{frame}


%%%%%%%%%%%%%%%%%%%%%%%%%%%%%%%%%%%%%%%%%%%%%%%%%%%%%%%%%%%%%%%%%%%%%%
%%%%%%%%%%%%%%%%%%%%%%%%%%%%%%%%%%%%%%%%%%%%%%%%%%%%%%%%%%%%%%%%%%%%%%

\begin{frame}[fragile]
\frametitle{Examples}
\framesubtitle{}
Different frameworks use different terminology, but usually all the
tests for a unit are a {\bf test suite}, and the individual tests in a
suite are {\bf test cases}, each of which has multiple assertions.

{\scriptsize\
\begin{verbatim}
int factorial(int n) { return n*factorial(n-1); } // <-- unit under test

// Test case for normal operations
TEST( FactorialTestSuite, NormalCases ) {  

  ASSERT_TRUE ( factorial(1) == 1 );  
  ASSERT_EQ   ( factorial(2) ,  2 );  
  ASSERT_EQ   ( factorial(8) , 40320);
}

// Test case for failure modes and edge cases
TEST( FactorialTestSuite, FailureModes ) {

  ASSERT_THROW ( factorial(-1) );
  ASSERT_EQ    ( factorial(78), std::numeric_limits<int>::max() );
}
\end{verbatim}}

A test suite might test a single complex function, or a collection of
related functions.

\end{frame}

%%%%%%%%%%%%%%%%%%%%%%%%%%%%%%%%%%%%%%%%%%%%%%%%%%%%%%%%%%%%%%%%%%%%%%
%%%%%%%%%%%%%%%%%%%%%%%%%%%%%%%%%%%%%%%%%%%%%%%%%%%%%%%%%%%%%%%%%%%%%%
\begin{frame}[fragile]
\frametitle{Unit testing classes}
Testing a class is more interesting (and we'll spend time on this later)
{\scriptsize\
\begin{verbatim}

// The class to test                       // Test suite

class Cup {
public:                                    TEST( CupTests, CupCtor) {
                                               Cup c;
  Cup(); // new cups are empty                 ASSERT_TRUE( c.isEmpty() );
                                            }
  bool fill();                            
                                           TEST( CupTests, CupFill) {
  bool drink();                                Cup c;
                                               ASSERT_TRUE( c.fill() );
  bool isEmpty();                              ASSERT_FALSE( c.isEmpty() );
                                            }
};
\end{verbatim}}
\vskip 6pt


Usually, a test suite corresponds to a single header file.  Usually,
this means one function or class, or a set of closely related
functions and classes.
\end{frame}

%%%%%%%%%%%%%%%%%%%%%%%%%%%%%%%%%%%%%%%%%%%%%%%%%%%%%%%%%%%%%%%%%%%%%%
%%%%%%%%%%%%%%%%%%%%%%%%%%%%%%%%%%%%%%%%%%%%%%%%%%%%%%%%%%%%%%%%%%%%%%
%% \begin{frame}[fragile]
%% \frametitle{General Comments}

%% General comments before diving in to the video:

%% \begin{itemi

%% % 5 properties

%% - easy to write
%% - easy to run
%% - easy to write
%% - unambigous output
%% - easy to write
%% - easy to figure out what went wrong
%% - easy to write


%% - test code like system code


%% 4:00 write tests
%%  - good tests are 100% better than bad tests
%%  - even bad tests are 10000% better than no tests


%% pillars of software engineering:

%% - code reviews
%% - testing (unit)
%% - vc & ci/cd (version the software, test all the time)

%% 9:00 ish (mocks)

%%   - note, mocks should be tested!  But that's a separate test suite

%% 9:17 readers of tests -- to ... correct by inspection, local information and data and reasoning
%% don't assume reader has/will go read other documentation.  This assumes the docs exist.
%% plus devs never read that stuff.  Make a narrative.

%% talk about fixtures 13:00
