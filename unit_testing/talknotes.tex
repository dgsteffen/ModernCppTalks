%%%%%%%%%%%%%%%%%%%%%%%%%%%%%%%%%%%%%%%%%%%%%%%%%%%%%%%%%%%%%%%%%%%%%%%%%%%%%%%%
%%%%%%%%%%%%%%%%%%%%%%%%%%%%%%%%%%%%%%%%%%%%%%%%%%%%%%%%%%%%%%%%%%%%%%%%%%%%%%%%
\section{Notes on ``All Your Tests are Terrible...''}
\begin{frame}[fragile,t]
\frametitle{Properties of good tests}

\begin{itemize}
\item \Emph{Unit tests must be easy to run (one button push).}
\item \Emph{Unit tests must return an unambiguous pass/fail result.}
\item Correctness : it tests what it's supposed to test (and
  nothing else)
\item Readability :
\begin{itemize}
  \item Tests don't have tests, so they \emph{must be correct by inspection}
  \item Should read like a novel: setup, action, conclusion, and a happy ending
\end{itemize}
\item Completeness : tests cover the whole API, and only the API
  \begin{itemize}
  \item including edge cases, \emph{particularly} including edge cases
  \item excluding stuff that isn't under test
  \end{itemize}
\item Demonstrability : best documentation of how to use the API.
\item Resilience : ``the notion that your tests should only fail if the
  thing they were actually intended to demonstrate became false, and
  should fail \emph{for no other reason}''
\item \Emph{Test code is just as important as system code}
\end{itemize}

"Note that we didn't call these rules, we did call them goals of good
tests and I don't know that you can necessarily always get all of
them.''
\end{frame}

%%%%%%%%%%%%%%%%%%%%%%%%%%%%%%%%%%%%%%%%%%%%%%%%%%%%%%%%%%%%%%%%%%%%%%%%%%%%%%%%
%%%%%%%%%%%%%%%%%%%%%%%%%%%%%%%%%%%%%%%%%%%%%%%%%%%%%%%%%%%%%%%%%%%%%%%%%%%%%%%%
\begin{frame}[fragile,t]
\frametitle{WRITE TESTS}
\framesubtitle{$\approx 4:00$}
\begin{itemize}
\item good tests are 100\% better than bad tests
\item even bad tests are 10000\% better than no tests
\end{itemize}

\end{frame}

%%%%%%%%%%%%%%%%%%%%%%%%%%%%%%%%%%%%%%%%%%%%%%%%%%%%%%%%%%%%%%%%%%%%%%%%%%%%%%%%
%%%%%%%%%%%%%%%%%%%%%%%%%%%%%%%%%%%%%%%%%%%%%%%%%%%%%%%%%%%%%%%%%%%%%%%%%%%%%%%%
\begin{frame}[fragile,t]
\frametitle{Mocks}
\framesubtitle{$\approx 7:55$} 
(from Wikipedia)
\begin{itemize}
\item {\bf Mock Objects} are simulated objects that mimic the behavior of real objects in controlled ways. 
\end{itemize}

Technically, a Mock is an object that simulates external components
that are changed by the operation in question.
\vskip 6pt
Things that simulate inputs are technically ``stubs'' or something,
but everyone calls them mocks anyway.
\vskip 6pt
Note: mocks need to be tested just like everything else

\end{frame}

%%%%%%%%%%%%%%%%%%%%%%%%%%%%%%%%%%%%%%%%%%%%%%%%%%%%%%%%%%%%%%%%%%%%%%%%%%%%%%%%
%%%%%%%%%%%%%%%%%%%%%%%%%%%%%%%%%%%%%%%%%%%%%%%%%%%%%%%%%%%%%%%%%%%%%%%%%%%%%%%%
\begin{frame}[fragile,t]
\frametitle{Locality}
\framesubtitle{$\approx 11:50$}
Readability: make it clear to the reader why the result is correct.
\vskip 6pt
This is an example of a more general code smell: \Emph{lack of locality}
\vskip 6pt
\begin{itemize}
\item {\bf Locality}: the ability to understand the code without
  looking elsewhere.
\end{itemize}
\vskip 6pt 
Don't assume the reader knows about the other code, has
read the other code, or is going to go look right now.

\vskip 6pt
\begin{center}
\Emph{Tests don't have tests, so correct-by-inspection is all we have
  to prove our tests are correct.}
\end{center}
\end{frame}

%13:00 test fixtures
%16:30 edge cases, 19 important

%%%%%%%%%%%%%%%%%%%%%%%%%%%%%%%%%%%%%%%%%%%%%%%%%%%%%%%%%%%%%%%%%%%%%%%%%%%%%%%%
%%%%%%%%%%%%%%%%%%%%%%%%%%%%%%%%%%%%%%%%%%%%%%%%%%%%%%%%%%%%%%%%%%%%%%%%%%%%%%%%
\begin{frame}[fragile,t]
\frametitle{Black box tests}
\framesubtitle{$\approx 21:00 - 33:00$}
Definitions:
\begin{itemize}
  \item {\bf Black-box testing:} tests use only the public
    interface of the test article.  (The test article is a ``black box'' you can't see
    inside of.)
  \item {\bf White-box testing:} tests access the internal state of
    the test article. (You can ``see inside the box'').  Also
    ``clear-box'', ``glass-box'', etc.
\end{itemize}
\vskip 6pt
Black box testing is much preferred over white-box, to the degree that
many people say ``Black-box is the only way to test'' and ``White-box
testing is a mistake''.

\vskip 6pt
That's overstating the case.  Black-box is better, but not always
doable. We'll see how to white-box test below,
and discuss the pros and cons.

\end{frame}

%%%%%%%%%%%%%%%%%%%%%%%%%%%%%%%%%%%%%%%%%%%%%%%%%%%%%%%%%%%%%%%%%%%%%%%%%%%%%%%%
%%%%%%%%%%%%%%%%%%%%%%%%%%%%%%%%%%%%%%%%%%%%%%%%%%%%%%%%%%%%%%%%%%%%%%%%%%%%%%%%
\begin{frame}[fragile,t]
\frametitle{Test Result Data}
\framesubtitle{$\approx 29:00$}
Hyrum and Titus make fun of the ``run your code to generate the output
that you verify your code against''.  But...
\begin{itemize}[<+->]
\item We did exactly this on OCEANA as a system test -- run the whole
  thing, get output, save it to a ``canon'' file, compare with that.
\item Note that this does not test that the system is correct, just
  that it hasn't changed.
\item This is \emph{still} really useful when refactoring or
  optimizing.
\item And.... \emph{it found a bug}.  The output didn't compare
  roughly 1/3 times.
\item We had nondeterministic behavior, and who knows how long it
  would have taken us to notice.
\item (Look up ``strict weak ordering'' in sorting predicates)
\end{itemize}
\pause{}
\Emph{Bad tests are still 10000\% better than no tests.}
\end{frame}



%%%%%%%%%%%%%%%%%%%%%%%%%%%%%%%%%%%%%%%%%%%%%%%%%%%%%%%%%%%%%%%%%%%%%%%%%%%%%%%%
%%%%%%%%%%%%%%%%%%%%%%%%%%%%%%%%%%%%%%%%%%%%%%%%%%%%%%%%%%%%%%%%%%%%%%%%%%%%%%%%
\begin{frame}[fragile,t]
\frametitle{Sustainability, and Properties of good tests}
\framesubtitle{$\approx 44:58 - 47:00$}


\begin{itemize}
\item Again: write tests, bad tests are better than no tests
\end{itemize}


\end{frame}

%%%%%%%%%%%%%%%%%%%%%%%%%%%%%%%%%%%%%%%%%%%%%%%%%%%%%%%%%%%%%%%%%%%%%%%%%%%%%%%%
%%%%%%%%%%%%%%%%%%%%%%%%%%%%%%%%%%%%%%%%%%%%%%%%%%%%%%%%%%%%%%%%%%%%%%%%%%%%%%%%
\begin{frame}[fragile,t]
\frametitle{Culture}
\framesubtitle{$\approx 57:25$}
\begin{itemize}
\item Tests are only part of the solution (Defence in depth)
\begin{itemize}
\item code review
\item proper testing
\item static analysis
\item dynamic analysis
\item hardening code
\item canarying
\item production monitoring
\end{itemize}

\item Culture
\end{itemize}


\end{frame}

