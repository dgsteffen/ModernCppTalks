\section{Engineering and Management Concerns}
%%%%%%%%%%%%%%%%%%%%%%%%%%%%%%%%%%%%%%%%%%%%%%%%%%%%%%%%%%%%%%%%%%%%%%%%%%%%%%%%
%%%%%%%%%%%%%%%%%%%%%%%%%%%%%%%%%%%%%%%%%%%%%%%%%%%%%%%%%%%%%%%%%%%%%%%%%%%%%%%%
\begin{frame}[fragile,t]
\frametitle{Software Engineering Considerations}
Code is hard or impossible to unit test \emph{unless designed with
  this in mind}.
\begin{itemize}
\item Testability is just as important as any other design
  consideration
\item Experience, across the industry and personally, shows that
  designing for testability \emph{improves design}.  (Examples to
  follow).
\item Experience across the industry shows that writing the test
  \emph{before} writing the code improves design.  This is the heart
  of TDD (Test-Driven Development).
\item Black-box testing is always preferred over white-box testing.
\item \Emph{Test code must be treated as system code}.  It's not a
  second-class citizen.
\begin{itemize}
  \item Test code must be as clean, elegant, and maintainable as
    system code.
  \item Test harness, frameworks, and tooling must be designed as
    carefully as anything else
  \item Test code must be code-reviewed just as intently, and held to
    the same high standards, as system code.
  \item \emph{Test code is what lets you sleep soundly at night.
      Treat it accordingly.}
\end{itemize}
\end{itemize}

\end{frame}


%%%%%%%%%%%%%%%%%%%%%%%%%%%%%%%%%%%%%%%%%%%%%%%%%%%%%%%%%%%%%%%%%%%%%%%%%%%%%%%%
%%%%%%%%%%%%%%%%%%%%%%%%%%%%%%%%%%%%%%%%%%%%%%%%%%%%%%%%%%%%%%%%%%%%%%%%%%%%%%%%
\begin{frame}[fragile,t]
\frametitle{Management Considerations}
\begin{itemize}
  \item Testing takes time.
  \item Writing unit tests for exisiting code that
    probably already works is \Emph{incorrectly} considered to be a
    zero-return-on-investment activity.
    \begin{itemize}
      \item ROI might be small but is not
        zero. Consider that without unit tests....
      \item You can't fix bugs in that old code \emph{reliably} 
      \item You can't refactor that old code \emph{reliably}
      \item You can't prove that the code continues to work across
        changes to the code base (or compiler, or deliver OS, or...)
        \item You can not \Emph{provably} maintain your code base without them.
    \end{itemize}
  \item Without unit tests, your code base \emph{is not maintainable}
    without constant activity from superheroes.
\end{itemize}
\end{frame}
%%%%%%%%%%%%%%%%%%%%%%%%%%%%%%%%%%%%%%%%%%%%%%%%%%%%%%%%%%%%%%%%%%%%%%%%%%%%%%%%
%%%%%%%%%%%%%%%%%%%%%%%%%%%%%%%%%%%%%%%%%%%%%%%%%%%%%%%%%%%%%%%%%%%%%%%%%%%%%%%%

\begin{frame}[fragile,t]
\frametitle{Management Considerations}
\framesubtitle{part 2: religion}


Problem: testing takes time, money, schedule, etc. but
prevents problems later.  How do you measure the time / schedule slip
/ budget overrun \emph{that never happens}?  
\vskip 6pt
You can't.
\vskip 6pt

So is this a kind of a religion?  (No, because we have vast amounts of
evidence.)

\begin{itemize}

\item Testing takes a lot of time, in predictable amounts, at predictable
and controllable times.

\item Tracking down mysterious bugs takes unknowable amounts of time,
  at unpredictable moments, usually right before a release.

\end{itemize}

Balancing testing vs budget vs schedule is always hard.  Hold the line.

\end{frame}


%%%%%%%%%%%%%%%%%%%%%%%%%%%%%%%%%%%%%%%%%%%%%%%%%%%%%%%%%%%%%%%%%%%%%%%%%%%%%%%%
%%%%%%%%%%%%%%%%%%%%%%%%%%%%%%%%%%%%%%%%%%%%%%%%%%%%%%%%%%%%%%%%%%%%%%%%%%%%%%%%
%% \begin{frame}[fragile,t]
%% \frametitle{Software Engineering Considerations}
%% \end{frame}
