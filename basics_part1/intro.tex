\section{Intro: \CC11}

\lyxframeend{}\subsection[\CC 11]{{\CC 11} high points}


\begin{frame}[fragile,t]
\frametitle{\CC11 : Major Changes}

\begin{itemize}

  \item \textquotedblleft{}\CC11 feels like a new language.\textquotedblright{} --- Bjarne Stroustrup
    \pause{}

  \item \textasciitilde{}10 features pervasively change \CC\ coding
    syle, idioms, and guidance.
    \pause{}
    {\scriptsize
    \begin{enumerate}
    \item \Emph{auto}
    \item \Emph{initializer lists}
    \item \Emph{range-based for loop}
    \item \Emph{nullptr}
    \item smart pointers
    \item lambdas
    \item move semantics
    \item Class support / extensions
    \item atomic operations / threading
    \item enum class
    \end{enumerate}
}
\end{itemize}

%}
%\end{columns}
\end{frame}

\lyxframeend{}


\begin{frame}[fragile]
\frametitle{\CC11 References}
\framesubtitle{E.G., where Dave stole all this material from}
\begin{itemize}
  \item New books to update style, idioms and guidance:
    \begin{itemize}
    \item \noun{What:} T\CC~PL 4th Ed (Stroustrup) now available

    \item \noun{Why and How}: Style (Meyers) -- ``Effective Modern
      \CC'',  2014.

    \item \noun{Thou Shalt}: \CC\ Coding Standards (Sutter \&
    Alexandrescu, 2005) updated by an extensive website \url{https://isocpp.org/faq}.

    \end{itemize}
\end{itemize}
\end{frame}
%% \lyxframeend{}




%% \begin{frame}[fragile,t]
%% \frametitle{\CC\ Standards through the ages}
%% \begin{description}
%% \item[1985]: First Commerical Release, T\CC~PL, 1st Ed
%% \item[1991]: T\CC~PL, 2nd Ed, added templates and exceptions
%% \item[1997]: T\CC~PL, 3rd Ed, introduced ISO standard, added STL
%% \item[1998]: ISO \CC Standard (``\CC 98'')
%% \item[2003]: \CC 03 (``bug fix'' ISO Standard)\par
%% \item[2007]: TR1 (library additions): shared\_ptr, hash tables
%% \item[2011]: \CC 11, major revisions (what we're talking about today)
%% \item[2013]: Complete \CC11 implementations (GCC 4.8, CLANG 3.2)
%% \item[2014]: \CC 14   ``Bug fix'' to \CC11
%% \item[2017]: \CC17 New stuff; complete implementation in GCC 7
%% \item[2020]: \CC20 (?) In progress
%% \end{description}
%% \end{frame}


%% \lyxframeend{}%\lyxframe{\CC11 : Major Changes}








%% \begin{frame}[fragile]
%% \frametitle{\CC11 References}
%% \framesubtitle{E.G., where Dave stole all this material from}

%% \begin{itemize}
%% \item {\bf The} \CC\ Web Page: {\url{http://isocpp.org}}.  Has links to the actual standard document, blogs, etc.
%%       \begin{itemize}
%%       \item The FAQ at {\footnotesize \url{https://isocpp.org/faq}} has a section devoted to \CC11 features.
%%       \end{itemize}
%% \item GoingNative12 and 13: a bunch of \emph{really} good
%%       talks. {\footnotesize \url{http://herbsutter.com/2012/02/08/going-native-sessions-online/}} and
%%       {\footnotesize \url{http://channel9.msdn.com/Events/GoingNative/2013}}
%% \item CppCon (took over for GoingNative in 2014) talks on YouTube.
%% \item Stroustrup's home page and FAQ. {\footnotesize \url{http://www2.research.att.com/~bs}}
%% \item Herb Sutter's Blog and GOTW {\footnotesize \url{http://herbsutter.com/}}
%% \item C++ Next, particularly Dave Abrahams' discusson of move
%% semantics: {\footnotesize \url{http://web.archive.org/web/20140113221447/http://cpp-next.com/archive/2009/08/want-speed-pass-by-value/}}


%% \item GCC \CC11/14/17 support: {\footnotesize \url{https://gcc.gnu.org/projects/cxx-status.html}}


%% \end{itemize}


%% \end{frame}



%% \begin{frame}[fragile]
%% \frametitle{The rest of this talk}
%% \framesubtitle{``Hitting the high points''}

%% \begin{itemize}

%%         \item auto
%%         \item uniform initialization / initializer lists
%%         \item range-based for loops
%%         \item nullptr
%% %        \item Class extensions
%% %        \item lambdas (and sermon on standard containers and algorithms)
%% %        \item move semantics
%% %        \item smart pointers (and sermon on RAII, SESE, etc)
%% \end{itemize}

%% \end{frame}
