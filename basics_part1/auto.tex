
\subsection{auto}
\lyxframeend{}

%\lyxframe{auto}

%%%%%%%%%%%%%%%%%%%%%%%%%%%%%%%%%%%%%%%%%%%%%%%%%%%%%%%%%%%%%%%%%%%%%%
%%%%%%%%%%%%%% auto 1
\begin{frame}[fragile]
\frametitle{auto: the compiler knows the type}

\begin{columns}[t]
\column{.5\textwidth}
{\scriptsize
\begin{itemize}
\item  <1->\begin{verbatim} map<int, string>::iterator i   \end{verbatim}
\item[]<1->\begin{verbatim}      = m.begin()\end{verbatim}
\item<1->\begin{verbatim} const Vector d = foo() \end{verbatim}
\item<1->\begin{verbatim} singleton& s = singleton::instance() \end{verbatim}
\item  <1->\begin{verbatim} binder2nd<greater> f   \end{verbatim}
\item[]<1->\begin{verbatim}             = bind2nd(greater(),42) \end{verbatim}
\end{itemize}
}
\column{.5\textwidth}
{\scriptsize
\begin{itemize}
\item<2->\begin{verbatim} auto i = m.begin() \end{verbatim}
\item[]<2->\begin{verbatim}  \end{verbatim}
\item<2->\begin{verbatim} auto const d = foo() \end{verbatim}
\item<2->\begin{verbatim} auto& s = singleton::instance() \end{verbatim}
\item<2->\begin{verbatim} auto x = bind2nd(greater(),42) \end{verbatim}
\item<3->\begin{verbatim} auto f = [](int i){return i > 42;} \end{verbatim}
\end{itemize}
}
\end{columns}
\vskip 24pt
\uncover<4->{ \hfill \textcolor{purple} {auto: just do it... probably} \hfill}

\end{frame}

%%%%%%%%%%%%%%%%%%%%%%%%%%%%%%%%%%%%%%%%%%%%%%%%%%%%%%%%%%%%%%%%%%%%%%
%%%%%%%%%%%%%% auto 2
\begin{frame}[fragile]
\frametitle{When \emph{must} you use auto?}

\begin{itemize} %1
\item Unutterable types:
  \begin{itemize} %2
    \item Lambdas: what is the type of {\scriptsize\texttt{  [](int i)\{return i > 42;\}}} ?
    \item \emph{Why would you do that?}
\begin{itemize}
\item To reuse a lambda (for example, multiple containers with
  consistent ordering):
{\scriptsize
\begin{verbatim}
auto my_order = [](int i, int j){return i > j;}
set<int,       my_order> set_ordered_my_way;
map<int, char, my_order> map_ordered_the_same_way
\end{verbatim}}
\item Increasing momentum towards a consistent function syntax:
{\scriptsize \begin{verbatim}
int process(int arg);                 // C++98

auto process(int arg) -> int;         // C++11 trailing return type

auto process = [](int arg)->int {...} // lambda
\end{verbatim}}

\end{itemize}
  \end{itemize} %2
\vskip 12pt
\item Implementation-defined types: there are a few here and there.
\begin{itemize}
  \item \texttt{size\_t} is implementation defined, but consistently
    named
  \item Occasionally an interface will have such a type that has no
    consistent name; \texttt{auto} is the only way to portably declare one
\end{itemize}
\end{itemize} %1

\end{frame}

%%%%%%%%%%%%%%%%%%%%%%%%%%%%%%%%%%%%%%%%%%%%%%%%%%%%%%%%%%%%%%%%%%%%%%
%%%%%%%%%%%%%%%%%%%%%%%%%%%%%%%%%%%%%%%%%%%%%%%%%%%%%%%%%%%%%%%%%%%%%%
%%%%%%%%%%%%%%%%%%%%%%%%%%%%%%%%%%%%%%%%%%%%%%%%%%%%%%%%%%%%%%%%%%%%%%

\begin{frame}[fragile]
\frametitle{When \emph{can't} you use auto?}

\begin{itemize} %1
\item When you want an explicit type conversion:
{\scriptsize
\begin{verbatim}
int foo();
long i = foo();
\end{verbatim}
}
\vskip 6pt
%  \end{itemize} %2

\item When you want polymorphism:
{\scriptsize
\begin{verbatim}
class Base {};
class Derived : public Base {};
Derived* derivedFactory();
Base* my_ptr = derivedFactory();
\end{verbatim} }
\vskip 6pt
\item When you don't want an ephemeral / proxy type.
{\scriptsize
\begin{verbatim}
std::vector<bool> my_vector_of_bool;
auto m = my_vector_of_bool[5];         // not a bool!

Eigen::Matrix<...> A, B, C;
auto m = A*B+C                        // not a Matrix!

auto x = { 42 };                      // std::initializer_list

\end{verbatim}
}
\vskip 6pt

%\end{itemize} %2

\end{itemize} %1
\end{frame}



%%%%%%%%%%%%%%%%%%%%%%%%%%%%%%%%%%%%%%%%%%%%%%%%%%%%%%%%%%%%%%%%%%%%%%
%%%%%%%%%%%%%%%%%%%%%%%%%%%%%%%%%%%%%%%%%%%%%%%%%%%%%%%%%%%%%%%%%%%%%%
%%%%%%%%%%%%%%%%%%%%%%%%%%%%%%%%%%%%%%%%%%%%%%%%%%%%%%%%%%%%%%%%%%%%%%

\begin{frame}[fragile]
\frametitle{When \emph{can't} you use auto?}
\frametitle{... ahh, but you can!}
All the previous examples can be fixed with explicit types:
\begin{itemize}
\item When you want an explicit type conversion:
{\scriptsize
\begin{verbatim}
int foo();
auto i = long(foo());    // rather than long i = foo();                            
\end{verbatim}
}
\vskip 6pt
%  \end{itemize} %2

\item When you want polymorphism:
{\scriptsize
\begin{verbatim}
class Base {};
class Derived : public Base {};
Derived* derivedFactory();

auto my_ptr = Base*(derivedFactory()); 

\end{verbatim} }
\vskip 6pt
\item When you don't want an ephemeral / proxy type.
{\scriptsize
\begin{verbatim}
std::vector<bool> my_vector_of_bool;

auto m = bool(my_vector_of_bool[5]);  // it's a bool

Eigen::Matrix<...> A, B, C;
auto m = Eigen::Matrix<...>(A*B+C);   // it's a Matrix

auto x = 42 ;                         // int (auto and brace
                                      // initializers interact)

\end{verbatim}
}
\vskip 6pt

%\end{itemize} %2

\end{itemize} %1
\end{frame}


%%%%%%%%%%%%%%%%%%%%%%%%%%%%%%%%%%%%%%%%%%%%%%%%%%%%%%%%%%%%%%%%%%%%%%
%%%%%%%%%%%%%%%%%%%%%%%%%%%%%%%%%%%%%%%%%%%%%%%%%%%%%%%%%%%%%%%%%%%%%%
%%%%%%%%%%%%%%%%%%%%%%%%%%%%%%%%%%%%%%%%%%%%%%%%%%%%%%%%%%%%%%%%%%%%%%

\begin{frame}[fragile]
\frametitle{Auto Style}

\begin{itemize}[<+->]
\item Some would say the previous slide is silly; just don't use auto.
\item Consider Sutter's talk, at about the 28 minute mark.  His point
\item {\bf Deduce to track} if you don't need to commit to a type:
{\scriptsize \begin{verbatim}
const char* s = "Hello";            auto s = "Hello";
Widget w   = get_widget();          auto w = get_widget();   
\end{verbatim} }
\item {\bf Commit to stick} to a specific type:
{\scriptsize \begin{verbatim}
employee e { empid  };           auto e = employee { empid  };
widget   w { 12, 34 };           auto w = widget   { 12, 34 };
\end{verbatim} }
\end{itemize} %1

\vskip 6pt
``But why?'' you ask.  Why not use the old style?

\vskip 6pt
See Sutter...

\end{frame}
%%%%%%%%%%%%%%%%%%%%%%%%%%%%%%%%%%%%%%%%%%%%%%%%%%%%%%%%%%%%%%%%%%%%%%
%%%%%%%%%%%%%%%%%%%%%%%%%%%%%%%%%%%%%%%%%%%%%%%%%%%%%%%%%%%%%%%%%%%%%%
%%%%%%%%%%%%%%%%%%%%%%%%%%%%%%%%%%%%%%%%%%%%%%%%%%%%%%%%%%%%%%%%%%%%%%

\begin{frame}[fragile]
\frametitle{Auto Style part 2}
\frametitle{We're going left-to-right}
\begin{itemize}[<+->]
\item Consider consistency with broader shift in syntax and style in
  Modern C++:
{\scriptsize \begin{verbatim}
employee e { empid  };            auto e = employee{ empid  };
widget   w { 12, 34 };            auto w = widget  { 12, 34 };
\end{verbatim} }
\item Now consider literal suffixes
{\scriptsize \begin{verbatim}
int x;                            auto x = 42;     // must initialize!
float f = 42.0;                   auto f = 42.0f;  // no narrowing
unsigned long y = 42;             auto y = 42ul;   // literal suffix
std::string s= "Hi";              auto s = "Hi"s;  //
std::chrono::nanoseconds x{42);   auto x = 42ns;
\end{verbatim} }
\item Functions (with new syntax), lambdas, aliases:
{\scriptsize \begin{verbatim}
int f( double );                  auto f( double )   -> int; 
                                  auto f = [](double)-> int {...};

typedef set<string> dict;         using dict = set<string>;

template<class T> struct myvec    template<class T>
{ typedef vector<T*> type; }      using myvec = vector<T*>;
\end{verbatim} }
\end{itemize} %1
Are you seeing a pattern?
\end{frame}



%%%%%%%%%%%%%%%%%%%%%%%%%%%%%%%%%%%%%%%%%%%%%%%%%%%%%%%%%%%%%%%%%%%%%%
%%%%%%%%%%%%%% auto 4
\begin{frame}[fragile]
\frametitle{When \emph{should} you use auto?}

\center{ \textcolor{purple} {Everywhere you can.} }

(note: interface change robustness)
\pause

\center{ \textcolor{purple} {(unless it's trivial?)} }
\center{ \textcolor{purple} {(unless it's confusing?)} }

\pause

\begin{itemize}
\item
\begin{verbatim} auto pi = 3.14 ; // double
\end{verbatim} 
\pause{}
\item
\begin{verbatim} auto pi = 3.14L; // long double
\end{verbatim} 
\pause{}
\item
\begin{verbatim} auto pi = 3.14l; // long int!
\end{verbatim} 
\end{itemize}

\pause
There is still some debate.  Herb Sutter advocates ``AAA'' (Almost Always Auto)
\url{https://herbsutter.com/2013/08/12/gotw-94-solution-aaa-style-almost-always-auto/}, others
disagree, I think the jury is still out, and it may be a matter for local style guides.


\end{frame}



%%%%%%%%%%%%%%%%%%%%%%%%%%%%%%%%%%%%%%%%%%%%%%%%%%%%%%%%%%%%%%%%%%%%%%
\begin{frame}[fragile]
\frametitle{\texttt{decltype}: the other side of the coin}
\texttt{decltype(expr)} computes the type of \texttt{expr} at compile time; \texttt{expr} is
\emph{not} evaluated.
{\scriptsize
\begin{verbatim}
int i;         decltype(i)     // int
double j;      decltype(i+j)   // double
int&& foo();   decltype(foo()) // int&&
\end{verbatim}}
Mainly used for type computations in templates:
{\scriptsize
\begin{verbatim}
template<class C> struct Foo {  typedef decltype( C().begin() ) iter_t; };

typedef set<int> Set;

Foo<Set>::iter_t         // Set::iterator
Foo<const Set>::iter_t   // Set::const_iterator

\end{verbatim}}

\center{ \textcolor{purple} {Needed when you need a type for something
that is not a variable, like return types or typedefs.} }


\end{frame}


%%%%%%%%%%%%%%%%%%%%%%%%%%%%%%%%%%%%%%%%%%%%%%%%%%%%%%%%%%%%%%%%%%%%%%

%%%%%
\begin{frame}[fragile]
\frametitle{New ``suffix'' function declaration syntax (`auto' functions)}
\begin{itemize}
\item For library writers, probably not seen in everyday use:
{\scriptsize
\begin{verbatim}
int  foo (double d);        // normal function
auto foo (double d) -> int; // the same thing
\end{verbatim}}
\vskip 6pt
\item Solves a problem for template functions: what type to return?
{\scriptsize
\begin{verbatim}
template<class T, class U>     // The old way:
??? mult(T const& t, U const& u)
    { return t * u; }
\end{verbatim}}
\vskip 6pt

\item The compiler knows what type to return, but we can't express
  that with the old syntax.
{\scriptsize
\begin{verbatim}
template<class T, class U>     // The new way:
auto add(T const& t, U const& u) -> decltype(t+u)
    { return t + u; }
\end{verbatim}}
\vskip 6pt
(Why can't we put the decltype up front?  Because \texttt{t} and
\texttt{u} aren't in scope yet!)
\end{itemize}
\end{frame}


%%%%%
\begin{frame}[fragile]
\frametitle{`suffix'' function declaration syntax part 2}
\begin{itemize}
\item When \emph{must} you use it: template functions with template-dependent return types, as above.
\item When must you \emph{not} use it: no cases we are aware of
\item Guidance: the jury is still out
  \begin{itemize}
  \item Using it everywhere was originally seen as annoying/twee/elitist/showing off
  \item On the other hand, syntax is consistent with lambdas (and there is a general movement
    towards ``All functions are lambdas'' in the community, see \CC14 generic lambdas, etc.)
  \item Probably: be consistent with existing code.  For new code bases, look at \CC14 usage before deciding.
\end{itemize}
\end{itemize}

Also see \url{https://www.youtube.com/watch?v=ZCGyvPDM0YY} for more discussion, \url{https://google.github.io/styleguide/cppguide.html#auto} for Google's advice which is IMHO pretty good.

\end{frame}
