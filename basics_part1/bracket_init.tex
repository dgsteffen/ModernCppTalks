%%%%%%%%%%%%%%%%%%%%%%%%%%%%%%%%%%%%%%%%%%%%%%%%%%%%%%%%%%%%%%%%%%%%%%
%%%%%%%%%%%%%%%%%%%%%%%%%%%%%%%%%%%%%%%%%%%%%%%%%%%%%%%%%%%%%%%%%%%%%%
%%%%%%%%%%%%%%%%%%%%%%%%%%%%%%%%%%%%%%%%%%%%%%%%%%%%%%%%%%%%%%%%%%%%%%
\subsection{Uniform (\{\}) initialization} \lyxframeend{}

\begin{frame}[fragile]
\frametitle{Uniform initialization and initializer lists}

\begin{columns}[t]
\column{.5\textwidth}
\begin{itemize}
\item<1->{\scriptsize\begin{verbatim} complex<double> c (2,72, 3.14); \end{verbatim}}
\item<1->{\scriptsize\begin{verbatim} rectangle w (origin(), extents());\end{verbatim}}

\note{NOTE: rectangle, origin, extents all types!}

\item<1->{\scriptsize\begin{verbatim} int i[] = { 1, 2, 3, 4 };  \end{verbatim}}
\item<1->{\scriptsize\begin{verbatim} vector<int> v;  \end{verbatim}}
\item[]<1->{\scriptsize\begin{verbatim} for(int i=1; i<=4; ++i)   \end{verbatim}}
\item[]<1->{\scriptsize\begin{verbatim}   v.push(back(i)); \end{verbatim}}
\item<3->{\scriptsize\begin{verbatim} int i = 42;  \end{verbatim}}
\item<4->{\scriptsize\begin{verbatim} draw_rect(rectangle(orig,ext));  \end{verbatim}}
\end{itemize}

\column{.5\textwidth}
\begin{itemize}
\item<2->{\scriptsize\begin{verbatim} complex<double> c {2,72, 3.14}; \end{verbatim}}
\item<2->{\scriptsize\begin{verbatim} rectangle w {origin(), extents()};\end{verbatim}}
\item<2->{\scriptsize\begin{verbatim} int     i[] { 1, 2, 3, 4 }; \end{verbatim}}
\item<2->{\scriptsize\begin{verbatim} vector<int> v { 1, 2, 3, 4 };  \end{verbatim}}
\item[] <2->{\scriptsize\begin{verbatim} \end{verbatim}}
\item[] <2->{\scriptsize\begin{verbatim} \end{verbatim}}
\item<3->{\scriptsize\begin{verbatim} int i  {42}; // different but ok\end{verbatim}}
\item<4->{\scriptsize\begin{verbatim} draw_rect({orig,ext});  \end{verbatim}}
\item[]<4->{\scriptsize(which may or may not be good style, depending on context
  and conventions, but it's logically consistent.)}
\end{itemize}
\end{columns}
\pause{}
\vskip 12pt
Scott Meyers prefers the term``braced initialization'':
\begin{quotation}
``Uniform initialization'' is an idea. ``Braced initialization'' is a
  syntactic construct.
\end{quotation}
\end{frame}




%%%%%%%%%%%%%%%%%%%%%%%%%%%%%%%%%%%%%%%%%%%%%%%%%%%%%%%%%%%%%%%%%%%%%%
%%%%%%%%%%%%%%%%%%%%%%%%%%%%%%%%%%%%%%%%%%%%%%%%%%%%%%%%%%%%%%%%%%%%%%
%%%%%%%%%%%%%%%%%%%%%%%%%%%%%%%%%%%%%%%%%%%%%%%%%%%%%%%%%%%%%%%%%%%%%%

\begin{frame}[fragile,t]
\frametitle{Using '=' for initialization}
\begin{itemize}[<+->]
\item Initializing variables with '=' is confusing to beginners, since looks
like an assignment is happening.
{\scriptsize\begin{verbatim}
Widget w1;      // default constructor
Widget w2 = w1; // Copy constructor *not* assignment op

w1 = w2;  // assignment operator
\end{verbatim}}
\item In all but one case, using '=' with braced initialization is
  optional.  These are all exactly identical:
{\scriptsize\begin{verbatim}
int i {42};                int i = {42};
std::string s {"Hi"};      std::string s = {"Hi"};
Widget w {this, that};     Widget w = {this, that};
\end{verbatim}}
\item This includes types that are movable but not copyable
{\scriptsize\begin{verbatim}
std::atomic<int> i   {42}; // fine
std::atomic<int> i = {42}; // fine
std::atomic<int> i =  42;  // error (use of deleted function)
\end{verbatim}}
\item The only exception involves \texttt{auto} and initializer lists
  (see below).
\end{itemize}
\end{frame}

%%%%%%%%%%%%%%%%%%%%%%%%%%%%%%%%%%%%%%%%%%%%%%%%%%%%%%%%%%%%%%%%%%%%%%
%%%%%%%%%%%%%%%%%%%%%%%%%%%%%%%%%%%%%%%%%%%%%%%%%%%%%%%%%%%%%%%%%%%%%%
%%%%%%%%%%%%%%%%%%%%%%%%%%%%%%%%%%%%%%%%%%%%%%%%%%%%%%%%%%%%%%%%%%%%%%


\begin{frame}[fragile,t]
\frametitle{\CC's Most Vexing Parse\textsuperscript{TM} ... Solved!}
A longstanding issue that is so infamous it has it's own Wikipedia
entry:
{\scriptsize\begin{verbatim}
struct S {   S();           };
struct T {   T(const S& )  ;};

S s();       // #1 invoke the default ctor? No!
T t( S() );  // #2 invoke with a default-constructed S? No!
\end{verbatim}}

\begin{itemize}
\item \#1 declares a function named \texttt{s} returning an \texttt{S}
  and taking no arguments.
\item \#2 declares a function named \texttt{t} returning a \texttt{T}
  and taking \emph{a pointer to
  a function taking no arguments and returing an S}.  Really.
\item This, due to a rule from C, that states that any construct that
  might possibly be a function declaraion \emph{is} a function
  declaration.
\end{itemize}

\Emph{Uniform initialization syntax to the rescue!}
\begin{verbatim}
S s{};         // Unambiguously invokes the default ctor
S s  ;         // The same, of course
T t { S() };   // Unambiguously T::T(const S&)
\end{verbatim}

Henceforth, \CC's Most Vexing Parse\textsuperscript{TM} can be
relegated to the dustbin of history... and legacy code.
\end{frame}

%%%%%%%%%%%%%%%%%%%%%%%%%%%%%%%%%%%%%%%%%%%%%%%%%%%%%%%%%%%%%%%%%%%%%%
%%%%%%%%%%%%%%%%%%%%%%%%%%%%%%%%%%%%%%%%%%%%%%%%%%%%%%%%%%%%%%%%%%%%%%
%%%%%%%%%%%%%%%%%%%%%%%%%%%%%%%%%%%%%%%%%%%%%%%%%%%%%%%%%%%%%%%%%%%%%%

\begin{frame}[fragile,t]
\frametitle{No narrowing conversions}
\framesubtitle{additional type safety}
As an additional bonus, brace-initialization prohibits implicit
\emph{narrowing conversions} (assigning a value value to a type that isn't
guaranteed to be able to hold it).
{\scriptsize\begin{verbatim}
  int i = 42;   // fine, obviously
  int i = 42.0; // no compiler error
  int i {42.0}; // [x86-64 gcc 4.8.5 #1] warning: narrowing conversion
                //  of  '4.2e+1' from 'double' to 'int' inside { } 
                //  [-Wnarrowing]

  double x, y, x;
  
  int sum = x + y + z; // truncated to int!
  
  int sum = { x + y + z}; // warning as above.
\end{verbatim}}
\begin{center}
\Emph{This is reason enough to use braces in \emph{all}
  initializations, no matter how trivial}
\end{center}

\end{frame}
%%%%%%%%%%%%%%%%%%%%%%%%%%%%%%%%%%%%%%%%%%%%%%%%%%%%%%%%%%%%%%%%%%%%%%
%%%%%%%%%%%%%%%%%%%%%%%%%%%%%%%%%%%%%%%%%%%%%%%%%%%%%%%%%%%%%%%%%%%%%%
%%%%%%%%%%%%%%%%%%%%%%%%%%%%%%%%%%%%%%%%%%%%%%%%%%%%%%%%%%%%%%%%%%%%%%

\begin{frame}[fragile,t]
\frametitle{Initializer lists}
\framesubtitle{from en.cppreference.com}
\begin{itemize}[<+->]
\item An object of type \texttt{std::initializer\_list<T>} is a lightweight proxy object that provides access to an array of objects of type const T. 
\item Initializer lists provide ``old C-style array initialization  syntax'' for user defined types.
\item A std::initializer\_list object is automatically constructed when:
  \begin{itemize}
  \item a braced-init-list is used to list-initialize an object, where the corresponding constructor accepts an std::initializer\_list parameter
  \item a braced-init-list is used as the right operand of assignment or as a function call argument, and the corresponding assignment operator/function accepts an std::initializer\_list parameter
  \item a braced-init-list is bound to auto, including in a ranged for loop 
  \end{itemize}
\end{itemize}
\pause{}
{\scriptsize\begin{verbatim}
double d[] = { 1.2, 2.3, 3.4 };    // old C-style array initialization

std::vector<double> v = { 1.2, 2.3, 3.4 }; // now this works too

                        ^^^^^^^^^^^^^^^^^^ std::initializer_list<double>
\end{verbatim}}
\end{frame}
%%%%%%%%%%%%%%%%%%%%%%%%%%%%%%%%%%%%%%%%%%%%%%%%%%%%%%%%%%%%%%%%%%%%%%
%%%%%%%%%%%%%%%%%%%%%%%%%%%%%%%%%%%%%%%%%%%%%%%%%%%%%%%%%%%%%%%%%%%%%%
%%%%%%%%%%%%%%%%%%%%%%%%%%%%%%%%%%%%%%%%%%%%%%%%%%%%%%%%%%%%%%%%%%%%%%



\begin{frame}[fragile,t]
\frametitle{Uniform initialization and initializer lists}
\framesubtitle{Meyers, item 7}
\begin{itemize}[<+->]
\item Enabling initialization-list construction for your own types:
{\scriptsize\begin{verbatim}
namespace std {
template<typename T> class vector {  // sample (bad) implementation
...
vector(size_t n, const T& val); // #1 initialize with n vals

vector(std::initializer_list<T> vals); // #2 initialize from 
                                       // std::initializer_list
\end{verbatim}}
(For details, see Meyers item 7).
\item In this case, there is seeming ambiguity:
{\scriptsize\begin{verbatim}
vector<double> d {3, 4.5};   // calls #1, holds {4.5, 4.5, 4.5}
vector<double> d {3.5, 4.5}; // calls #2, holds {3.5, 4.5}

vector<size_t> i {3, 4}; // ambiguous? No! #2 is strongly favored
                         // i holds {3,4} not {4,4,4,4}

vector<size_t  i (3,4) ; // calls #1, holds {4, 4, 4, 4}
\end{verbatim}}
In overload resolution, initializer list calls win.  \Emph{this is one
  of the few times you need to resort to parentheses.}
\end{itemize}
\pause{}
(this is now considered to be a defect in the interface of std::vector)
\end{frame}

%%%%%%%%%%%%%%%%%%%%%%%%%%%%%%%%%%%%%%%%%%%%%%%%%%%%%%%%%%%%%%%%%%%%%%
%%%%%%%%%%%%%%%%%%%%%%%%%%%%%%%%%%%%%%%%%%%%%%%%%%%%%%%%%%%%%%%%%%%%%%
%%%%%%%%%%%%%%%%%%%%%%%%%%%%%%%%%%%%%%%%%%%%%%%%%%%%%%%%%%%%%%%%%%%%%%

\begin{frame}[fragile,t]
\frametitle{Default construction disambiguation}
As a special rule, an empty set of braces is always interpreted as a
call to the default constructor
\vskip 6pt
\vskip 6pt
To call a constructor with a zero-length initializer list, use an
extra set of braces.
{\scriptsize\begin{verbatim}
  std::vector<int> vec_default{} ; // default constructor

  std::vector<int> vec_empty {{}}; // empty initializer list

\end{verbatim}}
\end{frame}



%%%%%%%%%%%%%%%%%%%%%%%%%%%%%%%%%%%%%%%%%%%%%%%%%%%%%%%%%%%%%%%%%%%%%%
%%%%%%%%%%%%%%%%%%%%%%%%%%%%%%%%%%%%%%%%%%%%%%%%%%%%%%%%%%%%%%%%%%%%%%
%%%%%%%%%%%%%%%%%%%%%%%%%%%%%%%%%%%%%%%%%%%%%%%%%%%%%%%%%%%%%%%%%%%%%%

\begin{frame}[fragile,t]
\frametitle{Brace Initialization style}
Meyers admits that the case is somewhat ambiguous, you can't always
use one or the other.  I say use 'em.

\begin{itemize}
\item Avoid \CC's Most Vexing Parse\textsuperscript{TM}!
\item int i {42.0} warns; take advantage of the
  no-narrowing-conversion rules.
\item Not what many of us old-timers are used to, but consistency is
  easier to teach
\end{itemize}

\center{ \textcolor{purple} {Use \{\} unless you have reason not to: }}
\begin{itemize}
\item Ambiguous constructors (\texttt{std::vector<int>})
\item Interaction with \texttt{auto} (see below)
\end{itemize}



\end{frame}


%% \vskip 12pt
%% \begin{itemize}
%% \item Unless you want to use auto
%% \begin{verbatim}
%% int n;
%% auto w(n);    // int
%% auto x = n;   // int
%% auto y {n};   // std::initializer_list<int>
%% \end{verbatim}
%% \end{itemize}
%\end{frame}
