%%%%%%%%%%%%%%%%%%%%%%%%%%%%%%%%%%%%%%%%%%%%%%%%%%%%%%%%%%%%%%%%%%%%%%
\begin{frame}[fragile,t]
\frametitle{RAII}
\framesubtitle{Resource Aquisition Is Initialization}
\begin{itemize}
\item Smart pointers are a specific case of a more general idiom, ``RAII''
\item Developed to manage resources in the presence of exceptions, but
  they are much more widely useful.
\item The key idea:  \Emph{Tie resource management to object lifetime}
\item Resource manager objects receive the resource in their
  ctor, and release it in their dtor.  Emphasis:
\begin{itemize}
  \item Allocate resource in constructor (at initialization)
  \item Release resources in destructor (at scope exit)
\end{itemize}
\item Normal scoping rules now ensure proper cleanup.
\item Exceptions cause local variables to be destructed, so no leaks.
\end{itemize}
\vskip 12pt
\pause{}
\center{\Emph{ The most important operation in \CC: }}
\pause{}
\center{\Emph{\texttt{\}}}}

\end{frame}


%%%%%%%%%%%%%%%%%%%%%%%%%%%%%%%%%%%%%%%%%%%%%%%%%%%%%%%%%%%%%%%%%%%%%%
\begin{frame}[fragile,t]
\frametitle{Dynamic Memory Is Easy}
{\scriptsize\
\begin{verbatim}
OLD:                              NEW:

void foo() {                      void foo() {
  Widget* d = new Widget();          smart_ptr<Widget> d = new Widget();
  ...                                ....
  if (condition1)                    if (condition1) 
    return;       // Leak!             return;        // no leak!
  ...
  if (error)                         if (error)
    throw ex;                          throw ex;      // no leak!
  ...                                ...
  foo(); // throws, leak!            foo();           // no leak!
  ...             

  delete d;      // cleanup          // no delete, no leak!
}                                 } 
\end{verbatim}
}

(for some definition of ``smart\_ptr'')

The smart pointer object receives a dynamically allocated object in
its ctor, and releases it in the dtor, which executes at scope exit.
\end{frame}

%%%%%%%%%%%%%%%%%%%%%%%%%%%%%%%%%%%%%%%%%%%%%%%%%%%%%%%%%%%%%%%%%%%%%%
%%%%%%%%%%%%%%%%%%%%%%%%%%%%%%%%%%%%%%%%%%%%%%%%%%%%%%%%%%%%%%%%%%%%%%
\begin{frame}[fragile,t]
\frametitle{And the code gets cleaner too}

Remember our Single-Entry-Single-Exit example:

\begin{columns}[t]
\column{.5\textwidth}
Heavily nested if-blocks:
{\scriptsize\begin{verbatim}
void foo() {
  if (precondition1) {
    double* d = new double[70];
    if (d) {
      if (conditionA) {
        Widget* w = new Widget;
        if (conditionB) {
           // finally do some work
           ...
           FiddleWidget(w);
           ...
           }
        delete w;
        }
     delete d[];
     }
   }
  return;
}
\end{verbatim}}
\pause{}
\column{.5\textwidth}
Becomes this:
{\scriptsize\begin{verbatim}
void foo() {
  if (!precondition1)  return;
  std::vector<double> d (70); // #1
                              // #2
  if (!precondition2) return;
  smart_ptr<Widget> w = new Widget;
  if (conditionB) return;
  // finally do some work
  ...
  FiddleWidget(w);
  ...






  return;
}
\end{verbatim}}
\end{columns}
\pause{}
\Emph{Even in the absence of exceptions, exception-safe code is cleaner,
faster, and more maintainable.}
\end{frame}
