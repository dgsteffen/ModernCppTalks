%%%%%%%%%%%%%%%%%%%%%%%%%%%%%%%%%%%%%%%%%%%%%%%%%%%%%%%%%%%%%%%%%%%%%%
\section[Summary]{Summary}

\begin{frame}[fragile]
\frametitle {Summary: Lvalues and Rvalues}
\begin{itemize}[<+->]
\item Lvalues: objects with names (the variables we all know and love)
\item \texttt{\&}: ref to an lvalue (not rvalue!)
\item \texttt{const\&}: ref to anything (lvalue or rvalue)
\item Rvalue: unnamed temporary or other ephemeral thing. 
\item \texttt{\&\&}: an \emph{lvalue}, but binds \emph{only} to rvalues.
\item \texttt{const\&\&}: possible but useless.
\vskip 6pt
\item Functions that take \texttt{\&\&}s can steal resources from them.
\end{itemize}
\end{frame}


\begin{frame}[fragile]
\frametitle {Rules Of Thumb 1: Code For Copy Elision}
\begin{itemize}[<+->]
\item Pass By Value If You Want A Copy:
{\scriptsize
\begin{verbatim}
void foo ( const bar& b) {  // not this
  bar local = b;
  // ... modify local
}

void foo ( local b ) {        // but this
 // ... modify local 
}

\end{verbatim}
}
because the compiler can either A) elide the copy or B) move to it if
possible.

\vskip 12pt
\item Enable RVO When Possible:
  \begin{itemize}
    \item GCC supports URVO in general.
    \item GCC supports NRVO, but all return statements must return the
      same variable.
    \item Don't warp code to enable RVO unless you know it's really
      critical.
   \end{itemize}

\end{itemize}
\end{frame}

%%%%%%%%%%%%%%%%%%%%%%%%%%%%%%%%%%%%%%%%%%%%%%%%%%%%%%%%%%%%%%%%%%%%%%

\begin{frame}[fragile]
\frametitle {Rules Of Thumb 2: Coding Movable Types}
\begin{itemize}[<+->]
\item Declaring a movable type looks like this:
{\scriptsize
\begin{verbatim}
struct A {
  A (const A&  lvalue);    // copy ctor
  A (const A&& rvalue);    // move ctor

  A& operator=(const A&  lval);  // copy assignment
  A& operator=(      A&& rval);  // move assignment
};

void swap(A& a, A& b);   // customize using move semantics

\end{verbatim}
}

\vskip 6pt


\item swap?    The standard provides:
{\scriptsize
\begin{verbatim}
template<class T> void swap (T&a, T&b) { T c(a); a=b; b=c; }
\end{verbatim}
}
which gets used a lot in standard containers and algorithms.  Provide
an optimized version if it makes sense for your class.  You can fiddle
pointers, or use move semantics directly:
{\scriptsize
\begin{verbatim}
template<class T> void swap (T&a, T&b) { 
T c( move(a) ); a=move(b); b=move(c); }
\end{verbatim}
}

\end{itemize}
\end{frame}

%%%%%%%%%%%%%%%%%%%%%%%%%%%%%%%%%%%%%%%%%%%%%%%%%%%%%%%%%%%%%%%%%%%%%%



\begin{frame}[fragile]
\frametitle{Rules Of Thumb 3}
\begin{itemize}
\item Leave moved-from objects in a destructable and assignable, but
  not necessarily consistent, state.
\item Move operations preserve user-visible side effects on both LHS
and RHS. (E.G., LHS's old resources get released immediately.)
Consensus seems to be moving towards 
``Any dtor side effects should explicitly carried out in move.''

\item Tell the compiler you're done with an object with std::move().
\item Remember: move is an optimization of copy, so all of the above
  can be ignored without any consequence worse than missed
  optimization opportunities.
\end{itemize}

\end{frame}
