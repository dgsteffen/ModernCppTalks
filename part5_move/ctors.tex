
\subsection{Delegating and Inheriting Ctors}
\lyxframeend{}

%\lyxframe{auto} 

%%%%%%%%%%%%%%%%%%%%%%%%%%%%%%%%%%%%%%%%%%%%%%%%%%%%%%%%%%%%%%%%%%%%%%
%%%%%%%%%%%%%% auto 1
\begin{frame}[fragile]
\frametitle{Delegating Ctors}
(from Stroustrup's page):
\begin{columns}[t]
\column{.5\textwidth}
{\scriptsize
In C++98, if you want two constructors to do the same thing, repeat yourself or call "an init() function." For example:

\begin{verbatim}  
class X {
  int a;

  validate(int x) { ... }

 public:

  X(int x) { validate(x); }
  X()      { validate(42); }
  X(string s) { int x = lexical_cast<int>(s); validate(x); }
  // ...
};
\end{verbatim}
}

\column{.5\textwidth}
{\scriptsize

Verbosity nders readability and repetition is error-prone. Both get in
the way of maintainability. So, in C++0x, we can define one
constructor in terms of another:

\begin{verbatim} 
class X {
  int a;

 public:

  X(int x) { ...  }
  X() :X{42} { }
  X(string s) :X{} { }
  // ...
};
\end{verbatim}
}
\end{columns}

Gotchas: none known yet.

However, one possibly important bit:

The lifetime of an object begins when any constructor (i.e., in the
case of delegation, the ulti‐ mate target constructor) is successfully
completed. This means that an exception thrown from the body of a
delegating constructor will cause the destructor to be invoked
automatically.

Note that delegating to a base class' ctors has always been
supported.  Now we can delegate to our own.


\end{frame}

%%%%%%%%%%%%%%%%%%%%%%%%%%%%%%%%%%%%%%%%%%%%%%%%%%%%%%%%%%%%%%%%%%%%%%
%%%%%%%%%%%%%%
\begin{frame}[fragile]
\frametitle{Inheriting Ctors}

People sometimes are confused about the fact that ordinary scope rules apply to class members. In particular, a member of a base class is not in the same scope as a member of a derived class:


	struct B {
		void f(double);
	};

	struct D : B {
		void f(int);
	};

	B b;   b.f(4.5);	// fine
	D d;   d.f(4.5);	// surprise: calls f(int) with argument 4

In C++98, we can "lift" a set of overloaded functions from a base class into a derived class:


	struct B {
		void f(double);
	};

	struct D : B {
		using B::f;     // bring all f()s from B into scope
		void f(int);    // add a new f()
	};

	B b;   b.f(4.5);	// fine
	D d;   d.f(4.5);	// fine: calls D::f(double) which is B::f(double)

I have said that "Little more than a historical accident prevents using this to work for a constructor as well as for an ordinary member function." C++0x provides that facility:


	class Derived : public Base { 
	public: 
		using Base::f;    // lift Base's f into Derived's scope -- works in C++98
		void f(char);     // provide a new f 
		void f(int);      // prefer this f to Base::f(int) 

		using Base::Base; // lift Base constructors Derived's scope -- C++0x only
		Derived(char);    // provide a new constructor 
		Derived(int);     // prefer this constructor to Base::Base(int) 
		// ...
	}; 

If you so choose, you can still shoot yourself in the foot by inheriting constructors in a derived class in which you define new member variables needing initialization:


	struct B1 {
		B1(int) { }
	};

	struct D1 : B1 {
		using B1::B1; // implicitly declares D1(int)
		int x;
	};

	void test()
	{
		D1 d(6);	// Oops: d.x is not initialized
		D1 e;		// error: D1 has no default constructor
	}

You might remove the bullet from your foot by using a member-initializer:


    	struct D1 : B1 {
    		using B1::B1;	// implicitly declares D1(int)
    		int x{0};	// note: x is initialized
    	};

    	void test()
    	{
    		D1 d(6);	// d.x is zero
    	}



\end{frame}


