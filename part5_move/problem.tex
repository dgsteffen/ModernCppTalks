\lyxframeend{}\section[The Problem]{If Move Semantics are the
Answer, What's the Question?}

%\subsection[Problem == Copies]{Death, Taxes, and Copies}\lyxframeend{}
\subsection[Problem == Copies]{Death, Taxes, and...}\lyxframeend{}

%%%%%%%%%%%%%%%%%%%%%%%%%%%%%%%%%%%%%%%%%%%%%%%%%%%%%%%%%%%%%%%%%%%%%%
%%%%%%%%%%%%%% auto 1

\begin{frame}[fragile,t]
\frametitle{Death, Taxes, and Copies}
\framesubtitle{Three things you can't avoid}

\begin{itemize}[<+->]

\item{\scriptsize
\begin{verbatim}
vector<string> get_names() {... return retval;} // generator fn
vector<string> names = get_names();             // use generator
\end{verbatim}
}

\item In principle, two copies of the vector, and a default constructor.

\item If there are N strings in the vector, each copy could require as many
as N+1 memory allocations and a whole slew of cache-unfriendly data
accesses as the string contents are copied.


\vskip 12pt
\item Option 1: don't make a copy, don't ever change the result
{\scriptsize
\begin{verbatim}
vector<string> const& names = get_names();
\end{verbatim}
}

\vskip 12pt

\item Option 2: commit to smart pointers and ownership transfer
{\scriptsize
\begin{verbatim}
unique_ptr<vector<string>> get_names();
\end{verbatim}
}
\vskip 12pt

\item Option 3: FORTRAN-style ``out parameters''
{\scriptsize
\begin{verbatim}
void get_names( vector<string>& m_names);
\end{verbatim}
}


\end{itemize}

\end{frame}

%%%%%%%%%%%%%%%%%%%%%%%%%%%%%%%%%%%%%%%%%%%%%%%%%%%%%%%%%%%%%%%%%%%%%%

\begin{frame}[fragile,t]
\frametitle{Copies Are Inevitable}
{\scriptsize
\begin{verbatim}
class Vector {...};

Vector operator+ (Vector const& X, Vector const& Y) {
  Vector result(X);
  return result += Y;
}
...
Vector C =    A + B  + D  + Foo(E+F);
   --->  = (((A + B) + D) + Foo(E+F));
\end{verbatim}
}

\begin{itemize}[<+->]
\item Five subexpressions means five unnamed temporaries, means five
  copies, and then the assignment.
\item Options 2 and 3 are not an option unless we give up the syntax.
\item Option 1 doesn't help and may not be a viable option anyway.
\end{itemize}

\pause

{ \hfill \textcolor{magenta} {Copies are the bane of
    high-performance computing in C++.} \hfill}


\end{frame}

%%%%%%%%%%%%%%%%%%%%%%%%%%%%%%%%%%%%%%%%%%%%%%%%%%%%%%%%%%%%%%%%%%%%%%
\begin{frame}[fragile,t]
\frametitle{Examples In Real Code}

\begin{itemize}[<+->]

\item Common Data Store (global data, with mutex locks)
{\scriptsize
\begin{verbatim}
virtual dataType Get() const;
virtual void     Get(dataType *item) const;  // Faster version.
                                             // Better for large data items.
\end{verbatim}
}
``Old'' (C-style) out parameter.  (One copy is mandatory, it's a shared
resource, which is why we want to avoid any more.)
\textcolor{magenta}{Rough syntax to use, and you have to remember it exists!}

\vskip 12pt

\item Old-style GUI, translate from UTF8 to ASCII
{\scriptsize
\begin{verbatim}
void UTF8String::getAscii (string& destination) { ... }
\end{verbatim}
}
Classic \CC-style out parameter.

\vskip 12pt

\end{itemize}
\pause

{ \textcolor{magenta} {Not just clumsy, but can also disable
optimizations, as we shall soon see.}}


\end{frame}

%%%%%%%%%%%%%%%%%%%%%%%%%%%%%%%%%%%%%%%%%%%%%%%%%%%%%%%%%%%%%%%%%%%%%%

\subsection[Compiler != Solution]{Compilers Offer A Partial Solution, Maybe}\lyxframeend{}


\begin{frame}[fragile,t]
\frametitle{Compilers aren't that stupid}

{\textcolor{magenta}{Copy Elision \& Co. to the rescue...}}
%\uncover<5>{\textcolor{magenta}{... maybe. Compilers can skip copy operations \emph{only if doing so has no observable effects}.}}

\begin{itemize}
\item<2-> All modern compilers make these the same (mandated in  \CC11):
{\scriptsize
\begin{verbatim}
vector<string> names = get_names();
vector<string> names(get_names);
\end{verbatim}}

\vskip 6pt

\item<3->Return Value Optimization (RVO):{\scriptsize
\begin{verbatim}
vector<string> get_names() {
  string twoB, not2B;
  ...
  return twoB + " or " + not2B; } 

vector<string> my_names = get_names();
\end{verbatim}}
\vskip 6pt
\item<4->''Named'' Return Value Optimization (NRVO):{\scriptsize
\begin{verbatim}
vector<string> get_names() {
  vector<string> result; 
  ... 
  return result; } 

vector<string> my_names = get_names();
\end{verbatim}}

\vskip 6pt


\end{itemize}

%% \onslide<4>{
%% \textcolor{magenta}{... maybe. Compilers can skip copy operations \emph{only if doing so
%%     has no observable effects}.}
%% }

\end{frame}

\begin{frame}[fragile,t]
\frametitle{... but compilers are limited}

\textcolor{magenta}{Copy elision is fragile. %Compilers can skip copy
                                %operations \emph{only if doing so has
                                %no observable effects}
}


\begin{itemize}
\item<1-> Copy elision not allowed if these have side effects:
{\scriptsize
\begin{verbatim}
vector<string> names = get_names();
vector<string> names(get_names);
\end{verbatim}
}

\item<2-> NRVO / URVO happens, or not, on a compiler-by-compiler
basis.
\item<3-> URVO is common and fairly robust, but NRVO is easily disabled in compiler specific ways:
{\scriptsize
\begin{verbatim}
Foo makefoo() {
  Foo a, b;
  ...
  if (expr) return a;
  return b;
}
\end{verbatim}
}
(This disables NRVO in GCC and probably most other compilers.)
\end{itemize}

\onslide<4>{
\center{\textcolor{magenta}{Copy elision happens at the whim of the compiler, and
  all you can do is hope.}}}
\end{frame}

