
\subsection{auto}
\lyxframeend{}

%\lyxframe{auto} 

%%%%%%%%%%%%%%%%%%%%%%%%%%%%%%%%%%%%%%%%%%%%%%%%%%%%%%%%%%%%%%%%%%%%%%
%%%%%%%%%%%%%% auto 1
\begin{frame}[fragile]
\frametitle{auto: the compiler knows the type}

\begin{columns}[t]
\column{.5\textwidth}
{\scriptsize
\begin{itemize}
\item  <1->\begin{verbatim} map<int, string>::iterator i   \end{verbatim}
\item[]<1->\begin{verbatim}      = m.begin()\end{verbatim}
\item<1->\begin{verbatim} const Vector d = foo() \end{verbatim}
\item<1->\begin{verbatim} singleton& s = singleton::instance() \end{verbatim}
\item  <1->\begin{verbatim} binder2nd<greater> x   \end{verbatim}
\item[]<1->\begin{verbatim}             = bind2nd(greater(),42) \end{verbatim}
\end{itemize}
}
\column{.5\textwidth}
{\scriptsize
\begin{itemize}
\item<2->\begin{verbatim} auto i = m.begin() \end{verbatim}
\item[]<2->\begin{verbatim}  \end{verbatim}
\item<2->\begin{verbatim} auto const d = foo() \end{verbatim}
\item<2->\begin{verbatim} auto& s = singleton::instance() \end{verbatim}
\item<2->\begin{verbatim} auto x = bind2nd(greater(),42) \end{verbatim}
\item<3->\begin{verbatim} auto x = [](int i){return i > 42;} \end{verbatim}
\end{itemize}
}
\end{columns}
\vskip 24pt
\uncover<4->{ \hfill \textcolor{purple} {auto: just do it} \hfill}

\end{frame}

%%%%%%%%%%%%%%%%%%%%%%%%%%%%%%%%%%%%%%%%%%%%%%%%%%%%%%%%%%%%%%%%%%%%%%
%%%%%%%%%%%%%% auto 2
\begin{frame}[fragile]
\frametitle{When \emph{must} you use auto?}

\begin{itemize} %1
\item Unutterable types:
  \begin{itemize} %2
    \item Lambdas: what is the type of {\scriptsize\texttt{  [](int i)\{return i > 42;\}}} ?
    \item \emph{Why would you do that?}  E.G., to
      be consistent with ordering:
{\scriptsize
\begin{verbatim} 
auto my_order = [](int i, int j){return i > j;}
set<int,       my_order> set_ordered_my_way;
map<int, char, my_order> map_ordered_the_same_way
\end{verbatim}
}
  
  \end{itemize} %2
\vskip 12pt
\item Implementation-defined types: there are a few here and there.

\end{itemize} %1

\end{frame}

%%%%%%%%%%%%%%%%%%%%%%%%%%%%%%%%%%%%%%%%%%%%%%%%%%%%%%%%%%%%%%%%%%%%%%
%%%%%%%%%%%%%% auto 2
\begin{frame}[fragile]
\frametitle{When \emph{can't} you use auto?}

\begin{itemize} %1
\item When you want an explicit type conversion:
{\scriptsize
\begin{verbatim}
int foo(); 
long i = foo(); 
\end{verbatim}
}
\vskip 6pt
%  \end{itemize} %2

\item When you want polymorphism:
{\scriptsize
\begin{verbatim}
class Base {};    
class Derived : public Base {};
Derived* derivedFactory();
Base* my_ptr = derivedFactory();
\end{verbatim} }     
\vskip 6pt
\item When you don't want an ephemeral / proxy type.
{\scriptsize 
\begin{verbatim}
auto m = downcastPtr(base_ptr);
auto p = _1 = 1; // boost::lambda;
auto m = Eigen::Matrix<..> * Eigen::Matrix<..>;
auto x = { 42 };
\end{verbatim}
}     

%\end{itemize} %2

\end{itemize} %1
\end{frame}

%%%%%%%%%%%%%%%%%%%%%%%%%%%%%%%%%%%%%%%%%%%%%%%%%%%%%%%%%%%%%%%%%%%%%%
%%%%%%%%%%%%%% auto 4
\begin{frame}[fragile]
\frametitle{When \emph{should} you use auto?}

\center{ \textcolor{purple} {Everywhere you can.} }

(note: interface change robustness)
\pause

\center{ \textcolor{purple} {(unless it's trivial?)} }
\center{ \textcolor{purple} {(unless it's confusing?)} }

\pause

\begin{itemize}
\item 
\begin{verbatim} auto pi = 3.14 ; // double 
\end{verbatim} 
\item 
\begin{verbatim} auto pi = 3.14L; // long double 
\end{verbatim} 
\item 
\begin{verbatim} auto pi = 3.14l; // long int!
\end{verbatim} 
\end{itemize}



\end{frame}



%%%%%%%%%%%%%%%%%%%%%%%%%%%%%%%%%%%%%%%%%%%%%%%%%%%%%%%%%%%%%%%%%%%%%%
\begin{frame}[fragile]
\frametitle{\texttt{decltype}: the other side of the coin}
\texttt{decltype(expr)} computes the type of \texttt{expr} at compile time; \texttt{expr} is
\emph{not} evaluated.
{\scriptsize
\begin{verbatim}
int i;         decltype(i)     // int
double j;      decltype(i+j)   // double
int&& foo();   decltype(foo()) // int&&
\end{verbatim}}
Mainly used for type computations in templates:
{\scriptsize
\begin{verbatim}
template<class C> struct Foo {  typedef decltype( C().begin() ) iter_t; };

typedef set<int> Set;

Foo<Set>::iter_t         // Set::iterator
Foo<const Set>::iter_t   // Set::const_iterator

\end{verbatim}}

\center{ \textcolor{purple} {Needed when you need a type for something
that is not a variable, like return types or typedefs.} }


\end{frame}


%%%%%%%%%%%%%%%%%%%%%%%%%%%%%%%%%%%%%%%%%%%%%%%%%%%%%%%%%%%%%%%%%%%%%%

%%%%%
\begin{frame}[fragile]
\frametitle{New ``suffix'' function declaration syntax (`auto' functions)}
\begin{itemize}
\item For library writers, probably not seen in everyday use:
{\scriptsize
\begin{verbatim}
int  foo (double d);        // normal function
auto foo (double d) -> int; // the same thing
\end{verbatim}}
\vskip 6pt
\item Solves a problem for template functions: what type to return?
{\scriptsize
\begin{verbatim}
template<class T, class U>     // The old way:
??? mult(T const& t, U const& u)
    { return t * u; }
\end{verbatim}}
\vskip 6pt

\item The compiler knows what type to return, but we can't express
  that with the old syntax.
{\scriptsize
\begin{verbatim}
template<class T, class U>     // The new way:
auto add(T const& t, U const& u) -> decltype(t+u)
    { return t + u; }
\end{verbatim}}
\vskip 6pt
(Why can't we put the decltype up front?  Because \texttt{t} and
\texttt{u} aren't in scope yet!)
\end{itemize}
\end{frame}

