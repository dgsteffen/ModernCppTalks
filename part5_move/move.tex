
\lyxframeend{}\section[The Solution]{Move Semantics are the Answer}

\subsection[Move Semantics]{Move Semantics}\lyxframeend{}

%\subsection[Rvalues]{Rvalues Only Live Once}\lyxframeend{}

\begin{frame}[fragile]
\frametitle{Enter Move Semantics}

\begin{itemize}[<+->]
\item ``Expensive Copy'' almost always means ``holds expensive resources''.

      \begin{itemize}[<+->]
      \item Copying floats or ints? Invisible
      \item Copying big structs? Easy.  (On stack, memcpy, etc)
\vskip 6pt
      \item \texttt{vector<vector<string>{}>}  ...  memory allocation hell
      \item shared resources -- mutexes (expensive), resource contention
      \item sockets, files, streams -- does copy even make sense?
      \end{itemize}

\vskip 6pt
\pause

\item In all those problem cases, we don't really want a copy: \emph{we're never going to
  use the original again!}
\begin{verbatim}
Vector C =    A + B  + D  + Foo(E+F);
\end{verbatim}


\vskip 6pt
\pause

\item ``Move Semantics'' -- don't copy the data, copy the handle.

    \begin{itemize}[<+->]
    \item Move a \texttt{vector<blah>} -- shallow copy (just copy the pointer)
    \item Shared resources, sockets, files, streams -- transfer the
    resource handle.

    \end{itemize}


\end{itemize}
\end{frame}


\begin{frame}[fragile]
\frametitle{Moved-from Semantics}

\begin{itemize}[<+->]
\item Isn't that a disaster waiting to happen?

    \begin{itemize}
    \item \Emph{We don't care, we didn't want the original anyway.}
    \item \Emph{We \emph{can't} care, the original is a temporary!}
    \item The only thing that can happen to them
    once we're done is destruction. They're gonna die anyway, so...
    \item Rvalues are organ donors. \Emph{We can steal their guts!}
    \end{itemize}

\vskip 6pt

\item  What is the state of a moved-from object?

    \begin{itemize}
    \item Anything it wants; moved-from objects are in an \emph{inconsistent} state.
    \item The only valid operation is destruction; so they need only be \emph{destructable}.
    \item (A detail: it's good for them to be \emph{assignable} as well.)
    \end{itemize}

\vskip 6pt

\item So, how do we know when we can do this to an object?  How do we
    write code for move semantics?

\end{itemize}
\end{frame}



\subsection[Definitions]{Definitions and Terminology}\lyxframeend{}

\begin{frame}[fragile]
\frametitle{Rvalues}

``We aren't going to use this object again'' has a
name: \textcolor{magenta} {rvalue}.


\pause
{
\vskip 6pt

\begin{itemize}[<+->]

\item {\bf lvalue}: can appear on the left side of an assignment
\begin{itemize}     
  \item Has an address and usually a name
  \item EG: variables (note: const objects are still lvalues!)
  \item Reference spelled ``\&'' as usual.
\end{itemize}

\vskip 6pt

\item {\bf rvalues} : can \emph{only} appear on the right side of an assignment
  \begin{itemize}
  \item No name, no accessable address
  \item EG: unnamed temporaries (might be in a register, or other
  compiler-generated no-man's-land)
  \item Rvalue references, spelled \textcolor{magenta}{\&\&} (``ref ref'') are the heart of move semantics.
  \end{itemize}

\vskip 6pt

\item Terminology inherited from C, so it's slighly wonky, but
  everyone is used to it by now.

\end{itemize}
}

\end{frame}

%%%%%%%%%%%%%%%%%%%%%%%%%%%%%%%%%%%%%%%%%%%%%%%%%%%%%%%%%%%%%%%%%%%%%%

\begin{frame}[fragile]
\frametitle{Lvalues and Refs Thereunto}
\framesubtitle{Things you already know}
{\scriptsize
\begin{verbatim}
int f();      // Generator function

int i,j,k;    // Lvalues: objects accessable via identifiers
int* b;       // Lvalue : object accessable via identifier 'b'
b = &i;       // lvalues: object 'i' has accessable address; store in 'b'
int& c = i;   // Lval ref : object 'i' has alternate name 'c',
              //            and indirect handle 'b'
int** d = &b; // Lvalue  : object 'd' holds pointer to 'b', etc.

i+j+k;         // Rvalue: expression, has no name or address
f()+j;         // Rvalue: ditto for function return value.  In both
               // cases, unnamed temps exist only long enough to 
               // participate in enclosing expression.

int q  = i+j;  // q is lvalue, initialized from rvalue

int z  = f();  // ditto

int& r = f();  // error; can't bind rvalue to nonconst reference

const int& s = f() // OK - can bind rvalue to a const ref.
\end{verbatim}
}
\vskip 12pt
\end{frame}

%%%%%%%%%%%%%%%%%%%%%%%%%%%%%%%%%%%%%%%%%%%%%%%%%%%%%%%%%%%%%%%%%%%%%%

\begin{frame}[fragile]
\frametitle{Rvalues and Refs Thereunto}
\framesubtitle{New!}

A reference to an rvalue is spelled ``\&\&'':
{\scriptsize
\begin{verbatim}
int f();     // Generator function
int i,j,k;   // Lvalues: objects accessable via identifiers


int& r1 = i;    // OK, lvalue ref
int& r2 = f();  // ERROR, f() is rvalue


int&& rr1 = f();  // OK, rvalue ref

int&& rr2 = i  ;  // ERROR, var is lvalue

int&& rr3 = true  // OK, ref to temp holding
                  // implicit conversion from bool

string&& ss = "aa"; // Ditto; ref to temp holding
                    // conversion from const char*

\end{verbatim}
}

\end{frame}


%%%%%%%%%%%%%%%%%%%%%%%%%%%%%%%%%%%%%%%%%%%%%%%%%%%%%%%%%%%%%%%%%%%%%%

\begin{frame}[fragile]
\frametitle{Binding Rules}

\newcommand{\ccheck}{{\textcolor{magenta}{\checkmark}}}

\center{

\begin{tabular}{cr|c|c|c|c|}
%\cline{3-6}
\multicolumn{2}{c}{} &  \multicolumn{4}{c}{\bf{Object Type}}\\ 
& \multicolumn{1}{c}{}& \multicolumn{1}{c}{ const Lvalue}
& \multicolumn{1}{c}{Lvalue} & \multicolumn{1}{c}{Rvalue}
& \multicolumn{1}{c}{const Rvalue} \\ \cline{3-6} 
\multirow{4}{*}{\begin{sideways}\bf{Ref Type}\end{sideways}} & const \& & \ccheck & \ccheck & \ccheck & \ccheck \\ \cline{3-6}
 & \& & & \ccheck & & \\ \cline{3-6}
 & \&\& & &  & \ccheck & \\ \cline{3-6}
 & const \&\& & & & & \ccheck \\ \cline{3-6}


\end{tabular}

\vskip 12pt

\begin{itemize}[<+->]
\item Everything  binds to \texttt{const \&}
\item All the rest of the binding rules are \emph{minimal}
\item \texttt{const \&\&} is effectively useless
\end{itemize}

}


\end{frame}
%%%%%%%%%%%%%%%%%%%%%%%%%%%%%%%%%%%%%%%%%%%%%%%%%%%%%%%%%%%%%%%%%%%%%%

\begin{frame}[fragile]
\frametitle{Reference Collapse}

Language syntax doesn't allow \texttt{int\&\& \& a}, but such
things can result from typedefs, template deduction, etc.

{\scriptsize
\begin{verbatim}

typedef int& IR;  

typedef int&& IRR;

// reference collapse

IRR && w ; // "int && && " --> int&&
IRR &  x ; // "int && &  " --> int&
IR  && y ; // "int & &&  " --> int&
IR  &  z ; // "int & &   " --> int&
\end{verbatim}
}

Nothing we can do type fiddling can turn an lvalue into an rvalue by
accident.  (Intentionally, yes, but no \emph{accidental} organ donors.)

\end{frame}

%%%%%%%%%%%%%%%%%%%%%%%%%%%%%%%%%%%%%%%%%%%%%%%%%%%%%%%%%%%%%%%%%%%%%%

\subsection[Coding]{Coding Move Semantics}\lyxframeend{}


\begin{frame}[fragile]
\frametitle{Coding Movable Types}

The canonical form for a movable type:

{\scriptsize
\begin{verbatim}
class Vector { 

private:
  float* data;
  int    len;

public:
  Vector( const Vector&  lvalue );  // copy ctor, deep copy
  Vector(       Vector&& rvalue );  // move ctor, steal

  Vector& operator=(const Vector&  lval);   // copy assignment
  Vector& operator=(      Vector&& rval );  // move assignment
};
\end{verbatim}
}
\pause
\begin{itemize}[<+->]
\item Everything can bind to a const lvalue ref, but
\item Only rvalues can bind to \&\&, and this is a better fit for overload resolution.
\item Therefore, functions that take \&\& know that the referent is
an organ donor.
\end{itemize}

\end{frame}

%%%%%%%%%%%%%%%%%%%%%%%%%%%%%%%%%%%%%%%%%%%%%%%%%%%%%%%%%%%%%%%%%%%%%%

\begin{frame}[fragile]
\frametitle{Moving Vectors, How Does It Work}
{\scriptsize
\begin{verbatim}
Vector(const Vector& v) : data(new float[v.len]), len(v.len) { memcpy(...); }

Vector(Vector&& rv)     : data(rv.data), len(rv.len) { rv.data = NULL; }    
                                                  // note: rv inconsistent!

Vector& operator=(const Vector& v   ) { 
  delete[] data;   
  data = new float[v.len]; 
  len = v.len;  
  memcpy(...);                        }

Vector& operator=(      Vector&& rv ) {
    len  = rv.len;     
    swap(data, rv.data)               }  // rv is inconsistent, rv.len is wrong!

  ~Vector() { delete[] data; }
};

Vector C =    A + B  + D  + Foo(E+F);
\end{verbatim}
}

Move operations leave \texttt{rv} in a state that is \emph{inconsistent} but
\emph{destructable}.
\vfill

\end{frame}


%%%%%%%%%%%%%%%%%%%%%%%%%%%%%%%%%%%%%%%%%%%%%%%%%%%%%%%%%%%%%%%%%%%%%%


\begin{frame}[fragile]
\frametitle{Coding move assignments}
\begin{itemize}[<+->]
\item Swap resources. Cleaner, faster?
{\scriptsize
\begin{verbatim}

  Vector& Vector::operator=(Vector&& rref) {
    swap(data, rref.data);  len = rref.len;
  }

\end{verbatim}
}
\item Delete the unneeded resources directly; deterministic, necessary?
{\scriptsize
\begin{verbatim}

  Vector& Vector::operator=(Vector&& rref) {
    delete[] data;
    data = rref.data;
    rref.data = NULL;
  }
\end{verbatim}
}
This is absolutely necessary for time-constrained resources like
mutexes.

\item Consensus seems to be leaning towards immediate release, but the
debate continues.

\end{itemize}

\end{frame}

%%%%%%%%%%%%%%%%%%%%%%%%%%%%%%%%%%%%%%%%%%%%%%%%%%%%%%%%%%%%%%%%%%%%%%

%% \begin{frame}[fragile]
%% \frametitle{Coding Move Assignments II: Breaking News}

%% Breaking news: Scott Meyers' blog: ``implementing move in terms of swap is
%% often a pessimization, and has other unpleasant implications.''
%% Consider a class that holds resources through a pointer.

%% \begin{itemize}
%% \item \texttt{swap} is three pointer assignments; no swap is only
%% two.  If we're doing this for performance reasons...
%% \item \texttt{swap} means ``my'' resources are cleaned up... later.
%% The work is the same, but the behavior can be very different.
%%     \begin{itemize}
%%     \item Consider an object that with a handle to 51\% of available
%%     RAM... how long can we wait?
%%     \end{itemize}
%% \item The issue is, maybe, still up in the air, but coding move as
%%     swap should probably \emph{not} be the default.
%%     \begin{itemize}
%%     \item Counter argument: remove element $n$ from a sequence by
%%     moving $n+1$ down, then $n+2$ down, etc... ``move by swap'' only
%%     runs one dtor.
%%     \end{itemize}
%% \end{itemize}
%% So, the debate still continues (but the issues become clearer).
%% \end{frame}


% http://accu.org/content/conf2014/Howard_Hinnant_Accu_2014.pdf slides
% 45-53


%%%%%%%%%%%%%%%%%%%%%%%%%%%%%%%%%%%%%%%%%%%%%%%%%%%%%%%%%%%%%%%%%%%%%%

\begin{frame}[fragile]
\frametitle{std::move}
How to force a move from something that isn't an rvalue already:
{\scriptsize
\begin{verbatim}
{

  Vector V;

  ...

  Vector Q = std::move(V);

  // don't use V again!

  cout << V.len;  // undefined behavior!

}
\end{verbatim}
}

This tells the compiler that we are done with an object, and therefore
it can move from it, but we'd better be right.  Lying to the compiler
is a \emph{really bad idea}.


\end{frame}


%%%%%%%%%%%%%%%%%%%%%%%%%%%%%%%%%%%%%%%%%%%%%%%%%%%%%%%%%%%%%%%%%%%%%%

\begin{frame}[fragile]
\frametitle{Writing Functions I}
\begin{itemize}
\item Return values?  As usual.
{\scriptsize
\begin{verbatim}

Foo   bar() { Foo f; ... return f; }      // Do This

Foo&& bar() { Foo f; ... return f; }      // Not This

Foo   bar() { Foo f;                   // Or this; the
              return std::move(f); }   // move is implied

\end{verbatim}
}
The return statement is the last time 'f' can be referenced,
so it's kind of an honorary rvalue.
\vskip 6pt
\texttt{return std::move(var)} is implied when returning a local variable.

\end{itemize}

\end{frame}

%%%%%%%%%%%%%%%%%%%%%%%%%%%%%%%%%%%%%%%%%%%%%%%%%%%%%%%%%%%%%%%%%%%%%%


\begin{frame}[fragile]
\frametitle{Writing Functions II}
\begin{itemize}
\item Pass by value when you want a copy.
{\scriptsize
\begin{verbatim}
void foo ( const bar& b) {  // not this
  bar local = b;
  // ... modify local
}

void foo ( bar local ) {        // but this
 // ... modify local 
}
\end{verbatim}
}
\pause
\begin{center}
\Emph{Don't force the compiler to make a copy if it can move or elide.}
\end{center}
\item Also, the API is clearer -- it's obvious to the caller that a
copy is being made, which has performance implications, and is useful
in other ways.
\begin{center}
\Emph{Maximize the amount of information in an interface.}
\end{center}
\end{itemize}

\end{frame}


%%%%%%%%%%%%%%%%%%%%%%%%%%%%%%%%%%%%%%%%%%%%%%%%%%%%%%%%%%%%%%%%%%%%%%

\begin{frame}[fragile]
\frametitle{Writing Functions III: Binding}
{\scriptsize
\begin{verbatim}
T Foo (const T& a, T& b, T c, T&& d)
{
  // a: lvalue,    const-ref to external l-or-rvalue
  // b: lvalue,    ref to external lvalue (not rvalue!)
  // c: lvalue,    local copy
  // d: lvalue(!), ref to external rvalue

  T e // lvalue, local variable

  T&& r = bar(a,b); // lvalue(!), local ref bound to rvalue
                    //      semantically an lvalue
                    //      but it's an organ donor

  // Return value options: choose one:

  return e;  // e is local, move enabled

  return r;  // same
  
  return d;            // d not local -- copy!
  return std::move(d); // move enabled

  return a;            // copy
  return std::move(a); // BOOM!
}
\end{verbatim}
}

\end{frame}


%%%%%%%%%%%%%%%%%%%%%%%%%%%%%%%%%%%%%%%%%%%%%%%%%%%%%%%%%%%%%%%%%%%%%%

\begin{frame}[fragile]
\frametitle{Wait, What?}
\begin{itemize}
\item \Emph{An rvalue reference is not an rvalue!}  Consider:
{\scriptsize
\begin{verbatim}
void foo (const Widget& lv ); // makes a copy of its arg
void foo (Widget&&      rv ); // destructive, moves from argument
\end{verbatim} }
\pause
\vskip 6pt
\item Now, I write this:
{\scriptsize
\begin{verbatim}
void baz(Widget&& R) 
{
  foo(R);   // Is R moved from here?
  foo(R);   // If so, Boom!
}
\end{verbatim}
}
\vskip 6pt
\pause
\item Is this code reasonable?  Yes.
\pause
\item Is it correct? Yes
\pause
\item Why?  Because \Emph{R has a name, it can be
referred to in more than once place, it's an lvalue} and is
therefore \Emph{not} moved from.

\end{itemize}

\end{frame}


%%%%%%%%%%%%%%%%%%%%%%%%%%%%%%%%%%%%%%%%%%%%%%%%%%%%%%%%%%%%%%%%%%%%%%

\begin{frame}[fragile]
\frametitle{Rvalue References are Lvalues!}
So, consider:
{\scriptsize
\begin{verbatim}
void foo (Widget     lv ); // takes by value, gets a copy
void foo (Widget&&   rv ); // destructive, moves from argument

void bar (Widget&&   rv ); // also destructive, no overloads

void baz(Widget&& R) 
{
  foo(R);            // Is R moved from here? NO!

  ...                // It's safe to use R in here

  bar(R);            // Won't compile!

  bar(std::move(R)); // OK, explicit organ donor

  ...  // Don't use R here, but you knew that!
}
\end{verbatim}
}

Repeat to yourself: \Emph{Rvalue References are Lvalues!} They have
names, can be referred to more than once, and you want it this way.

\end{frame}
%%%%%%%%%%%%%%%%%%%%%%%%%%%%%%%%%%%%%%%%%%%%%%%%%%%%%%%%%%%%%%%%%%%%%%

\begin{frame}[fragile]
\frametitle{Writing Functions III Reprise}
\framesubtitle{Take a second look at that}
{\scriptsize
\begin{verbatim}
A Foo (const A& a, A& b, A c, A&& d)
{
  // a: lvalue,    const-ref to external l-or-rvalue
  // b: lvalue,    ref to external lvalue (not rvalue!)
  // c: lvalue,    local copy
  // d: lvalue(!), ref to external rvalue

  A e // lvalue, local variable

  A&& r = bar(a,b); // lvalue(!), local ref bound to rvalue
                    // semantically an lvalue
                    // but it's an organ donor

  // Return value options: choose one:

  return e;  // e is local, move enabled

  return r;  // same
  
  return d;            // d not local -- copy!
  return std::move(d); // move enabled

  return a;            // copy
  return std::move(a); // BOOM!
}
\end{verbatim}
}

\end{frame}

%%%%%%%%%%%%%%%%%%%%%%%%%%%%%%%%%%%%%%%%%%%%%%%%%%%%%%%%%%%%%%%%%%%%%%

\begin{frame}[fragile]
\frametitle{Writing Functions IV: Returning Values}
{\scriptsize
\begin{verbatim}
Vector operator+(Vector const& x, Vector const& y) // #1
{ Vector temp = x; temp += y; return temp; }
 
Vector operator+(Vector&& temp, const Vector& y)   // #2
{ temp += y; return std::move(temp); }
 
Vector operator+(const Vector& x, Vector&& temp)   // #3
{ temp += x; return std::move(temp); }
 
Vector operator+(Vector&& temp, Vector&& y)        // #4
{ temp += y; return std::move(temp); }
\end{verbatim}
}
\begin{itemize}
\item \#1 : Move enabled, because \texttt{temp} is an automatic
  variable: \Emph{provably} never used again!
\item \#2 -- \#4: Use \texttt{std::move} because \texttt{temp} is \emph{not}
  an automatic variable

\item Again: a named rvalue reference is an lvalue! (It's not unnamed any
  more!) Inside \#2, \texttt{temp} is an lvalue, the compiler isn't
  allowed to move from it without our explicit permission.
\end{itemize}


\end{frame}



%%%%%%%%%%%%%%%%%%%%%%%%%%%%%%%%%%%%%%%%%%%%%%%%%%%%%%%%%%%%%%%%%%%%%%

\begin{frame}[fragile]
\frametitle{Aggregating move-enabled classes}
{\scriptsize
\begin{verbatim}
struct Widget {  

 std::string s;

  Widget(Widget&& rhs)
    : s(rhs.s)             // no move, copies
   {...}

  Widget(Widget&& rhs)
    : s(std::move(rhs.s))  // moves
   {...}
\end{verbatim}
}
\begin{itemize}
\item Remember, binding an rvalue to a name produces an lvalue (it
  has a name!); it's just an lvalue that you can cannibalize.
\end{itemize}

\center{\Emph{No accidental organ donors!}}

\end{frame}

%%%%%%%%%%%%%%%%%%%%%%%%%%%%%%%%%%%%%%%%%%%%%%%%%%%%%%%%%%%%%%%%%%%%%%

\begin{frame}[fragile]
\frametitle{Library Changes}
The Standard Library has a variety of changes involving move semantics
and perfect forwarding.  In particular, containers now
have \texttt{emplace} operations.
{\scriptsize
\begin{verbatim}
iterator insert (iterator pos, const T& v);   // basic insert via copy

// Equivalent operation, but construct the element in place
template<class... Args> iterator emplace (iterator pos, Args&&... a); 
\end{verbatim}}
\pause
{\scriptsize \begin{verbatim}
std::vector<Widget> v;
Widget w(a, b, c);       // make a Widget with some arguments

v.insert(pos, w);        // pos is iterator; insert copy of w
v.emplace(pos, a, b, c); // insert a Widget constructed in place -- no copy!

v.push_back(w);          // add a copy of w to the end
v.emplace_back(a,b,c);   // create new Widget at the end, but...

v.push_back(a,b,c);      // GCC doesn't implement the standard here --
                         // it just overloads.
\end{verbatim}
}
\pause
\begin{center}
Know that containers (now) have emplace operations.
\end{center}

\end{frame}


%%%%%%%%%%%%%%%%%%%%%%%%%%%%%%%%%%%%%%%%%%%%%%%%%%%%%%%%%%%%%%%%%%%%%%

\begin{frame}[fragile]
\frametitle{Final Comments}
\begin{itemize}

\item Can something be movable but not copyable?
\pause
      \begin{itemize}
      \item Absolutely.  Example: the standard I/O streams are movable but not
      copyable.
      \item Canonical example: \texttt{unique\_ptr}.  Replaces \texttt{auto\_ptr}, which
      is broken, and had to be, because it takes move semantics to get
      it right.
      \item Standard containers can now store
      movable-but-not-copyable objects, so \texttt{vector<ostream>} is now possible.
      \end{itemize}

\vskip 6pt
\pause
\item Standard containers are move-aware?
\pause
      \begin{itemize}
      \item Yes, and this can improve performance \emph{without any
      code changes}.
      \end{itemize}

\vskip 6pt

%% \item Can templates detect movable types at compile time?
%%       \begin{itemize}
%%       \item Yes, but (probably) only with \CC11 template facilities,
%%       and even then it's rough.  \CC17 concepts will help.
%%       \end{itemize}
\pause
\item Further Reading:
      \begin{itemize}
      \item Thomas Becker: {\tiny  \url{http://thbecker.net/articles/rvalue_references/section_01.html}}

      \item  Scott Meyer's blog:
      {\tiny \url{http://scottmeyers.blogspot.com/2014/06/the-drawbacks-of-implementing-move.html}}
      \item \mbox{Scott Meyers on Universal References:}
      {\tiny \url{http://channel9.msdn.com/Shows/Going+Deep/Cpp-and-Beyond-2012-Scott-Meyers-Universal-References-in-Cpp11}}
      \end{itemize}

\end{itemize}

\end{frame}

