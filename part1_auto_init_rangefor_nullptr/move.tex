
\subsection{\&\& / Move}\lyxframeend{}

%%%%%%%%%%%%%%%%%%%%%%%%%%%%%%%%%%%%%%%%%%%%%%%%%%%%%%%%%%%%%%%%%%%%%%
%%%%%%%%%%%%%% auto 1

\begin{frame}[fragile,t]
\frametitle{Copies Are The Problem}

\begin{itemize}

\item{\scriptsize
\begin{verbatim}
vector<string> get_names() {... return m_names;} // #1
vector<string> names = get_names(); // #2
\end{verbatim}
}

\item In principle, two copies of the vector, and a default constructor.

\item If there are N strings in the vector, each copy could require as many
as N+1 memory allocations and a whole slew of cache-unfriendly data
accesses as the string contents are copied.


\vskip 12pt
\item Option 1: don't ever change the result
{\scriptsize
\begin{verbatim}
vector<string> const& names = get_names();
\end{verbatim}
}

\vskip 12pt

\item Option 2: commit to smart pointers
{\scriptsize
\begin{verbatim}
shared_ptr<vector<string>> get_names();
\end{verbatim}
}
\vskip 12pt

\item Option 3: FORTRAN-style ``out parameters''
{\scriptsize
\begin{verbatim}
void get_names( vector<string>& m_names);
\end{verbatim}
}


\end{itemize}

\end{frame}

%%%%%%%%%%%%%%%%%%%%%%%%%%%%%%%%%%%%%%%%%%%%%%%%%%%%%%%%%%%%%%%%%%%%%%

\begin{frame}[fragile,t]
\frametitle{Copies Are Inevitable}
{\scriptsize
\begin{verbatim}
class Vector {...};

Vector operator+ (Vector const& A, Vector const& B) {
  Vector result(A);
  return result += B;
}
...
Vector C =    A + B  + D  + Foo(E+F);
   --->  = (((A + B) + D) + Foo(E+F));
\end{verbatim}
}

\begin{itemize}
\item Five subexpressions means five unnamed temporaries, means five
  copies, and then the assignment.
\item Options 2 and 3 are not an option unless we give up the syntax.
\item Option 1 doesn't help and may not be a viable option anyway.
\end{itemize}

\uncover<2>{ \hfill \textcolor{purple} {Copies are the bane of
    high-performance computing in \CC.} \hfill}


\end{frame}

%%%%%%%%%%%%%%%%%%%%%%%%%%%%%%%%%%%%%%%%%%%%%%%%%%%%%%%%%%%%%%%%%%%%%%

\begin{frame}[fragile,t]
\frametitle{Compilers aren't that stupid}

\textcolor{red}{Copy Elision to the rescue!}

\begin{itemize}
\item<1-> All modern compilers make these the same:
{\scriptsize
\begin{verbatim}
vector<string> names = get_names();
vector<string> names(get_names);
\end{verbatim}}

\item<1->Return Value Optimization (RVO):{\scriptsize
\begin{verbatim}
vector<string> get_names() {
  vector<string> result;
  ...
  return result; } 

  vector<string> names = get_names();
\end{verbatim}}

\item<1->''Unnamed RVO'':{\scriptsize
\begin{verbatim}
vector<string> get_names() {
  string twoB, not2B;
  ...
  return twoB + " or " + not2B; } 

  vector<string> names = get_names();
\end{verbatim}}

\textcolor{red}{Compilers can skip copy operations \emph{if doing so
    has no observable effects}.}
\end{itemize}
\end{frame}

\begin{frame}[fragile,t]
\frametitle{... but compilers are limited}

\textcolor{red}{Copy elision is fragile}


\begin{itemize}
\item<1-> Copy elision not allowed if these have side effects:
{\scriptsize
\begin{verbatim}
vector<string> names = get_names();
vector<string> names(get_names);
\end{verbatim}
}

\item<1-> NRVO / URVO happens, on a compiler-by-compiler basis.
\item<1-> RVO is easily disabled in compiler specific ways:
{\scriptsize
\begin{verbatim}
Foo makefoo() {
  Foo a;
  ...
  if (expr) return a;
  ...
  return b;
}
\end{verbatim}
}
\end{itemize}

\textcolor{red}{Copy elision happens at the whim of the compiler, and
  there's not much you can do to help.}
\end{frame}



\begin{frame}[fragile]
\frametitle{Enter Move Semantics}

In all these cases we don't want a copy: \emph{we're never going to
  use the original again!}

\begin{verbatim}
Vector C =    A + B  + D  + Foo(E+F);
\end{verbatim}


``We aren't going to use this object again'' has a name:

\uncover<2->{ \hfill \textcolor{purple} {rvalue reference, spelled \&\&} \hfill}
\uncover<3>{

\begin{itemize}
\item {\bf lvalue}: can appear on the left side of an assignment
\item[] EG: variables \note{const objects are lvalues!}
\item {\bf rvalues} : can \emph{only} appear on the right side of an assignment
\item[] EG: literals, unnamed temporaries
\end{itemize}
}

\end{frame}
%%%%%%%%%%%%%%%%%%%%%%%%%%%%%%%%%%%%%%%%%%%%%%%%%%%%%%%%%%%%%%%%%%%%%%
\begin{frame}[fragile]
\frametitle{rvalues}

Unnamed temporaries can only be referenced (viewed) once.  Therefore,
they can be organ donors.

{\scriptsize
\begin{verbatim}
class A {
  A( const A& lvalue };          // copy ctor
  A& operator=(const A& lvalue)  // copy assignment
  A( A&& rvalue );               // move ctor
  A& operator=(A&& rvalue);      // move assignment

  shared_ptr<Foo> r;
}
\end{verbatim}
}

Move: steal the guts from the organ donor.

{\scriptsize
\begin{verbatim}
A::A( A&& rvalue ) : r(rvalue.r) {}
\end{verbatim}
}
Usage:
{\scriptsize
\begin{verbatim}
A make_an_A();
...A a;
...
a = make_an_A();
\end{verbatim}
}
No Copies!
\end{frame}

%%%%%%%%%%%%%%%%%%%%%%%%%%%%%%%%%%%%%%%%%%%%%%%%%%%%%%%%%%%%%%%%%%%%%%

\begin{frame}[fragile]
\frametitle{Moving Vectors}
{\scriptsize
\begin{verbatim}
class Vector {
  double* data;   int len;

  Vector(Vector&& rref) : data(rref.data), len(rref.data)  {
    rref.m_data = nullptr;
  }

  Vector& operator=(Vector&& rref) {
    swap(data, rref.data)
    len  = rref.len;
  }

  ~Vector() { delete[] m_data; }

};


Vector C =    A + B  + D  + Foo(E+F);

\end{verbatim}
}

Moved-from objects are in an \emph{inconsistent} but
\emph{destructable} state.

\end{frame}

%%%%%%%%%%%%%%%%%%%%%%%%%%%%%%%%%%%%%%%%%%%%%%%%%%%%%%%%%%%%%%%%%%%%%%

\begin{frame}[fragile]
\frametitle{std::move}
Force a move from something that isn't an rvalue already:
{\scriptsize
\begin{verbatim}

Vector V;

...

Vector Q = std::move(V);

// don't use V again!

cout << V.len;  // undefined behavior!

\end{verbatim}
}

\end{frame}

%%%%%%%%%%%%%%%%%%%%%%%%%%%%%%%%%%%%%%%%%%%%%%%%%%%%%%%%%%%%%%%%%%%%%%

\begin{frame}[fragile]
\frametitle{Coding move assignments}
\begin{enumerate}
\item Swap resources. Cleaner, faster?
{\scriptsize
\begin{verbatim}
class Vector {
  double* data;

  Vector& operator=(Vector&& rref) {
    swap(data, rref.data)
  }

  ~Vector() { delete[] m_data; }
};
\end{verbatim}
}
\item Delete the unneeded resources directly; deterministic, necessary?
{\scriptsize
\begin{verbatim}
class Vector {
  ...

  Vector& operator=(Vector&& rref) {
    delete[] data;
    data = rref.data;
    rref.data = nullptr;
  }
};
\end{verbatim}
}
\end{enumerate}

The debate continues.
\end{frame}

%%%%%%%%%%%%%%%%%%%%%%%%%%%%%%%%%%%%%%%%%%%%%%%%%%%%%%%%%%%%%%%%%%%%%%

\begin{frame}[fragile]
\frametitle{Writing Functions}
\begin{itemize}
\item Return values?  As usual.
{\scriptsize
\begin{verbatim}
// Do this:               Not this!

Foo bar() {             Foo&& bar() {
  Foo f;                  Foo f;
  ...                     ...
  return f;               return f;
}                       }

\end{verbatim}
}
\item Pass by value when you want a copy.
{\scriptsize
\begin{verbatim}
void foo ( const bar& b) {  // not this
  bar local = b;
  // ... modify local
}

void foo ( bar b ) {        // but this
 // ... modify local b
\end{verbatim}
}
\end{itemize}

\end{frame}


%%%%%%%%%%%%%%%%%%%%%%%%%%%%%%%%%%%%%%%%%%%%%%%%%%%%%%%%%%%%%%%%%%%%%%

\begin{frame}[fragile]
\frametitle{Writing Functions II: Binding}
{\scriptsize
\begin{verbatim}
void Foo (const A& a, A& b, A c, A&& d)
{
  // a: lvalue,    const-ref to external l-or-rvalue
  // b: lvalue,    ref to external l-or-rvalue
  // c: lvalue,    local copy
  // d: lvalue(!), ref to external rvalue

  A e // lvalue, local variable

  A&& r = bar(a,b); // lvalue(!) - rref to local rvalue
                    // semantically an lvalue
                    // but it's an organ donor

  // Return value options: choose one:

  return e;  // e is local, move enabled

  return r;  // same
  
  return d;            // d not local -- copy!
  return std::move(d); // move enabled

  return a;            // copy
  return std::move(a); // BOOM!
}
\end{verbatim}
}

\end{frame}



%%%%%%%%%%%%%%%%%%%%%%%%%%%%%%%%%%%%%%%%%%%%%%%%%%%%%%%%%%%%%%%%%%%%%%

\begin{frame}[fragile]
\frametitle{Writing Functions III}
{\scriptsize
\begin{verbatim}
Matrix operator+(Matrix const& x, Matrix const& y) // #1
{ Matrix temp = x; temp += y; return temp; }
 
Matrix operator+(Matrix&& temp, const Matrix& y)   // #2
{ temp += y; return std::move(temp); }
 
Matrix operator+(const Matrix& x, Matrix&& temp)   // #3
{ temp += x; return std::move(temp); }
 
Matrix operator+(Matrix&& temp, Matrix&& y)        // #4
{ temp += y; return std::move(temp); }
\end{verbatim}
}

\begin{itemize}
\item \#1 : Move enabled, because \texttt{temp} is an automatic
  variable: provably never used again!
\item \#2 -- \#4: Use \texttt{std::move} because \texttt{temp} is \emph{not}
  an automatic variable

\item A named rvalue reference is an lvalue! (It's not unnamed any
  more!) Inside \#2, \texttt{temp} is an lvalue reference.
\end{itemize}


\end{frame}


%%%%%%%%%%%%%%%%%%%%%%%%%%%%%%%%%%%%%%%%%%%%%%%%%%%%%%%%%%%%%%%%%%%%%%

\begin{frame}[fragile]
\frametitle{Coding move-enabled classes}
{\scriptsize
\begin{verbatim}
struct Widget {  

 std::string s;

  Widget(Widget&& rhs)
    : s(rhs.s)             // Incorrect!
   {...}

  Widget(Widget&& rhs)
    : s(std::move(rhs.s))  // Correct!
   {...}
\end{verbatim}
}
\begin{itemize}
\item Remember, binding an rvalue to something produces an lvalue (it
  has a name!); it's just an lvalue that you can cannibalize.
\end{itemize}

\end{frame}


%%%%%%%%%%%%%%%%%%%%%%%%%%%%%%%%%%%%%%%%%%%%%%%%%%%%%%%%%%%%%%%%%%%%%%

\begin{frame}[fragile]
\frametitle{Move rules of thumb}
\begin{itemize}
\item Leave moved-from objects in a destructable and assignable state.
\item Move operations preserve user-visible side effects on the LHS.
  (E.G., LHS's resources get released immediately.  General
  consensus?)
\item Return values from functions as values, not \&\& or any other
  wierd thing
\item Pass by value if you want a copy ('cause you might get a move!)
\item Tell the compiler you're done with an object with std::move()
\item These rules of thumb haven't quite been worked out -- see our
  coding standards for some other issues.
\item Remember: move is an optimization of copy, so all of the above
  can be ignored without any consequence worse than missed
  optimization opportunities.
\end{itemize}

\end{frame}
