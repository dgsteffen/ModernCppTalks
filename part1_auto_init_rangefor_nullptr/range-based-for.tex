
\subsection{Range-based for} \lyxframeend{}

%%%%%%%%%%%%%%%%%%%%%%%%%%%%%%%%%%%%%%%%%%%%%%%%%%%%%%%%%%%%%%%%%%%%%%%%%%%%%%%%
\begin{frame}[fragile]
\frametitle{Range-based for, begin, end}

\begin{columns}[t]
\column{.6\textwidth}

\begin{itemize}
\item  <1->{\scriptsize\begin{verbatim}for(auto i = v.begin(); i != v.end() ++i)   \end{verbatim}}
\item[]<1->{\scriptsize\begin{verbatim}   total += *i; \end{verbatim}}
\item<2->{\scriptsize\begin{verbatim}sort(v.begin(), v.end());  \end{verbatim}}
\item<3->{\scriptsize\begin{verbatim}sort(&a[0], &a[1]+sizeof(a)/sizeof(a[0]));  \end{verbatim}}

\item<4-> {\scriptsize
\begin{verbatim}
vector<int> v;     // which is faster?
for( const auto& i : v)  // #1
  ..
for( const long& i : v)  // #2
  ..
\end{verbatim}
}

\end{itemize}

\column{.4\textwidth}
\begin{itemize}
\item  <1->{\scriptsize\begin{verbatim}for (const auto& d : v )  \end{verbatim}}
\item[]<1->{\scriptsize\begin{verbatim}   total += d; \end{verbatim}}
\item<2->{\scriptsize\begin{verbatim}sort(begin(v), end(v));  \end{verbatim}}
\item<3->{\scriptsize\begin{verbatim}sort(begin(a), end(a));  \end{verbatim}}
\item<5-> \#1 is faster: the type mismatch \emph{forces} a copy on each iteration.
\end{itemize}

\end{columns}

\end{frame}

%%%%%%%%%%%%%%%%%%%%%%%%%%%%%%%%%%%%%%%%%%%%%%%%%%%%%%%%%%%%%%%%%%%%%%%%%%%%%%%%

\begin{frame}[fragile]
\frametitle{Range-based for, details}

Some details: given a \texttt{vector<int> v} ...


\begin{columns}[t]
\column{.6\textwidth}

\begin{itemize}
\item<1-> {\scriptsize \begin{verbatim}
for( const auto& i : v)
  ..
\end{verbatim}}
\item<2-> Faster -- 'auto' makes the right type
\end{itemize}

\column{.4\textwidth}
\begin{itemize}

\item  <1->{\scriptsize\begin{verbatim}
for( const long& i : v)
  ..
\end{verbatim}
}
\item<2-> Slower, we have to make copies
\end{itemize}

\end{columns}


\visible<3->{\Emph{If we take the 'const' off...}}

\begin{columns}[t]
\column{.6\textwidth}

\begin{itemize}
\item<4-> {\scriptsize \begin{verbatim}
for( auto& i : v)
  i *= 2;
\end{verbatim}
}
\item<5-> Fine, modifies the vector
\end{itemize}

\column{.4\textwidth}
\begin{itemize}

\item  <4->{\scriptsize\begin{verbatim}
for( long& i : v)
  i *= 2;
\end{verbatim}
}
\item<5-> Won't Compile! Can't bind a non-const ref to a temporary.
\end{itemize}

\end{columns}
\end{frame}

%%%%%%%%%%%%%%%%%%%%%%%%%%%%%%%%%%%%%%%%%%%%%%%%%%%%%%%%%%%%%%%%%%%%%%%%%%%%%%%%


\begin{frame}[fragile]
\frametitle{Would you ever want to make temporaries?}
\framesubtitle{The Charge of a Thousand Anonymous Rhinos}
{\scriptsize \begin{verbatim}
class Rhino {
  Rhino(int weight);
  void Charge() const;  // has side effects
  // big, memory-consuming member data here
};
\end{verbatim}}
\begin{itemize}
\item Suppose we have a vector<int> v, with 1000 entries (Rhino weights).
\item Suppose Rhinos take up a lot of space, so we don't want a
  vector<Rhino> with 1000 elements
\end{itemize}

\pause{}

{\scriptsize \begin{verbatim}
for (const auto& i : v)   // Loop through vector
{
  Rhino bob(i);           // Make short-lived Rhino
  bob.Charge();           // Charge!
}
\end{verbatim}
}

\pause{}

Is the same as

{\scriptsize \begin{verbatim}
for (const Rhino& bob : v)  // Loop through vector making Rhino temps
{                           // with no explicit variable creation
  bob.Charge();             // Charge!
}
\end{verbatim}
}



\end{frame}


%%%%%%%%%%%%%%%%%%%%%%%%%%%%%%%%%%%%%%%%%%%%%%%%%%%%%%%%%%%%%%%%%%%%%%%%%%%%%%%%
\begin{frame}[fragile]
\frametitle{Range-based for loops: Advice}

\begin{itemize}[<+->]

\item \Emph{Prefer range-based loops when possible}
  \begin{itemize}
    \item They more clearly express intent.
  \item Generally, this means ``when looping over the whole container''.
  \end{itemize}

\item \Emph{Prefer free-function begin/end} : they're more adaptable

\item \Emph{ And use auto! }
\end{itemize}

\end{frame}
