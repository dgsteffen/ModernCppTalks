
\subsection{Class Extensions}
\lyxframeend{}

%\lyxframe{auto} 

%%%%%%%%%%%%%%%%%%%%%%%%%%%%%%%%%%%%%%%%%%%%%%%%%%%%%%%%%%%%%%%%%%%%%%
%%%%%%%%%%%%%% auto 1
\begin{frame}[fragile]
\frametitle{Delegating Ctors}
\framesubtitle{(Mostly directly from Stroustrup's website)}
\begin{columns}[t]
\column{.5\textwidth}
{\scriptsize

\begin{verbatim}  
class X {
  int a;

  validate(int x) { stuff }

 public:

  X(int x) { validate(x); }
  X()      { validate(42); }
  X(string s) { int x = lexical_cast<int>(s); 
                validate(x); }
};
\end{verbatim}
}
\pause{}
\column{.5\textwidth}
{\scriptsize
\begin{verbatim} 
class X {
  int a;



 public:

  X(int x)            { stuff  }
  X()         : X{42} { }
  X(string s) : X{lexical_cast<int>(s)} { }
};
\end{verbatim}
}
\end{columns}
\pause{}
\begin{itemize}
\item Gotchas: none known yet.
\item Only the delegating ctor is allowd in the init list.  Delegate
  or initialize, not both.
\item Lifetime begins when \Emph{any} constructor completes.
  Therefore an exception thrown from the body of a
  delegating constructor will cause the destructor to be invoked.

\item Note that delegating to a base class' ctors has always been
supported.  Now we can delegate to our own.

\end{itemize}
\end{frame}

%%%%%%%%%%%%%%%%%%%%%%%%%%%%%%%%%%%%%%%%%%%%%%%%%%%%%%%%%%%%%%%%%%%%%%
%%%%%%%%%%%%%%
\begin{frame}[fragile]
\frametitle{Inheriting Ctors}

\begin{columns}[t]
\column{.5\textwidth}
A member of a base class is not in the same scope as a member of a derived class:
{\scriptsize
\begin{verbatim} 
struct B            {void f(double);};

struct D : public B {void f(int);};

B b;   b.f(4.5);  // fine
D d;   d.f(4.5);  // surprise: calls f(int) 
                  // with argument 4
\end{verbatim}
}
\pause{}
\column{.5\textwidth}
In \CC98, we can "lift" a set of overloaded functions from a base class into a derived class:
{\scriptsize
\begin{verbatim}
struct B {void f(double);};

struct D : public B {
  using B::f;  // bring all f()s from B 
               // into scope

  void f(int); // add a new f()
};

B b;   b.f(4.5); // fine
D d;   d.f(4.5); // fine: calls D::f(double) 
                 // which is B::f(double)
\end{verbatim}
}
\end{columns}
\pause{}
\vskip 24pt
"Little more than a historical accident prevents using this to work for a constructor as well as for an ordinary member function." --Stroustrup
\end{frame}

\begin{frame}[fragile]
\frametitle{Inheriting Ctors II}

Syntax: 


{\scriptsize
\begin{verbatim}
struct B
{
    B(int);    // normal constructor 1
    B(string); // normal constructor 2
};

struct D : public B
{
    using B::B; // inherit (all!) constructors from B
};
\end{verbatim}
}
This implicitly defines
{\scriptsize
\begin{verbatim}
D::D(int);    // inherited
D::D(string); // inherited
\end{verbatim}
}
\end{frame}

\begin{frame}[fragile]
\frametitle{Inheriting Ctors III}
Gotcha: inherit ctor in a derived class with new member variables that
need initialization:

{\scriptsize
\begin{verbatim}
struct B1 {
  B1(int i) : a(i) { }
  int a;
};

struct D1 : public B1 {
  using B1::B1; // implicitly declares D1(int)
  int x;
};

D1 d(6);	// Oops: d.x is not initialized
D1 e;           // error: D1 has no default constructor
\end{verbatim}
}

Inherited ctors don't know about added member data, but you can call
base ctors directly:
{\scriptsize
\begin{verbatim}
struct D1 : public B1 {
  using B1::B1;
  D1(int a, int b) : B1(a), x(b) {}
}
\end{verbatim}
}

\end{frame}


\begin{frame}[fragile]
\frametitle{Override controls: override}
\framesubtitle{(straight from the isocpp website)}
Confusing:
\begin{columns}[t]
\column{.5\textwidth}
{\scriptsize
\begin{verbatim}
struct B {
    virtual void f();
    virtual void g() const;
    virtual void h(char);
    void k();   // not virtual
};
struct D : B {
    void f();   // overrides B::f()
    void g();   // doesn't override B::g() 
                //   (wrong type)
    void h(char); // overrides B::h()

    void k();   // doesn't override B::k(), 
                // B::k() is not virtual
};
\end{verbatim}
}
\pause{}
\column{.5\textwidth}
{\scriptsize
\begin{verbatim}
 
 
 
 
 

struct D : B {
  void f() override;  // OK: overrides B::f()
  void g() override;  // error: wrong type

  void h(char);       // overrides B::h(); 
                      //   likely warning
  void k() override;  // error: B::k() 
                      //   is not virtual
};
\end{verbatim}
}
\end{columns}
\pause{}
\begin{itemize}
\item A declaration marked override is only valid if there is a function to override.
\item 'override' is only a contextual keyword, so you can still use it
  as an identifier (but arguably you shouldn't).
\end{itemize}
\end{frame}


\begin{frame}[fragile]
\frametitle{Override controls: final}
\framesubtitle{(straight from the isocpp website)}
How to keep a virtual function from beeing overridden?
\begin{columns}[t]
\column{.5\textwidth}
\Emph{\CC 03: comments...}
{\scriptsize
\begin{verbatim}
struct A { virtual void f(); };

struct B : public A {
  // DO NOT OVERRIDE, THIS MEANS YOU
  virtual void f(); 
};

struct C : public B {
  // compiler doesn't read comments
  virtual void f(); 
};
\end{verbatim}
}
\pause{}
\column{.5\textwidth}
\Emph{\CC 11: final keyword}
{\scriptsize
\begin{verbatim}
struct A { virtual void f(); };

struct B : public A {
  virtual void f() final; 
};

struct C : public B {
  virtual void f(); // error
};
\end{verbatim}
}
\end{columns}
\pause{}
\begin{itemize}
\item Use for semantic purposes only, not optimization (this is not Java)
\item Frequently used to cover OO design error.  Closes off future
  customization... do you really want to do that?
\item final is only a contextual keyword, so you can still use it as
  an identifier (but you probably shouldn't).
\end{itemize}
\end{frame}




\begin{frame}[fragile]
\frametitle{Deleted Members}
\framesubtitle{(straight from the isocpp website)}
How to prohibit copying?
\begin{columns}[t]
\column{.5\textwidth}
\Emph{\CC 03: The old way...}
{\scriptsize

\begin{verbatim}
struct A {

 // ...
private:

 A(const A&); // copy ctor
 A& operator=( const A&);
};
\end{verbatim}

Make copy ops private and don't define them
}
\pause{}
\column{.5\textwidth}
\Emph{\CC 11: explicitly delete 'em}
{\scriptsize
\begin{verbatim}
struct A {

 // ...
 A(const A&) = delete; 
 A& operator=(const A&) = delete;
};
\end{verbatim}

Explicitly prevent compiler from generating the default implementations.
}
\end{columns}

\pause{}
\vskip 12pt
Can also be used to eliminate undesired converions:
{\scriptsize
\begin{verbatim}
struct Z {
  // ...
  Z(long long);     // can initialize with a long long
  Z(long) = delete; // but not anything smaller
};
\end{verbatim}
}

\end{frame}



\begin{frame}[fragile]
\frametitle{Defaulted Members}
\framesubtitle{(straight from the isocpp website)}
Conversely, sometimes we'd like the compiler to provide a default: 
{\scriptsize
\begin{verbatim}
class Y {
    // ...
    Y(const Widget&); // disables compiler-generated defaults

    Y& operator=(const Y&) = default;   // default copy semantics
    Y(const Y&) = default;
};
\end{verbatim}
}
\pause{}
\begin{itemize}
\item Useful when the compiler-generated defaults have been disabled (by
providing other ctors)
\item Arguably, can increase clarity even when not needed.
\end{itemize}
\end{frame}
