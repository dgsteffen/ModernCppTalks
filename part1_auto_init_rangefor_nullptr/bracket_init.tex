%%%%%%%%%%%%%%%%%%%%%%%%%%%%%%%%%%%%%%%%%%%%%%%%%%%%%%%%%%%%%%%%%%%%%%
\subsection{\{\} initialization} \lyxframeend{}

\begin{frame}[fragile]
\frametitle{Uniform initialization and initializer lists}

\begin{columns}[t]
\column{.5\textwidth}
\begin{itemize}
\item<1->{\scriptsize\begin{verbatim} complex<double> c (2,72, 3.14); \end{verbatim}}
\item<1->{\scriptsize\begin{verbatim} rectangle w (origin(), extents());\end{verbatim}}

\note{NOTE: rectangle, origin, extents all types!}

\item<1->{\scriptsize\begin{verbatim} int i[] = { 1, 2, 3, 4 };  \end{verbatim}}
\item<1->{\scriptsize\begin{verbatim} vector<int> v;  \end{verbatim}}
\item[]<1->{\scriptsize\begin{verbatim} for(int i=1; i<=4; ++i)   \end{verbatim}}
\item[]<1->{\scriptsize\begin{verbatim}   v.push(back(i)); \end{verbatim}}
\item<3->{\scriptsize\begin{verbatim} int i = 42;  \end{verbatim}}
\item<4->{\scriptsize\begin{verbatim} draw_rect(rectangle(orig,ext));  \end{verbatim}}
\end{itemize}

\column{.5\textwidth}
\begin{itemize}
\item<2->{\scriptsize\begin{verbatim} complex<double> c {2,72, 3.14}; \end{verbatim}}
\item<2->{\scriptsize\begin{verbatim} rectangle w {origin(), extents()};\end{verbatim}}
\item<2->{\scriptsize\begin{verbatim} int         i { 1, 2, 3, 4 }; \end{verbatim}}
\item<2->{\scriptsize\begin{verbatim} vector<int> v { 1, 2, 3, 4 };  \end{verbatim}}
\item[] <2->{\scriptsize\begin{verbatim} \end{verbatim}}
\item[] <2->{\scriptsize\begin{verbatim} \end{verbatim}}
\item<3->{\scriptsize\begin{verbatim} int i = {42}; // silly but ok\end{verbatim}}
\item<4->{\scriptsize\begin{verbatim} draw_rect({orig,ext});  \end{verbatim}}
\item[]<4->{\scriptsize(which may or may not be good style, depending on context
  and conventions, but it's logically consistent.)}
\end{itemize}

\end{columns}
\end{frame}

\begin{frame}[fragile,t]
\frametitle{\CC's Most Vexing Parse\textsuperscript{TM} ... Solved!}
A longstanding issue that is so infamous it has it's own Wikipedia
entry:
{\scriptsize\begin{verbatim}
struct S {   S();           };
struct T {   T(const S& )  ;};

S s();       // #1 invoke the default ctor? No!
T t( S() );  // #2 invoke with a default-constructed S? No!
\end{verbatim}

\begin{itemize}
\item \#1 declares a function named 's' taking no arguments and
  returning an S.
\item \#2 declares a function named 't' that returns a T and taking as parameters, \emph{a pointer to
  a function taking no arguments and returing an S}.  Really.
\item This, due to a rule from C, that states that any construct that
  might possibly be a function declaraion \emph{is} a function
  declaration.
\end{itemize}

\Emph{Uniform initialization syntax to the rescue!}
\begin{verbatim}
S s{};         // Unambiguously invokes the default ctor
S s  ;         // The same, of course
T t { S() };   // Unambiguously T::T(const S&)
\end{verbatim}
}
Henceforth, \CC's Most Vexing Parse\textsuperscript{TM} can be
relegated to the dustbin if history... and legacy code.
\end{frame}




\begin{frame}[fragile,t]
\frametitle{Uniform initialization and initializer lists}
\center{ \textcolor{purple} {Use \{\} unless you have reason not to. }}
\begin{itemize}
\item int i = 42; // too simple and obvious to change
\vskip 6pt
\item Unless you need to construct with args that would be a valid initializer list: 
\begin{verbatim}
vector<int> v(100); // 100 ints
vector<int> v{100}; // one element of value 100
\end{verbatim}
\vskip 6pt
\item Unless there is ambiguity: (GCC bug 52522)
\begin{verbatim}
Matrix hC(const std::vector<Vector>& vectors);
Matrix hC(const std::vector<Matrix>& matrices);
Matrix a,b;

hC({a,b});              // BOOM

hC(vector<Matrix>{a,b}) // OK

\end{verbatim}
\end{itemize}
\end{frame}


\begin{frame}[fragile,t]
\frametitle{Uniform initialization and initializer lists}

\center{ \textcolor{purple} {Use \{\} unless you have reason not to. }}
\vskip 12pt
\begin{itemize}
\item Unless you want to use auto
\begin{verbatim}
int n;
auto w(n);    // int
auto x = n;   // int
auto y {n};   // std::initializer_list<int>
\end{verbatim}
\end{itemize}
\end{frame}
