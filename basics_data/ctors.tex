
\subsection{Delegating Constructors}
\lyxframeend{}

%\lyxframe{auto}

%%%%%%%%%%%%%%%%%%%%%%%%%%%%%%%%%%%%%%%%%%%%%%%%%%%%%%%%%%%%%%%%%%%%%%
%%%%%%%%%%%%%% auto 1
\begin{frame}[fragile]
\frametitle{Delegating Ctors}
\framesubtitle{(Mostly directly from Stroustrup's website)}
Initialization code common to many constructors
\begin{columns}[t]
\column{.5\textwidth}
{\scriptsize
Old: repeat yourself or call an init() function.
\begin{verbatim}
class X {
  int a;
  validate(int x) { stuff }

 public:

  X(int x) { validate(x); }
  X()      { validate(42); }
  X(string s) { 
    int x = lexical_cast<int>(s);
    validate(x);      }
};
\end{verbatim}
}
\column{.5\textwidth}<2->
{\scriptsize
New: constructors call other constructors

\begin{verbatim}
class X {
  int a;


 public:

  X(int x)  { validate(x);  }
  X()         : X{42} {}
  X(string s) : X{lexical_cast<int>(s)} {}
};
\end{verbatim}
}
\end{columns}

\pause{}

\begin{itemize}
\item Gotchas: none known yet.
\item Only the delegating ctor is allowed in the init list.  Delegate
  or initialize, not both.
\item Lifetime begins when \Emph{any} constructor completes,
  so an exception thrown from a
  delegating constructor will invoke the destructor.

\item Note that delegating to a base class' ctors has always been
supported.  Now we can delegate to our own.

\end{itemize}
\end{frame}

%%%%%%%%%%%%%%%%%%%%%%%%%%%%%%%%%%%%%%%%%%%%%%%%%%%%%%%%%%%%%%%%%%%%%%
%%%%%%%%%%%%%%
\subsection{Inheriting Constructors}

\begin{frame}[fragile]
\frametitle{Inheriting Ctors}

\begin{columns}[t]
\column{.5\textwidth}
A member of a base class is not in the same scope as a member of a derived class:
{\scriptsize
\begin{verbatim}
struct B            {void f(double); };

struct D : public B {void f(int);    };






B b;   b.f(4.5);  // fine

D d;   d.f(4.5);  // surprise: calls f(int)
                  // with argument 4

d.B::f(4.5)  // Explicit scope, now calls
             // f(double)
\end{verbatim}
}
\column{.5\textwidth}<2->
In \CC98, we can "lift" a set of overloaded functions from a base class into a derived class:
{\scriptsize
\begin{verbatim}
struct B {void f(double);  };

struct D : public B {
  using B::f;  // bring all f()s from B
               // into scope

  void f(int); // add a new f()
};

B b;   b.f(4.5); // fine

D d;   d.f(4.5); // fine: calls D::f(double)
                 // which is B::f(double)
\end{verbatim}
}
\end{columns}
\pause{}
\vskip 24pt
"Little more than a historical accident prevents using this for
constructors as well as for ordinary member functions." --Stroustrup
\end{frame}

\begin{frame}[fragile]
\frametitle{Inheriting Ctors II}

Syntax:


{\scriptsize
\begin{verbatim}
struct B
{
    B(int);    // normal constructor 1
    B(string); // normal constructor 2
};

struct D : public B
{
    using B::B; // inherit (all!) constructors from B
};
\end{verbatim}
}
This implicitly defines
{\scriptsize
\begin{verbatim}
D::D(int);    // inherited
D::D(string); // inherited
\end{verbatim}
}
\end{frame}

\begin{frame}[fragile]
\frametitle{Inheriting Ctors III}
Gotcha: inheriting ctors in a derived class with new member variables that
need initialization:

{\scriptsize
\begin{verbatim}
struct B1 {
  B1(int i) : a(i) { }
  int a;
};

struct D1 : public B1 {
  using B1::B1; // implicitly declares D1(int)
  int x;
};

D1 d {6};   // Oops: d.x is not initialized
D1 e;      // error: D1 has no default constructor
\end{verbatim}
}

Inherited ctors don't know about added member data, but you can call
base ctors directly:
{\scriptsize
\begin{verbatim}
struct D1 : public B1 {
  using B1::B1;
  int x;
  D1(int a, int b) : B1(a), x(b) {}
}
\end{verbatim}
}

\end{frame}

\subsection{Virtual Function Controls}

\begin{frame}[fragile]
\frametitle{Override}
\framesubtitle{(straight from the isocpp website)}
Common errors with virtual functions:
\begin{columns}[t]
\column{.5\textwidth}<1->
The problem:
{\scriptsize
\begin{verbatim}
struct B {
    virtual void f();
    virtual void g() const;
    virtual void h(char);
    void k();   // not virtual
};
struct D : B {
    void f();   // overrides B::f()
    void g();   // doesn't override B::g()
                //   (wrong type)
    void h(char); // overrides B::h()

    void k();   // doesn't override B::k(),
                // B::k() is not virtual
};
\end{verbatim}
}
\column{.5\textwidth}<2->
The solution:
{\scriptsize
\begin{verbatim}






struct D : B {
  void f() override;  // OK: overrides B::f()
  void g() override;  // error: wrong type

  void h(char);       // overrides B::h();
                      //   likely warning
  void k() override;  // error: B::k()
                      //   is not virtual
};
\end{verbatim}
}
\end{columns}
\onslide<3->
\begin{itemize}
\item A declaration marked override is only valid if there is a function to override.
\item ``override'' is a contextual keyword, so you can still use it
  as an identifier (but arguably you shouldn't).
\end{itemize}
\begin{center}\Emph{Use everywhere -- this is possibly the first thing
to adopt in \CC11}\end{center}
\end{frame}


\begin{frame}[fragile]
\frametitle{final}
\framesubtitle{(straight from the isocpp website)}
The flip side: how to keep a virtual function from beeing overridden?
\begin{columns}[t]
\column{.5\textwidth}<1->
\Emph{\CC 03: comments...}
{\scriptsize
\begin{verbatim}
struct A { virtual void f(); };

struct B : public A {
  // DO NOT OVERRIDE, THIS MEANS YOU
  virtual void f();
};

struct C : public B {
  // compiler doesn't read comments
  virtual void f();
};
\end{verbatim}
}
\column{.5\textwidth}<2->
\Emph{\CC 11: final keyword}
{\scriptsize
\begin{verbatim}
struct A { virtual void f(); };

struct B : public A {
  virtual void f() final;
};

struct C : public B {
  virtual void f(); // error
};
\end{verbatim}
}
\end{columns}
\pause{}
\begin{itemize}
\item Frequently used to cover OO design error.  Closes off future
  customization... do you really want to do that?
\item final is only a contextual keyword, so you can still use it as
  an identifier (but you probably shouldn't).
\end{itemize}
\end{frame}


\begin{frame}[fragile]
\frametitle{Final}
When to use \texttt{final}?  Almost never.  Use cases are rare.
\begin{itemize}
\item To prohibit a derived class from giving the method a different
  meaning:
{\scriptsize
\begin{verbatim}
struct A { 
            void f(int, double)       {}; 
    virtual void g(int, double) final {}; 
    
};

struct B : public A {
    void f(int, double) {} // hides A::f
    void g(int, double) {} // compile error
};
\end{verbatim}}
\item For opimization (``De-virtualization''), but note that many
  compilers can do this anyway.  (Live demo)
\item For multi-layered NVI patterns (unusual but it does happen)
\end{itemize}
    
\end{frame}


%%%%%%%%%%%%%%%%%%%%%%%%%%%%%%%%%%%%%%%%%%%%%%%%%%%%%%%%%%%%%%%%%%%%%%
%%%%%%%%%%%%%%%%%%%%%%%%%%%%%%%%%%%%%%%%%%%%%%%%%%%%%%%%%%%%%%%%%%%%%%
%%%%%%%%%%%%%%%%%%%%%%%%%%%%%%%%%%%%%%%%%%%%%%%%%%%%%%%%%%%%%%%%%%%%%%


\begin{frame}[fragile]
\frametitle{Final Classes}
When used in a class definition, it prevents inheritance.
{\scriptsize
\begin{verbatim}
struct A {  virtual void f(); };

struct B final : public A { void f() override; };

struct C : public B {};  // error: cannot derive from 'final' base 'B' in derived type 'C'
};
\end{verbatim}
}
Why?
\begin{itemize}
\item efficiency: as with final virtual functions, this can assist the
  compiler in de-virtualizing a function call.
\item safety: ensure that your class is not used as a base class (for
  example, to be sure that you can copy objects without fear of
  slicing).
\end{itemize}

    
\end{frame}


%%%%%%%%%%%%%%%%%%%%%%%%%%%%%%%%%%%%%%%%%%%%%%%%%%%%%%%%%%%%%%%%%%%%%%
%%%%%%%%%%%%%%%%%%%%%%%%%%%%%%%%%%%%%%%%%%%%%%%%%%%%%%%%%%%%%%%%%%%%%%
%%%%%%%%%%%%%%%%%%%%%%%%%%%%%%%%%%%%%%%%%%%%%%%%%%%%%%%%%%%%%%%%%%%%%%
\subsection{Member Function Control}

\begin{frame}[fragile]
\frametitle{Deleted Members}
\framesubtitle{(straight from the isocpp website)}
How to prohibit copying?
\vskip 12pt
\begin{columns}[t]
\column{.5\textwidth}
\Emph{\CC 03: The old way...}
{\scriptsize

\begin{verbatim}
struct A {

 // ...
private:

 A(const A&); // copy ctor
 A& operator=( const A&);
};
\end{verbatim}

Make copy ops private and don't define them
}
\column{.5\textwidth}<2->
\Emph{\CC 11: explicitly delete 'em}
{\scriptsize
\begin{verbatim}
struct A {

 // ...
 A(const A&) = delete;
 A& operator=(const A&) = delete;
};
\end{verbatim}

Explicitly prevent compiler from generating the default implementations.
}
\end{columns}

\end{frame}


\begin{frame}[fragile]
\frametitle{Deleted Members -- existential crisis}
Deleted functions \Emph{do not cease to exist}.  They exist, but it's
an error to call them.
\begin{itemize}
\item They participate in overload lookup:
\pause{}
\item ... given a set of overloads, the compiler considers deleted
  functions.
\item If the result of overload resolution is a deleted member
  function, it's an error
\pause{}
\end{itemize}

\pause{}
\vskip 12pt
This can therefore be used to eliminate undesired converions:
{\scriptsize
\begin{verbatim}
struct Z {
  // ...
  Z(long long);     // can initialize with a long long
  Z(long) = delete; // but not anything smaller
};
\end{verbatim}
}




\end{frame}


\begin{frame}[fragile]
\frametitle{Defaulted Members}
\framesubtitle{(straight from the isocpp website)}
Conversely, sometimes we'd like the compiler to provide a default:
{\scriptsize
\begin{verbatim}
class Y {
    // ...
    Y(const Widget&); // disables compiler-generated defaults

    Y& operator=(const Y&) = default;   // default copy semantics
    Y(const Y&)            = default;
};
\end{verbatim}
}
\pause{}
\begin{itemize}
\item Useful when the compiler-generated defaults have been disabled (by
providing other ctors)
\item Arguably, can increase clarity even when not absolutely needed.
\end{itemize}
\end{frame}


\subsection{Explicit Conversions}

%%%%%%%%%%%%%%%%%%%%%%%%%%%%%%%%%%%%%%%%%%%%%%%%%%%%%%%%%%%%%%%%%%%%%%

\begin{frame}[fragile]
\frametitle{Explicit conversions}
\framesubtitle{isocpp.org}
C++98 has implicit and explicit constructors:
\begin{columns}[t]
\column{.5\textwidth}
{\scriptsize
\begin{verbatim}
struct A {A(int) ;};

void foo (A a);

A a {5}; // OK, explicit constructor call
foo (5); // OK, implicit constructor call
\end{verbatim}
}
\column{.5\textwidth}
{\scriptsize \begin{verbatim}
struct A { explicit A(int) ;};

void foo (A a);

A a {5}; // OK, explicit constructor call
foo (5); // Error
\end{verbatim}
}
\end{columns}

\pause{}
\vskip 6pt
\begin{columns}[t]
\column{.5\textwidth}
{\scriptsize
And implicit conversion ops
\begin{verbatim}
struct A { ... operator int(); };

void foo (int);

A a;

foo(static_cast<int>(a)); // explicit
foo (a);                  // implicit

\end{verbatim}
}
\column{.5\textwidth}
{\scriptsize
\Emph{C++11 adds explicit conversion ops}
\begin{verbatim}
struct A { explicit operator int(); };

void foo (int);

A a;

foo(static_cast<int>(a)); // explicit
foo (a);                  // error


\end{verbatim}
}
\end{columns}

\end{frame}

%%%%%%%%%%%%%%%%%%%%%%%%%%%%%%%%%%%%%%%%%%%%%%%%%%%%%%%%%%%%%%%%%%%%%%

\begin{frame}[fragile]
\frametitle{Explicit conversions II}
\framesubtitle{isocpp.org}


\begin{itemize}
\item Make all conversion operators explicit unless there's good
  reason not to.

  \begin{itemize}
    \item Including single-argument constructors
  \end{itemize}
\item ``Good reason'' means it's fundamental in the application domain
  and frequently needed.
\item Make all conversion to bool (operator bool) explicit, period.
  \Emph{Implicit conversion to bool is very dangerous}
\end{itemize}
\pause{}
Does this print a correct statement in C++98?
\begin{verbatim}
cout << "8+8 = " <<
cout << "6";
\end{verbatim}
\pause{}
One of the scariest bugs I've seen
\vskip 6pt
$A^{-1}BCB^TA + .5/D$
(it was supposed to be .5*D
\pause{}

\Emph{This evaluated to 1.5 }

\end{frame}
